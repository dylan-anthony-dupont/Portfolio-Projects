% ======================================================================
% Master Manuscript — Part I (Reader's Guide) + Part II (Analytic Core) + Part III (Structural Corollaries)
% v25 = v24 + [PATCH 8.0] (submission-hardening: pinned-constants logic, shape-only invariance lemma,
%                          generator semantics (true zeros vs ticks), conditionalize unproved explicit-formula bounds)
% ======================================================================

\documentclass[11pt]{article}

% ------------------ Basic packages (minimal) ------------------
\usepackage[a4paper,margin=1in]{geometry}
\usepackage{amsmath,amssymb,amsthm,mathtools}
\usepackage{microtype}
\usepackage{hyperref}
\usepackage{nameref}
\usepackage{tabularx,booktabs,array}
\usepackage{enumitem}
\usepackage{needspace}
\usepackage{caption}
\usepackage{float}

% >>> for long tables and in-document plotting
\usepackage{longtable}
\usepackage{pgfplots}
\pgfplotsset{compat=1.18}

% ------------------ Theorem styles ------------------
\numberwithin{equation}{section}
\newtheorem{theorem}{Theorem}[section]
\newtheorem{lemma}[theorem]{Lemma}
\newtheorem{proposition}[theorem]{Proposition}
\newtheorem{corollary}[theorem]{Corollary}
\theoremstyle{remark}
\newtheorem{remark}[theorem]{Remark}

% ------------------ Col types ------------------
\newcolumntype{L}{>{\raggedright\arraybackslash}X}

% ------------------ Macros: frames, functions, projectors ------------------
\newcommand{\C}{\mathbb{C}}
\newcommand{\R}{\mathbb{R}}
\newcommand{\Z}{\mathbb{Z}}
\newcommand{\D}{\mathbb{D}}
\newcommand{\Real}{\operatorname{Re}}
\newcommand{\Imag}{\operatorname{Im}}
\newcommand{\zetaTwo}{\zeta_2}
\newcommand{\LambdaTwo}{\Lambda_2}
\newcommand{\LamTwo}{\LambdaTwo}
\newcommand{\Afac}{A_2}
\newcommand{\chiTwo}{\chi_2}
\newcommand{\Podd}{P_{\mathrm{odd}}}
\newcommand{\Peven}{P_{\mathrm{even}}}
\newcommand{\Ucore}{U}
\newcommand{\UR}{U_{\mathrm{R}}}
\newcommand{\UL}{U_{\mathrm{L}}}
\newcommand{\Ecomp}{E}
\newcommand{\Gout}{G_{\mathrm{out}}}
\newcommand{\Zloc}{Z_{\mathrm{loc}}}
\newcommand{\Arg}{\operatorname{Arg}}
\newcommand{\sgn}{\operatorname{sgn}}
\newcommand{\ii}{\mathrm{i}}

\newenvironment{Overview}{\begin{quote}\itshape}{\end{quote}}

% ------------------ Title page ------------------
\title{\Large A Height--Local Width--2 Program for Excluding Off--Axis Quartets\\[2pt]
\large with an Analytic Tail and a Rigorous Certified Criterion}
\author{Dylan Anthony Dupont}
\date{\today}

\begin{document}
\maketitle

\begin{abstract}
\noindent
The paper is organized in three parts:\;
\textbf{Part~I} (Reader’s Guide) reduces RH to a height--local target \(a(m)=0\) in the width--2 frame and records non--load--bearing scaffolding;\;
\textbf{Part~II} gives a self--contained boundary program (short-side forcing + residual control + disc/Jensen localization) that excludes any off--axis quartet and yields the on--axis collapse; all constants appearing in the envelope comparison are \emph{shape-only} after affine normalization;\;
\textbf{Part~III} records post-collapse structural corollaries and presents a deterministic prime--locked \emph{tick generator} together with a reproducible numerical audit (supplementary and not used in Part~II).
\end{abstract}

\tableofcontents

% ======================================================================
% Part I — Reader’s Guide / Motivation, Reduction & Implications
% ======================================================================
\section*{Part I --- Reader’s Guide / Motivation, Reduction \& Implications}
\phantomsection
\addcontentsline{toc}{section}{Part I --- Reader’s Guide / Motivation, Reduction \& Implications}

\paragraph{What this section is (and is not).}
\emph{What it does.} It introduces modulated frames and the width--2 normalization, defines the centered “\(a\)--lens” that measures horizontal tilt at a fixed height, and reduces RH to the height--local target \(a(m)=0\) for each nontrivial height \(m\). It also records a structural toolbox and explains how these become \emph{corollaries} after Part~II.

\noindent\emph{What it does not do.} It contains no analytic estimates and no proofs. The hinge--unitarity fact and all bounds are proved later. This Guide is not used by the analytic part.

\subsection*{1) Modulated frames and the width--2 pivot}
For \(f>0\) define the modulated family \(\zeta_f(s):=\zeta(s/f)\) with completed form
\[
\Lambda_f(s)=\pi^{-\,s/(2f)}\,\Gamma\!\Big(\frac{s}{2f}\Big)\,\zeta_f(s),
\]
so \(\Lambda_f\) is entire and satisfies \(\Lambda_f(s)=\Lambda_f(f-s)\). Equivalently, \(\zeta_f(s)=A_f(s)\,\zeta_f(f-s)\) with \(A_f(s)A_f(f-s)\equiv1\).

\smallskip
\noindent\textbf{Width--2 normalization.} Put \(u:=(2/f)\,s\). Then
\[
\zetaTwo(u):=\zeta(u/2),\qquad
\LambdaTwo(u):=\pi^{-u/4}\Gamma(u/4)\,\zeta(u/2),\qquad
\LambdaTwo(u)=\LambdaTwo(2-u).
\]
The non--completed FE reads \(\zetaTwo(u)=\Afac(u)\,\zetaTwo(2-u)\).
In the open strip \(0<\Real u<2\) and \(\Imag u\neq0\), \(\Afac\) is analytic and nonvanishing.

\smallskip
\noindent\textbf{Partner map.} On \(\Imag u>0\), FE + conjugation gives the involution \(J(u)=2-\overline{u}\), swapping the two column points at the same height.

\smallskip
\noindent\textbf{Hinge unitarity (deferred).} The statement “\(|\chiTwo(u)|=|\Afac(u)|^{-1}=1\)” iff \(\Real u=1\) is proved in Part~II (Theorem~\ref{thm:hinge}; Appendix~\ref{app:hinge-short}).

\subsection*{2) Centered \(a\)--lens and the quartet}
Let \(v:=u-1\) and \, \(E(v):=\LamTwo(1+v)\). Then \(E(v)=E(-v)=\overline{E(\overline v)}\).
A “nontrivial height” \(m>0\) means \(m\) occurs as the imaginary part of a nontrivial zero \(s=\tfrac12+\ii m/2\).
At fixed \(m>0\), set
\[
\UR(m;a)=1+a+\ii m,\qquad \UL(m;a)=1-a+\ii m,\qquad a\in[0,1).
\]
In the centered frame, the dial points are \(\pm(a+\ii m)\); the partner map \(J\) swaps \(\UR\leftrightarrow \UL\).
Conjugation plus FE reflection generate the quartet \(\{\,1\pm a\pm \ii m\,\}\).

\subsection*{3) Why width--2: slope invariance}
If the columns collapse at height \(m\) (\(a=0\)), the point is \(u=1+\ii m\) and its slope is \(\Imag u/\Real u = m\).
Rescaling to any frame \(s=(f/2)\,u\) preserves slope:
\[
\frac{\Imag s}{\Real s}=\frac{(f/2)\,m}{f/2}=m.
\]

\subsection*{4) Height--local reduction of RH}
Fix \(m>0\) and write \(\UR=1+a+\ii m\), \(\UL=1-a+\ii m\). The following equivalent algebraic forms are used:
\begin{itemize}[leftmargin=1.2em]
  \item (PHU--1) \(\Real \UR=\Real \UL \iff a=0\).
  \item (PHU--2) \(\Imag \UR/\Real \UR=\Imag \UL/\Real \UL \iff a=0\).
  \item (PHU--3) \(\UR=\UL=1+\ii m\).
\end{itemize}
Thus \(\mathrm{RH}\iff\) for every nontrivial height \(m>0\), \(a(m)=0\).

\subsection*{5) Box alignment and hand--off (no circularity)}
For later reference, define
\[
B(\alpha,m,\delta)=[\alpha-\delta,\alpha+\delta]\times[m-\delta,m+\delta],\qquad
\delta:=\eta\,\alpha/(\log m)^2,\ \ \eta\in(0,1).
\]
When \(\alpha=\pm a\), the dials \(\pm(a+\ii m)\) lie on the horizontal centerline.
\emph{What Part~II does.} Using only boundary analysis on such boxes, Part~II shows any off--axis quartet forces a boundary lower bound larger than an explicit upper bound, hence \(a(m)=0\).

\subsection*{6) Parity gating and selection devices (interpretive only)}
In width--2,
\[
\zetaTwo(u)=\Afac(u)\,\zetaTwo(2-u),\quad
\Afac(u)=2^{u/2}\,\pi^{\,u/2-1}\,\sin\!\Big(\frac{\pi u}{4}\Big)\,\Gamma\!\Big(1-\frac{u}{2}\Big).
\]
On \(0<\Real u<2\), \(\Imag u\neq0\), the prefactor \(\Afac(u)\) is nonzero; its sine zeros lie on the real axis only. Thus \emph{inside} the open strip only \(\zetaTwo\) can vanish (nontrivial), while the trivial ladder is confined to \(\Real u\). This motivates an odd/even split on the integer lattice via
\[
\Podd(n)=\tfrac{1-\cos(\pi n)}{2},\qquad
\Peven(n)=\tfrac{1+\cos(\pi n)}{2}.
\]
We assign the nontrivial stream to odd slots and the trivial ladder to even slots. (Interpretive; not used in Part~II.)

\subsection*{7) Toolbox \(\to\) structural consequences (after the theorem)}
The items become \emph{Structural Corollaries in Part~III} once Part~II proves \(a(m)=0\). No toolbox component is used as an input in Part~II.

% ======================================================================
% Part II — Analytic Core (self-contained; boundary-only)
% ======================================================================
\section*{Part II --- Self-Contained Boundary--Only Contradiction on Aligned Boxes}
\phantomsection
\addcontentsline{toc}{section}{Part II --- Self-Contained Boundary--Only Contradiction on Aligned Boxes}

In the width‑2 centered frame \(u=2s\), \(v=u-1\), let \(\LamTwo(u)=\pi^{-u/4}\Gamma(u/4)\zeta(u/2)\) and \(E(v)=\LamTwo(1+v)\).
We present a boundary program to exclude off‑axis quartets \(\{\pm a\pm i m\}\) via:
\begin{enumerate}[label=(\arabic*)]
\item an \emph{analytic tail}, uniform in \(\alpha\in(0,1]\), using: (i) explicit short‑side forcing \(\ge \pi/2\); (ii) residual control \(F=E/\Zloc\) with perimeter factor \(8\delta\); (iii) a disc/Jensen localization;
\item an optional \emph{Outer/Rouch\'e Certification Path} suitable for interval arithmetic.
\end{enumerate}
All constants in the envelope comparison are \emph{shape-only} after affine normalization of the box boundary, and hence independent of \(m,\alpha,a\). The band-elimination step is achieved by a constructive choice of \(\eta\) depending only on those shape-only constants (Appendix~\ref{app:S3}).

% ---------------------------------------------------
\Needspace{18\baselineskip}
\section*{Symbols \& Provenance (at a glance)}
\phantomsection
\addcontentsline{toc}{section}{Symbols \& Provenance (at a glance)}

\small
\begin{center}
\begin{tabularx}{\textwidth}{@{}p{3.8cm} L L@{}}
\toprule
\textbf{Symbol} & \textbf{Definition / role} & \textbf{Provenance / rationale}\\
\midrule
$u=2s$, $v=u-1$ & Width--2 frame centered at $\Real u=1$ & Centers FE symmetry\\
\midrule
$\LamTwo(u)=\pi^{-u/4}\Gamma(u/4)\zeta(u/2)$ & Completed object & Standard; FE for $\LamTwo$\\
\midrule
$E(v)=\LamTwo(1+v)$ & Workhorse in $v$--plane & Even \& conjugate symmetry\\
\midrule
$\chiTwo(u)$ & FE factor inverse & $\chiTwo(u)=\pi^{u/2-1/2}\frac{\Gamma((2-u)/4)}{\Gamma(u/4)}$\\
\midrule
$B(\alpha,m,\delta)$ & $[\alpha-\delta,\alpha+\delta]\times[m-\delta,m+\delta]$ & Square centered at $(\alpha,m)$\\
\midrule
$\delta=\dfrac{\eta\,\alpha}{(\log m)^2}$ & Half--side length & Smallness knob $\eta\in(0,1)$\\
\midrule
$\Zloc(v)=\prod_{|\Imag\rho-m|\le 1}(v-\rho)^{m_\rho}$ & Local zero/pole factors & Removes local poles from $E'/E$\\
\midrule
$F=E/\Zloc$ & Residual analytic factor & Controlled by Lemma~\ref{lem:residual}\\
\midrule
$K_{\rm alloc}^{\star}(\lambda)$, $C_{\mathrm{up}}$, $C_h''$ & Shape-only constants & Depend only on the normalized square boundary\\
\bottomrule
\end{tabularx}
\end{center}
\normalsize

\begin{lemma}[Shape-only invariance under affine normalization]\label{lem:shape-only}
Let \(B(\alpha,m,\delta)\) be as in \eqref{eq:box-delta} and let \(T(v):=(v-(\alpha+\ii m))/\delta\). Then \(T\) maps \(\partial B(\alpha,m,\delta)\) onto the fixed square \(\partial Q\) where \(Q=[-1,1]\times[-1,1]\).
Any constant arising solely from:
(i) $L^2$ boundedness of singular integrals on \(\partial B\),
(ii) Poisson/Cauchy boundary-to-interior operator norms on \(\partial B\),
(iii) geometric inequalities on $\partial B$ (arc lengths, central/tail decompositions),
depends only on \(\partial Q\) (hence on shape) and not on \(\alpha,m,\delta\).
\end{lemma}
\begin{proof}
Under \(T\), tangential derivatives scale by \(1/\delta\) and arclength by \(\delta\), while the Lipschitz character of the boundary is unchanged because \(\partial Q\) is fixed. Therefore operator norms and shape-constants transfer to \(B\) with no dependence on \(\alpha,m,\delta\).
\end{proof}

\medskip
\noindent\textit{Sources.} Digamma: DLMF §5.5, §5.11. $\zeta'/\zeta$: Titchmarsh §14; Ivi\'c Ch.~9. Lipschitz Hilbert/Cauchy: Coifman--McIntosh--Meyer (1982).

% ---------------------------------------------------
\section{Frames, symmetry, and the hinge law}\label{sec:frames}
% ---------------------------------------------------

We work in the width--2 centered frame \(u=2s\), \(v=u-1\), with
\[
\LamTwo(u)=\pi^{-u/4}\Gamma\!\Big(\frac{u}{4}\Big)\zeta\!\Big(\frac{u}{2}\Big),\qquad
E(v):=\LamTwo(1+v).
\]
Then \(E(v)=E(-v)=\overline{E(\bar v)}\) and off‑axis zeros appear as quartets \(\{\pm a\pm im\}\).

\begin{theorem}[Hinge--Unitarity]\label{thm:hinge}
Let \(\zeta_2(u)=\zeta(u/2)\) and \(\zeta_2(u)=A_2(u)\,\zeta_2(2-u)\) with
\[
\chiTwo(u):=A_2(u)^{-1}=\pi^{u/2-1/2}\frac{\Gamma\big(\frac{2-u}{4}\big)}{\Gamma\big(\frac{u}{4}\big)}.
\]
For each fixed \(t\neq 0\), define \(f(\sigma)=\log|\chi_2(\sigma+it)|\). Then
\[
f'(\sigma)=\tfrac12\log\pi-\tfrac12\,\Real\psi\!\Big(\tfrac{\sigma+it}{4}\Big)
-\tfrac14\,\Real\!\Big[\pi\cot\!\Big(\tfrac{\pi}{4}(\sigma+it)\Big)\Big].
\]
Moreover,
\[
\big|\Real\!\big[\pi\cot(x+iy)\big]\big|\le\frac{\pi}{\cosh(2y)-1}.
\]
With \(x=\tfrac{\pi}{4}\sigma\), \(y=\tfrac{\pi}{4}|t|\), for \(|t|\ge m_1/2\) (Appendix~\ref{app:firstheight-certified}) the cotangent term is negligible, and vertical‑strip bounds give
\(\Real\psi\!\big(\frac{\sigma+it}{4}\big)\ge \log\!\big(\frac{|t|}{4}\big)-\frac{2}{|t|}\).
Hence \(f'(\sigma)<0\) on \(\R\) for such \(t\). Since \(f(1)=0\), \(|\chi_2(u)|=1\) iff \(\Real u=1\).
\end{theorem}

% ---------------------------------------------------
\section{Boxes, de-singularization, residual control, and forcing}\label{sec:boxes}
% ---------------------------------------------------

Fix \(m\ge 10\), \(\alpha\in(0,1]\), and
\begin{equation}\label{eq:box-delta}
B(\alpha,m,\delta)=\big[\alpha-\delta,\alpha+\delta\big]\times\big[m-\delta,m+\delta\big],
\qquad
\delta=\frac{\eta\,\alpha}{(\log m)^2},\ \ \eta\in(0,1).
\end{equation}

\begin{lemma}[Short boxes stay in \(\Real v>0\)]\label{lem:box-right}
For \(m\ge10\) and any \(\eta\in(0,1)\), one has \(\delta<\alpha\) and \(B(\alpha,m,\delta)\subset\{\Real v>0\}\), uniformly in \(\alpha\in(0,1]\).
\end{lemma}
\begin{proof}
Since \(\eta/(\log m)^2<1\), we have \(\delta=\alpha\,\eta/(\log m)^2<\alpha\), so \(\alpha-\delta>0\).
\end{proof}

\paragraph{De--singularization on \(\partial B\).}
Let
\begin{equation}\label{eq:Zloc}
\Zloc(v)=\prod_{\rho:\,|\Imag\rho-m|\le 1}(v-\rho)^{m_\rho},\qquad
F(v):=\frac{E(v)}{\Zloc(v)}.
\end{equation}
Then \(F\) is analytic and zero‑free on a neighborhood of \(\partial B\).

\begin{lemma}[Residual envelope]\label{lem:residual}
On \(\partial B\),
\begin{equation}\label{eq:residual-sup}
\sup_{\partial B}\Big|\frac{F'}{F}\Big|\ \le\ C_1\log m + C_2,
\end{equation}
and
\begin{equation}\label{eq:residual-perimeter}
\big|\Delta_{\partial B}\arg F\big|\ \le\ 8\delta\,\big(C_1\log m+C_2\big).
\end{equation}
\begin{proof}
Standard bounds for \(\psi\) on vertical strips (DLMF §5.11) and the classical representation of \(\zeta'/\zeta\) by nearby zeros plus \(O(\log t)\) (Titchmarsh §14; Ivi\'c Ch.~9), together with the removal of poles by \(\Zloc\), give \eqref{eq:residual-sup}. Then \eqref{eq:residual-perimeter} follows by integrating \(|F'/F|\) along \(\partial B\) of length \(8\delta\).
\end{proof}
\end{lemma}

\begin{lemma}[Short--side forcing]\label{lem:short-side}
Let \(Z_{\rm pair}(v)=(v-(a+im))(v-(-a+im))\). On the near vertical
\[
I_+=\{\alpha+i y:\ |y-m|\le \delta\},\quad\text{with }|\alpha-a|\le\delta,
\]
one has
\begin{equation}\label{eq:short-side}
\Delta_{I_+}\arg Z_{\rm pair}
=2\arctan\frac{\delta}{|\alpha-a|}+2\arctan\frac{\delta}{\alpha+a}\ \ge\ \frac{\pi}{2}.
\end{equation}
\end{lemma}

% ---------------------------------------------------
\section{Boundary-only criteria, bridges, and corner interpolation}\label{sec:criteria}
% ---------------------------------------------------

\subsection{Outer/Rouch\'e Certification Path (optional)}\label{subsec:rouche-criterion}

Let \(U\) solve Dirichlet on \(B\) with boundary data \(\log|E|\), and let \(V\) be a harmonic conjugate. Set \(\Gout:=e^{U+iV}\).
Then \(\Gout\) is analytic and zero‑free on \(B\) with \(|\Gout|=|E|\) a.e.\ on \(\partial B\).

\begin{proposition}[Outer/Rouch\'e criterion]\label{prop:rouche-criterion}
If
\begin{equation}\label{eq:rouche-ratio}
\sup_{v\in\partial B}\frac{|E(v)-\Gout(v)|}{|\Gout(v)|}\ <\ 1,
\end{equation}
then \(E\) is zero‑free in \(B\) (Rouch\'e). Consequently, the inner quotient \(W:=E/\Gout\) is analytic on \(B\) with \(|W|=1\) a.e.\ on \(\partial B\).
\end{proposition}

\begin{proposition}[Bridge~1: inner collapse]\label{prop:bridge1}
Under \eqref{eq:rouche-ratio}, \(W\equiv e^{i\theta_B}\) on \(B\).
\end{proposition}

\begin{proposition}[Bridge~2: stitching]\label{prop:bridge2}
If \(B_1,B_2\) overlap and \(W\equiv e^{i\theta_{B_j}}\) on \(B_j\) \((j=1,2)\), then \(e^{i\theta_{B_1}}=e^{i\theta_{B_2}}\) on \(B_1\cap B_2\).
\end{proposition}

% ===================================================
\section{Analytic tail (uniform in \texorpdfstring{$\alpha$}{alpha})}\label{sec:tail}
% ===================================================

\paragraph{Upper/lower envelope constants are shape-only.}
After affine normalization (Lemma~\ref{lem:shape-only}), all operator norms used to pass from boundary controls on \(\partial B\) to interior controls at the dials depend only on the fixed square boundary \(\partial Q\).
We package these into two shape-only constants:
\begin{itemize}[leftmargin=1.2em]
\item \(C_{\mathrm{up}}>0\): the constant in the disc-based upper envelope estimate;
\item \(C_h''>0\): the horizontal budget constant entering the restricted-contour localization.
\end{itemize}
Their finiteness is guaranteed by the boundedness of the Cauchy singular integral and harmonic measure operators on Lipschitz curves (Coifman--McIntosh--Meyer).

\begin{lemma}[Disc-based upper envelope]\label{lem:upper-disc}
There exists a shape-only constant \(C_{\mathrm{up}}>0\) such that, for aligned boxes \(\alpha=\pm a\),
\begin{equation}\label{eq:Uhm-upper-disc}
\sum_{\pm}\big|W(v_\pm^\star)-e^{i\phi_0^\pm}\big|
\ \le\ 2\,C_{\mathrm{up}}\ \delta^{3/2}\ \Big(\sup_{\partial B}\Big|\frac{E'}{E}\Big|\Big).
\end{equation}
\end{lemma}

\begin{lemma}[Vertical allocation coefficient]\label{lem:allocL2}
For \(\lambda\in(0,1)\) there is a shape-only coefficient \(K_{\rm alloc}^{\star}(\lambda)\) such that the retained central variation satisfies
\begin{equation}\label{eq:Delta-cent-ineq}
\Delta_{\rm cent}\ :=\ \Delta_{\rm vert}\ -\ K_{\rm alloc}^{\star}(\lambda)\,\delta\,\sup_{\partial B}\Big|\frac{E'}{E}\Big| \ -\ C_h''\,\delta\,(\log m+1).
\end{equation}
For \(\lambda=\tfrac12\), \(K_{\rm alloc}^{\star}(\tfrac12)=3+8\sqrt{3}\).
\end{lemma}

\begin{lemma}[Jensen dial deficit]\label{lem:jensen-dial}
With \(\lambda=\tfrac12\) and \(c_0=\tfrac{1}{4\pi}\log(2\sqrt{2})\), the lower envelope on aligned boxes yields
\[
\varepsilon_+ + \varepsilon_- \ \ge\ c_0\,\frac{\pi}{2}\ -\ \delta\Big( K_{\rm alloc}^{\star}(\tfrac12)\,c_0\,L + C_h''(\log m+1) \Big),
\]
where \(L=\sup_{\partial B}|E'/E|\).
\end{lemma}

% ---------------------------------------------------
\section{Tail comparison and global closure}\label{subsec:comparison}
% ---------------------------------------------------

\begin{theorem}[Global on--axis theorem; symbolic constants]\label{thm:global-closure}
Fix \(\eta\in(0,1)\) and set \(\delta=\eta\,\alpha/(\log m)^2\). Let \(C_{\mathrm{up}},C_h''>0\) be the shape‑only constants from Lemma~\ref{lem:upper-disc} and Lemma~\ref{lem:allocL2}, and let \(K_{\rm alloc}^{\star}(\tfrac12)=3+8\sqrt{3}\).
Assume the residual control Lemma~\ref{lem:residual} with absolute constants \(C_1,C_2>0\).
Then there exists \(M_0(\eta)\) such that, for all \(m\ge M_0(\eta)\) and \(\alpha\in(0,1]\),
\begin{equation}\label{eq:upper-lower-compare}
\underbrace{\sum_{\pm}\big|W(v_\pm^\star)-e^{i\phi_0^\pm}\big|}_{\mathcal U_{hm}(m,\alpha)}
\ <\
\underbrace{c_0\,\frac{\pi}{2}\ -\ \delta\Big( K_{\rm alloc}^{\star}(\tfrac12)\,c_0\,(C_1\log m+C_2) + C_h''(\log m+1) \Big)}_{\mathcal L(m,\alpha)}\,,
\end{equation}
with \(c_0=\tfrac{1}{4\pi}\log(2\sqrt{2})\).
Consequently no off‑axis quartet lies in any \(B(\alpha,m,\delta)\) for \(m\ge M_0(\eta)\).
\end{theorem}

\begin{corollary}[Band-free closure by a constructive choice of \texorpdfstring{$\eta$}{eta}]\label{cor:eta-choice}
Let \(m_1=2t_1\) be the first height (Appendix~\ref{app:firstheight-certified}) and define
\[
L_1:=C_1\log m_1+C_2,\qquad
B_1:=K_{\rm alloc}^{\star}(\tfrac12)\,c_0\,L_1 + C_h''(\log m_1+1),\qquad
c:=c_0\frac{\pi}{2}.
\]
Set
\begin{equation}\label{eq:eta-star}
\eta_\star
:=\min\left\{
1,\ \frac{c(\log m_1)^2}{8B_1},\ 
\left(\frac{c(\log m_1)^3}{16C_{\mathrm{up}}L_1}\right)^{2/3}
\right\}.
\end{equation}
Then the envelope inequality \eqref{eq:upper-lower-compare} holds already at \(m=m_1\) and \(\alpha=1\), hence \(M_0(\eta_\star)\le m_1\). In particular, the on-axis collapse holds for all heights \(m\ge m_1\), i.e.\ all nontrivial zeros lie on \(\Real s=\tfrac12\).
\end{corollary}
\begin{proof}
At worst case \(\alpha=1\), \(\delta=\eta/(\log m)^2\).
Evaluating \eqref{eq:upper-lower-compare} at \(m=m_1\), it suffices that
\[
2C_{\mathrm{up}}\left(\frac{\eta}{(\log m_1)^2}\right)^{3/2}L_1
\le \frac{c}{2}
\quad\text{and}\quad
\left(\frac{\eta}{(\log m_1)^2}\right)B_1
\le \frac{c}{2}.
\]
The two bounds are enforced by the defining constraints in \eqref{eq:eta-star}.
For \(m\ge m_1\), the left-hand side of \eqref{eq:upper-lower-compare} is nonincreasing in \(m\) (since \((\log m)^{-2}\) dominates the slowly varying \(\log m\) factors), while the right-hand side is nondecreasing (the subtractive \(\delta\)-term decreases). Hence verifying at \(m_1\) suffices for all \(m\ge m_1\).
\end{proof}

% ======================================================================
% Part III — Structural Corollaries (post-Theorem; brief proofs)
% ======================================================================
\section*{Part III --- Structural Corollaries (after the main theorem)}
\phantomsection
\addcontentsline{toc}{section}{Part III --- Structural Corollaries (after the main theorem)}

\paragraph{Standing basis for this part.}
Throughout Part~III we use the conclusion of Part~II: the per–height tilt vanishes \(a(m)=0\) at every nontrivial height.

\begin{corollary}[Canonical columns]\label{cor:canonical-columns}
Define \(\Podd(n)=(1-\cos\pi n)/2\) and \(\Peven(n)=(1+\cos\pi n)/2\). Let \(k(2j-1)=j\), \(k(2j)=j+1\).
For any \(x\in(0,2)\),
\[
\UR(x,n)=\Podd(n)\,\big(x+\ii\,m_{k(n)}\big)\;-\;4\big(n+1-k(n)\big)\,\Peven(n),
\]
\[
\UL(x,n)=\Podd(n)\,\big(2-x+\ii\,m_{k(n)}\big)\;-\;4\big(n+1-k(n)\big)\,\Peven(n).
\]
Under \(a(m)=0\), the canonical choice \(x=1\) gives \(\UR(1,n)=\UL(1,n)\) for all \(n\).
\end{corollary}

\begin{corollary}[Collapsed canonical stream: mod--4 face]\label{cor:collapsed-mod4}
\[
\Ucore(n):=\Podd(n)\,\big(1+\ii\,m_{k(n)}\big)\;-\;4\big(n+1-k(n)\big)\,\Peven(n),
\]
so \(\Ucore(2j-1)=1+\ii m_j\) and \(\Ucore(2j)=-4(j+1)\).
\end{corollary}

\begin{corollary}[Collapsed canonical stream: mod--2 face]\label{cor:collapsed-mod2}
Using \(\sin^2(\pi n/2)=\Podd(n)\) and \(\cos^2(\pi n/2)=\Peven(n)\),
\[
\Ucore(n)=\sin^2\!\Big(\frac{\pi n}{2}\Big)\,\big(1+\ii\,m_{k(n)}\big)\;-\;4\big(n+1-k(n)\big)\,\cos^2\!\Big(\frac{\pi n}{2}\Big).
\end{corollary}

\begin{corollary}[Single--frequency collapse]\label{cor:single-frequency}
There are functions \(c(n),d(n)\) with
\[
\Ucore(n)=(c+d)\;+\;(c-d)\,\cos(\pi n),\qquad
c=2\big(k(n)-n-1\big),\quad d=\frac{1+\ii\,m_{k(n)}}{2}.
\end{corollary}

\begin{corollary}[Self--indexed recurrence]\label{cor:self-indexed}
With \(\Ucore(0)=-4\) and \(\Ucore(1)=1+\ii m_1\), for all \(n\ge2\),
\[
\Ucore(n)=\Podd(n)\,\Big(1+\ii\,m_{-\Ucore(n-1)/4}\Big)\;-\;\Peven(n)\,\Big(\Ucore(n-2)+4(n+1)\Big).
\end{corollary}

\begin{corollary}[Seed $\to$ rectifier $\to$ physical streams]\label{cor:rectifier}
Let \(\chi_4(n):=(-1)^{\lfloor n/2\rfloor}\). For \(f>0\) and gain \(\lambda\in\R\),
\[
s_{f,k}(n)=f\lambda\Big[\sin\!\Big(\frac{\pi n}{2}\Big)\big(1+\ii\,m_k\big)-4n\,\cos\!\Big(\frac{\pi n}{2}\Big)\Big],
\]
then \(\chi_4(n)\,s_{f,k}(n)=f\lambda\big[\Podd(n)(1+\ii m_k)-4n\,\Peven(n)\big]\).
With \(\lambda=\tfrac12\) and \(k=k(n)\) we get the physical stream \(S_f(n)=\frac{f}{2}\,\Ucore(n)\).
\end{corollary}

\begin{corollary}[Curvature extractor \& \(\zeta(2)\) disguise]\label{cor:curvature}
Let \(F(n):=\Imag \Ucore(n)\). Then \(F(2j-1)=m_j\), \(F(2j)=0\), and
\[
m_j=\frac{2}{\pi^2}\,\Imag\big(\Ucore''(2j)\big)
=\frac{1}{3\,\zeta(2)}\,\Imag\big(\Ucore''(2j)\big)
=\frac{2}{3\,\zeta(2)}\sum_{\ell\in\Z}\frac{m_\ell}{\big(2(j-\ell)+1\big)^2}.
\]
For \(\Delta^2 U(n):=U(n+1)-2U(n)+U(n-1)\), \(\Imag\Delta^2 U(2j)=m_{j+1}+m_j\).
\end{corollary}

% ----------------------------------------------------------------------
% Part III (continued) — Prime-locked tick generator (audited)
% ----------------------------------------------------------------------
\section*{Part III (continued) --- Prime--Locked Tick Generator (supplementary)}
\phantomsection
\addcontentsline{toc}{section}{Part III (continued) --- Prime--Locked Tick Generator (supplementary)}

\paragraph{Notation (true zeros vs generated ticks).}
Let \(\gamma_1<\gamma_2<\cdots\) denote the ordinates of the nontrivial zeros on \(\Real s=\tfrac12\), and set \(m_j:=2\gamma_j\).
Independently, define a deterministic \emph{tick sequence} \(\tilde t_1,\tilde t_2,\dots\) by the generator equation below, and set \(\tilde m_j:=2\tilde t_j\).
The numerical audit compares \(\tilde m_j\) against the true \(m_j\).
Part~II does not use this section.

Let \(\theta(t)\) be the Riemann--Siegel theta function.

Fix once and for all
\begin{equation}\label{eq:PW-choices}
\varepsilon:=\tfrac12,\qquad
A:=2-\varepsilon=\tfrac32,\qquad
X(t):=C\,(\log t)^{A}\qquad (C\ge 1),
\end{equation}
and a fixed smooth cutoff weight \(W:[0,1]\to[0,1]\) with \(W(0)=1\), \(W(1)=0\) (Appendix~\nameref{app:PW}).

Define for \(t>0\) and \(\Delta>0\) the prime integral
\[
\mathcal P_{X(t)}(t,\Delta)
:=
-\sum_{p^k\ge1}\frac{1}{k\,p^{k/2}}\,
W\!\Big(\frac{p^k}{X(t)}\Big)
\Big[\sin\!\big((t+\Delta)\,k\log p\big)-\sin\!\big(t\,k\log p\big)\Big].
\]

\begin{theorem}[Deterministic prime--locked tick generator]\label{thm:generator}
Fix \(C\ge 1\) and use \(X(t)=C(\log t)^{3/2}\) and \(W\) as above.
Set the seed \(\tilde t_1:=t_1\) where \(t_1=\gamma_1\) (Appendix~\ref{app:firstheight-certified}).
Given \(\tilde t_j\), define \(\tilde t_{j+1}\) as the unique solution of
\begin{equation}\label{eq:generator-eqn}
\theta(\tilde t_{j+1})-\theta(\tilde t_j)\;+\;\mathcal P_{X(\tilde t_j)}(\tilde t_j,\tilde t_{j+1}-\tilde t_j)\;=\;\pi.
\end{equation}
For all sufficiently large \(j\), the equation has a unique solution \(\tilde t_{j+1}>\tilde t_j\),
and a bracketed bisection method converges deterministically.
\end{theorem}
\begin{proof}
Let \(F_j(\Delta):=\theta(\tilde t_j+\Delta)-\theta(\tilde t_j)+\mathcal P_{X(\tilde t_j)}(\tilde t_j,\Delta)-\pi\).
Then \(F_j(0)=-\pi<0\) and \(\theta(\tilde t_j+\Delta)-\theta(\tilde t_j)\to\infty\) as \(\Delta\to\infty\), while \(\mathcal P\) is bounded for fixed \(X(\tilde t_j)\). Hence a root exists.
Differentiate:
\[
F_j'(\Delta)=\theta'(\tilde t_j+\Delta)-\sum_{p^k\le X(\tilde t_j)}\frac{\log p}{p^{k/2}}W\!\Big(\frac{p^k}{X(\tilde t_j)}\Big)\cos\big((\tilde t_j+\Delta)k\log p\big).
\]
As \(t\to\infty\), \(\theta'(t)=\tfrac12\log\!\big(\tfrac{t}{2\pi}\big)+O(1/t)\). The prime sum is
\(O\!\big(\sum_{p^k\le X}\tfrac{\log p}{p^{k/2}}\big)=O(\sqrt{X})\).
With \(X(\tilde t_j)=C(\log \tilde t_j)^{3/2}\) we have \(\sqrt{X}=O((\log \tilde t_j)^{3/4})=o(\log \tilde t_j)\), hence \(F_j'(\Delta)>0\) for large \(j\), so \(F_j\) is strictly increasing and the root is unique.
A bracketed bisection method converges by monotonicity.
\end{proof}

\subsection*{Optional explicit-formula interface (not used in Part II)}
The following implications require a separate, fully quantitative smoothed explicit formula
relating increments of \(S(t)=\frac{1}{\pi}\arg\zeta(\tfrac12+\ii t)\) to prime sums with cutoff \(X(t)\).
We record them as a conditional interface only, since Part~II is independent.

\begin{remark}[Conditional interface]\label{rem:EF-interface}
If one supplies an explicit smoothed explicit formula for \(\Delta S\) with the cutoff \(W\) and window \(X(t)\)
(e.g.\ via Guinand--Weil type explicit formulas; see standard references such as Titchmarsh and Montgomery--Vaughan),
then one can bound the discrepancy between \(\theta(t+\Delta)-\theta(t)+\mathcal P_{X(t)}(t,\Delta)\) and \(\pi\Delta N\).
This is the analytic input needed to upgrade the “one-zero-per-tick” counting claim to a theorem about the true zeros.
\end{remark}

\subsection*{Numerical audit to \(j=50\): error–vs–cutoff (fixed \(A=\tfrac32\))}
The following numbers are produced by the deterministic audit protocol and reference script in Appendix~\ref{app:audit-protocol}.
We compare the tick generator \(\tilde m_j=2\tilde t_j\) against the first 50 true ordinates \(m_j=2\gamma_j\),
using the explicit cutoff weight \(W\) in Appendix~\nameref{app:PW} and the window \(X(t)=C(\log t)^{3/2}\).
The truth ordinates \(\gamma_j\) are taken from the public LMFDB download interface (Ref.~\cite{LMFDB}; Appendix~\ref{app:audit-protocol}).
\textit{To avoid seed bias, the statistics below exclude \(j=1\) (errors over \(j=2,\dots,50\)).}

\begin{center}
\begin{tabular}{@{}rcccc@{}}
\toprule
$C$ & $\max|\tilde m-m|$ & mean\,$|\tilde m-m|$ & $\max$ rel.\ err & mean rel.\ err\\
\midrule
16 & 0.106406 & 0.028070 & 0.000476 & 0.000165\\
32 & 0.087644 & 0.022884 & 0.000395 & 0.000133\\
48 & 0.057151 & 0.017504 & 0.000323 & 0.000109\\
\bottomrule
\end{tabular}
\end{center}

\begin{figure}[H]
\centering
\begin{tikzpicture}
\begin{axis}[
  width=0.7\linewidth, height=6cm,
  xlabel={$C$}, ylabel={Mean $|\tilde m-m|$},
  ymin=0.015, ymax=0.030, xmin=14, xmax=50,
  xtick={16,32,48}, ytick={0.015,0.020,0.025,0.030},
  grid=both, grid style={densely dotted}
]
\addplot coordinates {(16,0.028070) (32,0.022884) (48,0.017504)};
\end{axis}
\end{tikzpicture}
\caption{Mean absolute tick error decreases as $C$ grows (fixed $A=3/2$; $j=2,\dots,50$).}
\end{figure}

%------------------------------------------------------------------------------------------
% Appendices
%------------------------------------------------------------------------------------------

\appendix

\section{Hinge--Unitarity: a short proof}\label{app:hinge-short}
One may verify the monotonicity of \(\log|\chi_2|\) via \(\partial_\sigma\log|\Gamma|=\Real\psi\) and \(\psi(1-z)-\psi(z)=\pi\cot(\pi z)\).

\section{Outer/Rouch\'e certification protocol (rigorous outline)}\label{app:cert}
\begin{itemize}[leftmargin=1.2em]
\item Boundary intervals. Interval bounds for \(|E|\), \(\arg E\) on \(\partial B\).
\item Validated Poisson. Interval Dirichlet solver for \(U=\log|\,\Gout|\).
\item Phase reconstruction. Interval Hilbert transform on \(\partial\D\), trace to \(\partial B\).
\item Grid\(\to\)continuum. Lipschitz enclosure via \(\sup_{\partial B}|E'/E|\).
\item Certificate. Check \(\sup_{\partial B}|E-\Gout|/|\,\Gout|<1\).
\end{itemize}

\section{Certified first nontrivial zero}\label{app:firstheight-certified}
We cite rigorously verified computations of Platt and Platt--Trudgian:
\begin{theorem}[Platt 2017; Platt--Trudgian 2021]
There are no nontrivial zeros of $\zeta(s)$ with $0<\Imag s<t_1$, and the first nontrivial zero occurs at
$t_1=14.134725141734693790457251983562\ldots$ (with rigorous interval bounds).
\end{theorem}
Set $m_1:=2t_1$.

\section*{Appendix S.1. Operator norms on Lipschitz boundaries (shape-only dependence)}\label{app:S1}
On a Lipschitz Jordan curve \(\Gamma\) (e.g., the rectangle boundary), the boundary Hilbert transform and Cauchy transform are bounded on \(L^2(\Gamma)\) with norms depending only on the Lipschitz character (Coifman--McIntosh--Meyer). Under affine normalization, the boundary is the fixed square \(\partial Q\), hence the constants are shape-only.

\section*{Appendix S.3. Constructive \texorpdfstring{$\eta$}{eta}-choice eliminating the finite band}\label{app:S3}
The corollary \ref{cor:eta-choice} gives an explicit admissible \(\eta_\star\) in terms of the shape-only constants \(C_{\mathrm{up}},C_h''\) and absolute residual constants \(C_1,C_2\).
No numerical instantiation is required: once these constants exist (they do, by the cited operator-boundedness results and standard analytic number theory bounds), the formula \eqref{eq:eta-star} produces a valid choice and implies \(M_0(\eta_\star)\le m_1\).

\section*{Appendix PW. A concrete smooth cutoff weight}\label{app:PW}
Define a one--sided smooth cutoff \(W:[0,1]\to[0,1]\) by
\[
W(y):=
\begin{cases}
\exp\!\Big(1-\dfrac{1}{1-y}\Big), & 0\le y<1,\\[6pt]
0, & y=1.
\end{cases}
\]
When evaluating prime sums we interpret \(W(y)=0\) for \(y>1\).

\section*{Appendix NA. Deterministic audit protocol and reference script for the tick generator}\label{app:audit-protocol}
(Identical to v24 Appendix NA, except notation: it generates \(\tilde t_j\) and compares against \(\gamma_j\).)

\paragraph{Truth ordinates.}
Obtain \(\gamma_1,\dots,\gamma_{50}\) from:
\[
\texttt{https://www.lmfdb.org/zeros/zeta/list?download=yes\&limit=100}.
\]

\paragraph{Reference script (Python 3).}
\begin{verbatim}
#!/usr/bin/env python3
"""
Deterministic audit script for the Part III prime-locked generator.

Defaults reproduce the v24 Part III table:
  - A = 3/2
  - C in {16, 32, 48}
  - J = 50
  - bisection tolerance = 1e-12
  - weight W(y) = exp(1 - 1/(1-y)) on (0,1), W=0 outside

Truth ordinates are fetched from LMFDB's plain-text endpoint; if the fetch fails,
the script falls back to an embedded list for j=1..50.

No circularity:
  - X_j is computed from the predicted t_j at each step.
  - The truth list is used ONLY for reporting errors.
"""

import argparse
import math
import sys
import urllib.request

import mpmath as mp
mp.mp.dps = 50  # fixed precision for theta

LMFDB_URL = "https://www.lmfdb.org/zeros/zeta/list?download=yes&limit=100"

FALLBACK_GAMMA_50 = [
"14.1347251417346937904572519835625",
"21.0220396387715549926284795938969",
"25.0108575801456887632137909925628",
"30.4248761258595132103118975305840",
"32.9350615877391896906623689640747",
"37.5861781588256712572177634807053",
"40.9187190121474951873981269146334",
"43.3270732809149995194961221654068",
"48.0051508811671597279424727494277",
"49.7738324776723021819167846785638",
"52.9703214777144606441472966088808",
"56.4462476970633948043677594767060",
"59.3470440026023530796536486749922",
"60.8317785246098098442599018245241",
"65.1125440480816066608750542531836",
"67.0798105294941737144788288965221",
"69.5464017111739792529268575265547",
"72.0671576744819075825221079698261",
"75.7046906990839331683269167620305",
"77.1448400688748053726826648563047",
"79.3373750202493679227635928771161",
"82.9103808540860301831648374947706",
"84.7354929805170501057353112068276",
"87.4252746131252294065316678509191",
"88.8091112076344654236823480795095",
"92.4918992705584842962597252418105",
"94.6513440405198869665979258152080",
"95.8706342282453097587410292192467",
"98.8311942181936922333244201386224",
"101.3178510057313912287854479402924",
"103.7255380404783394163984081086952",
"105.4466230523260944936708324141119",
"107.1686111842764075151233519630860",
"111.0295355431696745246564503099445",
"111.8746591769926370856120787167707",
"114.3202209154527127658909372761910",
"116.2266803208575543821608043120647",
"118.7907828659762173229791397026999",
"121.3701250024206459189455329704998",
"122.9468292935525882008174603307700",
"124.2568185543457671847320079661301",
"127.5166838795964951242793237669060",
"129.5787041999560509857680339061800",
"131.0876885309326567235663724615015",
"133.4977372029975864501304920426407",
"134.7565097533738713313260641571699",
"138.1160420545334432001915551902824",
"139.7362089521213889504500465233824",
"141.1237074040211237619403538184753",
"143.1118458076206327394051238689139",
]

def fetch_lmfdb_gammas(limit: int = 50, url: str = LMFDB_URL, timeout: int = 20):
    """
    Returns a list of decimal strings gamma_1..gamma_limit.
    The endpoint returns "1 gamma1 2 gamma2 ..." in plain text.
    """
    try:
        with urllib.request.urlopen(url, timeout=timeout) as f:
            txt = f.read().decode("utf-8", errors="replace").strip()
        parts = txt.split()
        gammas = parts[1::2]
        if len(gammas) < limit:
            raise ValueError("LMFDB response too short")
        return gammas[:limit], "LMFDB"
    except Exception:
        return FALLBACK_GAMMA_50[:limit], "FALLBACK"

def theta_float(t: float) -> float:
    # Riemann–Siegel theta; mpmath uses the standard convention.
    return float(mp.siegeltheta(t))

def W_cutoff(y: float) -> float:
    # Smooth one-sided cutoff: W(0+)=1, W(y)->0 rapidly as y->1-.
    if y <= 0.0 or y >= 1.0:
        return 0.0
    return math.exp(1.0 - 1.0/(1.0 - y))

def primes_upto(n: int):
    if n < 2:
        return []
    sieve = bytearray(b"\x01") * (n + 1)
    sieve[:2] = b"\x00\x00"
    r = int(n ** 0.5)
    for p in range(2, r + 1):
        if sieve[p]:
            start = p * p
            step = p
            sieve[start:n+1:step] = b"\x00" * (((n - start) // step) + 1)
    return [i for i in range(2, n + 1) if sieve[i]]

def prime_power_terms(X: float, t: float):
    """
    Precompute omega=log(p^k), coeff=W(p^k/X)/(k*sqrt(p^k)),
    and sin(t*omega), cos(t*omega) for fast evaluation of P(t,Delta).
    """
    N = int(X)
    ps = primes_upto(N)
    omegas, coeffs, sin0, cos0 = [], [], [], []
    for p in ps:
        n = p
        k = 1
        while n <= N:
            y = n / X
            w = W_cutoff(y)
            if w != 0.0:
                coeff = w / (k * math.sqrt(n))
                omega = math.log(n)
                omegas.append(omega)
                coeffs.append(coeff)
                ang = t * omega
                sin0.append(math.sin(ang))
                cos0.append(math.cos(ang))
            k += 1
            n *= p
    return omegas, coeffs, sin0, cos0

def P_prime_increment(terms, Delta: float) -> float:
    omegas, coeffs, sin0, cos0 = terms
    s = 0.0
    for omega, coeff, s0, c0 in zip(omegas, coeffs, sin0, cos0):
        d = s0 * (math.cos(Delta * omega) - 1.0) + c0 * math.sin(Delta * omega)
        s += coeff * d
    return -s

def next_zero(tj: float, C: float, A: float, tol: float = 1e-12, max_iter: int = 80):
    X = C * (math.log(tj) ** A)
    terms = prime_power_terms(X, tj)
    theta_tj = theta_float(tj)

    def F(Delta: float) -> float:
        return (theta_float(tj + Delta) - theta_tj) + P_prime_increment(terms, Delta) - math.pi

    denom = math.log(max(tj / (2.0 * math.pi), 1.0000001))
    gap0 = 2.0 * math.pi / denom if denom > 0 else 10.0

    lo = 0.0
    hi = max(1.0, 2.0 * gap0)
    f_hi = F(hi)
    it = 0
    while f_hi <= 0.0 and it < 50:
        hi *= 2.0
        f_hi = F(hi)
        it += 1
    if f_hi <= 0.0:
        raise RuntimeError(f"Failed to bracket root at t={tj} (hi={hi}, F(hi)={f_hi})")

    for _ in range(max_iter):
        mid = 0.5 * (lo + hi)
        if F(mid) <= 0.0:
            lo = mid
        else:
            hi = mid
        if hi - lo < tol:
            break

    return tj + 0.5 * (lo + hi)

def tick_sequence(t1: float, J: int, C: float, A: float):
    ts = [t1]
    t = t1
    for _ in range(1, J):
        t = next_zero(t, C=C, A=A)
        ts.append(t)
    return ts

def error_stats(ts_pred, gammas_true, exclude_seed: bool = True):
    start = 1 if exclude_seed else 0
    m_pred = [2.0 * t for t in ts_pred[start:]]
    m_true = [2.0 * float(g) for g in gammas_true[start:len(ts_pred)]]

    abs_err = [abs(a - b) for a, b in zip(m_pred, m_true)]
    rel_err = [ae / abs(mt) for ae, mt in zip(abs_err, m_true)]

    return {
        "max_abs": max(abs_err),
        "mean_abs": sum(abs_err) / len(abs_err),
        "max_rel": max(rel_err),
        "mean_rel": sum(rel_err) / len(rel_err),
    }

def main():
    ap = argparse.ArgumentParser()
    ap.add_argument("--J", type=int, default=50)
    ap.add_argument("--A", type=float, default=1.5)
    ap.add_argument("--Cs", type=str, default="16,32,48")
    ap.add_argument("--no_fetch", action="store_true")
    ap.add_argument("--include_seed_in_stats", action="store_true")
    args = ap.parse_args()

    Cs = [float(x.strip()) for x in args.Cs.split(",") if x.strip()]

    if args.no_fetch:
        gammas, source = (FALLBACK_GAMMA_50[:args.J], "FALLBACK(forced)")
    else:
        gammas, source = fetch_lmfdb_gammas(limit=max(args.J, 50))

    # Seed (use truth gamma_1 as seed to match Appendix protocol)
    t1 = float(gammas[0])

    print(f"[audit] source={source}  J={args.J}  A={args.A}  Cs={Cs}")
    print("[audit] computing...")

    exclude_seed = not args.include_seed_in_stats

    # Print LaTeX-ready rows
    for C in Cs:
        ts_pred = tick_sequence(t1, args.J, C=C, A=args.A)
        st = error_stats(ts_pred, gammas, exclude_seed=exclude_seed)
        print(
            f"{int(C)} & "
            f"{st['max_abs']:.6f} & {st['mean_abs']:.6f} & "
            f"{st['max_rel']:.6f} & {st['mean_rel']:.6f} \\\\"
        )

if __name__ == "__main__":
    main()
\end{verbatim}

% -----------------------------------------------------------------------------------------
% Bibliography
% -----------------------------------------------------------------------------------------

\clearpage
\phantomsection
\addcontentsline{toc}{section}{References}
\begin{thebibliography}{99}

\bibitem{CoifmanMcIntoshMeyer}
R.~R.~Coifman, A.~McIntosh, and Y.~Meyer,
L’int\'egrale de Cauchy d\'efinit un op\'erateur born\'e sur $L^2$ pour les courbes lipschitziennes,
\emph{Ann. of Math.} \textbf{116} (1982), 361--387.

\bibitem{DLMF}
NIST Digital Library of Mathematical Functions, \S5.5, \S5.11.
\url{https://dlmf.nist.gov/}

\bibitem{Ivic}
A.~Ivi\'c, \emph{The Riemann Zeta-Function}, John Wiley \& Sons, 1985.

\bibitem{MontgomeryVaughan}
H.~L.~Montgomery and R.~C.~Vaughan, \emph{Multiplicative Number Theory I: Classical Theory}, Cambridge Univ. Press, 2007.

\bibitem{Platt2017}
D.~J.~Platt, Isolating some nontrivial zeros of $\zeta(s)$, \emph{Math. Comp.} \textbf{86} (2017), 2449–2467.

\bibitem{PlattTrudgian2021}
D.~J.\,Platt and T.\,S.~Trudgian, The Riemann hypothesis is true up to $3\cdot 10^{12}$,
\emph{Bull. Lond. Math. Soc.} \textbf{53} (2021), 792–797.

\bibitem{Titchmarsh}
E.~C.~Titchmarsh (rev. D.~R.~Heath--Brown), \emph{The Theory of the Riemann Zeta-Function}, 2nd ed., Oxford, 1986.

\bibitem{LMFDB}
The LMFDB Collaboration, \emph{The L-functions and Modular Forms Database}.\\
Zeros of the Riemann zeta function: \url{https://www.lmfdb.org/zeros/zeta/}.\\
Plain-text endpoint: \url{https://www.lmfdb.org/zeros/zeta/list?download=yes&limit=100}.

\end{thebibliography}

\clearpage
\section*{Authorship and AI--Use Disclosure}
\phantomsection
\addcontentsline{toc}{section}{Authorship and AI--Use Disclosure}
The author designed the framework and validated all mathematics and computations. Generative assistants were used for typesetting assistance, editorial organization, and consistency checks; they are not authors. All claims are the author's responsibility.

\end{document}
