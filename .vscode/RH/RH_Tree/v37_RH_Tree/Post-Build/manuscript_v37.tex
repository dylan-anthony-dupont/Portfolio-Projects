\documentclass[11pt]{article}

\usepackage{amsmath,amssymb,amsthm}
\usepackage{geometry}
\usepackage{hyperref}
\usepackage{booktabs}
\usepackage{longtable}
\usepackage{listings}
\usepackage{xcolor}
\usepackage{enumitem}

\geometry{margin=1in}

% --- Listings (Python/JSON) ---
\lstdefinestyle{auditcode}{
  basicstyle=\ttfamily\footnotesize,
  columns=fullflexible,
  breaklines=true,
  keepspaces=true,
  showstringspaces=false,
  upquote=true
}
\lstset{style=auditcode}

% --- Theorem environments ---
\newtheorem{theorem}{Theorem}[section]
\newtheorem{lemma}[theorem]{Lemma}
\newtheorem{proposition}[theorem]{Proposition}
\newtheorem{corollary}[theorem]{Corollary}
\theoremstyle{definition}
\newtheorem{definition}[theorem]{Definition}
\theoremstyle{remark}
\newtheorem{remark}[theorem]{Remark}

% --- Notation helpers ---
\newcommand{\C}{\mathbb{C}}
\newcommand{\R}{\mathbb{R}}
\newcommand{\RePart}{\operatorname{Re}}
\newcommand{\ImPart}{\operatorname{Im}}
\newcommand{\abs}[1]{\left|#1\right|}
\newcommand{\dist}{\operatorname{dist}}

\title{A Width--2 Boundary Program for Excluding Off--Axis Quartets\\
with a Certified Tail Criterion and a Finite--Height Front--End (v37)}
\author{Dylan Anthony Dupont}
\date{v37 compiled: 2026-01-25}

\begin{document}
\maketitle

\begin{abstract}
This document is a truth--latching \emph{architecture} build (v37) of the width--2 boundary program.
It retains the v36 NO--GO guardrails (exponent budget, UE scaling NO--GO for pointwise/sup endpoints, and constant--limited forcing)
and it installs an S5$'$ toolkit for \emph{phase/winding} endpoints (Section~\ref{sec:S5-frontier}):
a collar--safe phase increment calculus, a residual phase budget, and a buffered boundary phase endpoint
$\widetilde D_B(W)$ with a direct $\pi/2$ forcing lemma via the argument principle
(Definition~\ref{def:Db-tilde-phase}, Lemma~\ref{lem:phase-forcing-argprinciple}).

\smallskip
\noindent\textbf{Header note (no drift):} v37 makes \emph{no} claim of uniform tail closure and \emph{no} claim of the Riemann Hypothesis.
The remaining analytic frontier is explicit and single: invent a budget--eligible endpoint mechanism that is simultaneously
(i) forceable and (ii) $\delta$--shrinkable under the nominal scale (OPEN box in Section~\ref{sec:S5-frontier}).

\smallskip
For global hygiene, v35 corrected completion/holomorphy; v37 retains this convention: the working function is the entire width--2 completion
$\Xi_2(u):=\xi(u/2)$, recentered as $E(v):=\Xi_2(1+v)$, so all uses of ``$E$ is entire'' are literally true.
\end{abstract}


\tableofcontents

\section*{Executive Proof Status}
\label{sec:status}

\textbf{Status (v37):} v37 is an \emph{architecture} build: it preserves the decisive v36 NO--GO constraints and installs an S5$'$ phase/winding endpoint toolkit (Section~\ref{sec:S5-frontier}) so that future work is forced onto a single explicit frontier and cannot drift back to the (now invalid) v33 absorption narrative or into naive endpoint redesigns.
The main unconditional output remains the \emph{tail criterion family} (Theorem~\ref{thm:tail-inequality}) together with a finite--height front--end (Definition~\ref{def:frontend}).
No claim of uniform tail closure is made in this version.

\medskip
\noindent\textbf{Proof--grade NO--GO constraints (now explicit):}
\begin{enumerate}[leftmargin=1.5em]
\item \emph{Exponent budget obstruction.} Under the nominal scaling $\delta_0(m,\alpha)=\eta\alpha/(\log m)^2$,
a local/collar blow--up exponent $\theta$ and UE exponent $p$ must satisfy $p-\theta\ge \tfrac12$ for uniform
$\eta$--shrinking closure (Theorem~\ref{thm:exponent-budget} and Theorem~\ref{thm:S5-budget}).
In the present pointwise collar bound one has $\theta=1$.
\item \emph{UE scaling obstruction (pointwise/sup).} With the pointwise/sup endpoint $\sup_{\partial B}|E'/E|$ and shape--only constants,
one cannot achieve any UE exponent $p>1$ (Lemma~\ref{lem:UE-scaling-nogo}). The proved pointwise bound has $p=1$.
\item \emph{Forceability gate + constant-limited forcing.} The current single--box forcing architecture forces only the dial deviation $D_B(W)$
and the available forcing margin is $O(1)$ (Lemma~\ref{lem:force-constant-limited}). Any S5 endpoint redesign is invalid unless it
dominates $D_B(W)$ or supplies a new forcing lemma (Remark~\ref{rem:s5-forceability-gate}).
\item \emph{Boundary modulus has no converse.} The condition $|W|=1$ on $\partial B$ does not exclude interior zeros; Bridge~1 is strictly one--directional
(Remark~\ref{rem:no-converse}).
\item \emph{Absolute $L^r(\partial B)$ log-derivative endpoints are dead.} In any absolute $L^r$ endpoint class,
the best possible UE exponent collapses to $p(r)=1-1/r$ (Lemma~\ref{lem:S5_Lp_collapse}), while the collar/local split has the same exponent
$\theta(r)=1-1/r$ (Proposition~\ref{prop:S5_Lp_nogo}), so $p(r)-\theta(r)=0$ and the local term is $\delta$--inert.
\item \emph{Projection endpoints are dead under current forcing.} Endpoints that annihilate the local kernel span cannot control the forced dial deviation
$D_B(W)$ without a new forcing link (Lemma~\ref{lem:S5-projection-nogo}).
\end{enumerate}

\medskip
\noindent\textbf{Immediate consequence:} the v33/v32 style $\eta$--absorption closure route based on the pointwise/sup upper envelope and the collar is \emph{formally discarded} in v35 and remains discarded in v36 (Appendix~\ref{app:discarded}). Any future closure must change the envelope endpoint and/or the local interface.

\medskip
\noindent\textbf{Completion / holomorphy hygiene (fixed):} the working function is the entire width--2 completion $\Xi_2(u):=\xi(u/2)$ and $E(v):=\Xi_2(1+v)$ (Section~\ref{sec:width2}). All uses of ``$E$ is entire'' are now literally correct.

\medskip
\noindent\textbf{Open proof--grade blockers (v37):}
\begin{enumerate}[leftmargin=1.5em]
\item \textbf{S5 UE redesign (primary frontier).} Replace the pointwise/sup UE endpoint by a non--pointwise functional that controls the same dial deviation $D_B(W)$ while avoiding the $\delta^{-1}$ collar blow--up; see the S5 acceptance criterion (Remark~\ref{rem:S5-accept}) and the baseline NO--GO filters (Subsection~\ref{subsec:S5-nogo-baseline}) (Section~\ref{sec:S5-frontier}). (G--4/G--5 in the prior register.)
\item \emph{Residual envelope ledger.} Lemma~\ref{lem:residual-envelope} still imports a standard RH--free bound for $\zeta'/\zeta$ with local--zero subtraction; this must be proved in--text or cited in a referee--acceptable way with explicit constants. (G--1/G--12.)
\item \emph{Front--end dependence.} The finite--height hypothesis remains an external input (Appendix~\ref{app:frontend}). (G--11.)
\end{enumerate}

\medskip
\noindent\textbf{Reproducibility posture (v37):} numerical artifacts remain an \emph{audit harness} only (Appendix~\ref{app:certificate}).
The v36 repro pack is hardened to record endpoint metadata (functional class + exponent budget parameters + forceability mode). Missing fields render a certificate invalid (Appendix~\ref{app:certificate}), so future redesigns cannot silently mismatch the forcing chain.


\part{Reader's Guide / Definitions and Reduction}
\label{part:guide}

\section{Width--2 normalization}
\label{sec:width2}

Let $s$ denote the usual complex variable for $\zeta(s)$. We pass to the width--$2$ coordinate
\[
u:=2s,\qquad \zeta_2(u):=\zeta(u/2).
\]
Define the width--$2$ completed zeta
\[
\Lambda_2(u):=\pi^{-u/4}\Gamma(u/4)\zeta(u/2).
\]
Then $\Lambda_2$ is \emph{meromorphic} (simple poles at $u=0$ and $u=2$) and satisfies the functional equation
\[
\Lambda_2(u)=\Lambda_2(2-u).
\]
Define the entire completion
\[
\Xi_2(u):=\xi(u/2)=\frac{u(u-2)}{8}\,\Lambda_2(u),
\]
so that $\Xi_2$ is entire and obeys $\Xi_2(u)=\Xi_2(2-u)$.

We recenter at $u=1$:
\[
v:=u-1,\qquad E(v):=\Xi_2(1+v).
\]
Then $E$ is entire and satisfies the evenness relation
\[
E(v)=E(-v),
\]
and complex conjugation gives $E(\overline v)=\overline{E(v)}$.

\begin{remark}[Zeros]
The zeros of $\Xi_2(u)=\xi(u/2)$ are exactly the \emph{nontrivial} zeros of $\zeta(s)$ under the map $u=2s$, with multiplicity.
All boxes used in the tail program lie at heights $m\ge 10$, so the only zeros that can occur in the relevant local windows are nontrivial zeros.
\end{remark}

\section{Heights and horizontal displacement (RH--free)}
\label{sec:heights}

Let $\rho=\beta+i\gamma$ be any nontrivial zero of $\zeta(s)$ (no assumption on $\beta$). In width--2
we write
\begin{equation}
  u_\rho := 2\rho = (1+a)+im,\qquad a:=2\beta-1\in(-1,1),\qquad m:=2\gamma>0.
\end{equation}
Thus RH is equivalent to $a=0$ for every nontrivial zero.

\section{Quartet symmetry in width--2}
\label{sec:quartets}

The functional equation and conjugation imply that any off--axis zero with parameters $(a,m)$
produces a quartet
\begin{equation}
  \{\,1\pm a\pm im\,\} \subset \{u\in\C: \Xi_2(u)=0\}.
\end{equation}
In the centered $v$--coordinate this becomes $\{\pm a\pm im\}\subset\{v\in\C:E(v)=0\}$.

\section{Finite-height front-end after lowering the tail anchor}
\label{sec:frontend}

Once the tail anchor is lowered to $m_\star$, the analytic tail argument covers all $m\ge m_\star$.
The remaining region corresponds to classical heights
\begin{equation}
  0 < \ImPart(s) < H_0 := m_\star/2.
\end{equation}
In v31 we take $m_\star=10$, hence $H_0=5$.

\begin{definition}[Front-end statement]
\label{def:frontend}
We say that \emph{RH holds up to height $H_0$} if every nontrivial zero $\rho=\beta+i\gamma$ with
$0<\gamma\le H_0$ satisfies $\beta=1/2$.
\end{definition}

\begin{remark}[How v31 discharges the front-end]
The required statement for v31 is RH up to height $H_0=5$.
This is a tiny special case of published rigorous verifications of RH to enormous heights.
For example, Platt--Trudgian prove RH for all zeros with $0<\gamma\le 3\cdot 10^{12}$ using interval
arithmetic, which immediately implies RH up to $H_0=5$.
Appendix~\ref{app:frontend} records this discharge in a pinned JSON file.
\end{remark}

\part{Self-Contained Boundary Program and Tail Closure}
\label{part:core}

\section{Aligned boxes and the $\delta(m)$ scale}
\label{sec:boxes}

Let $m>0$ and $\alpha\in(0,1]$. Fix a parameter $\eta\in(0,1)$ and define the \emph{nominal} box scale
\begin{equation}
  \delta_0=\delta_0(m,\alpha):=\frac{\eta\alpha}{(\log m)^2}.
\end{equation}
We will work with aligned boxes $B(\alpha,m,\delta)$ at scales $0<\delta\le \delta_0$.
By default one may take $\delta=\delta_0$, but later (Definition~\ref{def:collar-admissible}) we allow
shrinking $\delta$ to enforce $\kappa$--admissibility; this is non-circular and monotone-safe
(Lemmas~\ref{lem:collar-existence} and \ref{lem:delta-shrink-monotone}).

Define the (width--2) box centered at $\alpha+im$ by
\begin{equation}
  B(\alpha,m,\delta) := \{\,v\in\C: \abs{\RePart v-\alpha}\le \delta,\ \abs{\ImPart v-m}\le \delta\,\}.
\end{equation}
We will also use the symmetric dial centers $v_\pm:=\pm\alpha+im$.

\section{Local factor and finiteness}
\label{sec:local-factor}

For a fixed $m>0$, let
\begin{equation}\label{eq:Z-window}
  Z(m):=\{\,\rho: E(\rho)=0,\ \abs{\ImPart \rho-m}\le 1\,\}
\end{equation}
(zeros counted with multiplicity). Define the local zero factor and residual:
\begin{equation}\label{eq:zloc-def}
  Z_{\mathrm{loc}}(v):=\prod_{\rho\in Z(m)} (v-\rho)^{m_\rho}.
\end{equation}
\begin{equation}\label{eq:F-def}
  F(v):=\frac{E(v)}{Z_{\mathrm{loc}}(v)}.
\end{equation}

\begin{lemma}[Finiteness of $Z_{\mathrm{loc}}$]
\label{lem:zloc-finite}
For each fixed $m>0$ the set $Z(m)$ is finite; hence $Z_{\mathrm{loc}}$ is a finite product and $F$ is
meromorphic globally and analytic in any neighborhood of $\partial B(\alpha,m,\delta)$ that contains
no zeros of $E$.
\end{lemma}

\begin{proof}
Nontrivial zeros of $\zeta$ satisfy $0<\beta<1$, hence in the $v$--coordinate one has
$\RePart v\in(-1,1)$ for all nontrivial zeros.
Therefore the set
\(\{\abs{\ImPart v-m}\le 1\}\cap\{\abs{\RePart v}\le 1\}\)
is compact.
Since $E$ is entire and its zeros are discrete, only finitely many zeros can lie in this compact set.
\end{proof}

\section{Residual envelope bound and the constants ledger}
\label{sec:ledger}
\begin{remark}[Constant gate for the residual term (what is and is not assumed)]
\label{rem:residual-constant-gate}
The tail criterion uses a bound of the form
\[
\sup_{v\in\partial B(\alpha,m,\delta)}\Bigl|\frac{F'(v)}{F(v)}\Bigr|
\ \le\ C_1\log m + C_2,
\]
with constants that must be (i) unconditional (not RH-equivalent) and (ii) uniform in
$(\alpha,\delta,\eta,\kappa)$ once $m\ge 10$ and $0<\alpha\le 1$.
The proof below reduces this to standard RH-free bounds for $\zeta'/\zeta$ in the critical strip with
local-zero subtraction, plus a Stirling-type bound for $\Gamma'/\Gamma$.
\end{remark}

\begin{lemma}[Residual envelope inequality (\texorpdfstring{$\delta$}{delta}-uniform)]
\label{lem:residual-envelope}
Fix $m\ge 10$ and $\alpha\in(0,1]$.
Let $\eta\in(0,1]$ and set the nominal width $\delta_0:=\eta\alpha/(\log m)^2$.
Let $\delta\in(0,\delta_0]$ and set $B:=B(\alpha,m,\delta)$.

Define $E$, $Z_{\rm loc}$ and $F:=E/Z_{\rm loc}$ as in \S\ref{sec:local-factor}
(equations \eqref{eq:zloc-def}--\eqref{eq:F-def}).
Assume \emph{boundary-contact} on $\partial B$ (i.e.\ $E\neq 0$ on $\partial B$; hence $F$ is holomorphic on a
neighborhood of $\partial B$).
Then there exist absolute constants $C_1,C_2>0$ (independent of $m,\alpha,\delta,\eta,\kappa$ and of the zero configuration) such that
\[
\sup_{v\in\partial B}\Bigl|\frac{F'(v)}{F(v)}\Bigr|
\ \le\ C_1\log m + C_2.
\]
\end{lemma}

\begin{proof}[Proof sketch with explicit dependency control]
Write $u:=1+v$ and $s:=u/2=(1+v)/2=\sigma+it$.
For $v\in\partial B(\alpha,m,\delta)$ we have $\RePart(s)\in[0,1]$ and
\[
\ImPart(s)=\frac{\ImPart(v)}{2}\in\Bigl[\frac m2-\frac\delta2,\frac m2+\frac\delta2\Bigr].
\]
Since $m\ge 10$ and $\delta\le \delta_0\le 1/(\log 10)^2<1/5$, we have $\ImPart(s)\asymp m$ uniformly in $\delta$.

\smallskip
\noindent\textbf{1) Log-derivative identity in the $s$-frame.}
From $\Xi_2(u)=\frac{u(u-2)}{8}\Lambda_2(u)$ and $\Lambda_2(u)=\pi^{-u/4}\Gamma(u/4)\zeta(u/2)$ we obtain,
for $u=1+v$,
\[
\frac{E'(v)}{E(v)}
=\frac{\Xi_2'(u)}{\Xi_2(u)}
=\Bigl(\frac{1}{u}+\frac{1}{u-2}\Bigr)\ -\ \frac14\log\pi\ +\ \frac14\frac{\Gamma'}{\Gamma}\Bigl(\frac{u}{4}\Bigr)
\ +\ \frac12\,\frac{\zeta'}{\zeta}(s),
\qquad (u=1+v,\ s=u/2).
\]
Since $u=1+v$ has $\ImPart(u)=m\ge 10$, the completion terms $(1/u+1/(u-2))$ are $O(1/m)$ on $\partial B$ and are absorbed into the absolute constants in the bound.

Moreover, since $v=2s-1$, the local factor derivative satisfies
\[
\frac{Z_{\rm loc}'(v)}{Z_{\rm loc}(v)}=\sum_{\rho\in Z(m)}\frac{m_\rho}{v-\rho}
=\frac12\sum_{\rho_s\in Z_s(m)}\frac{m_{\rho_s}}{s-\rho_s},
\]
where $Z_s(m)$ denotes the corresponding multiset of nontrivial zeros $\rho_s=\beta+i\gamma$ of $\zeta(s)$
with $|\gamma-\tfrac m2|\le \tfrac12$.

Therefore
\[
\frac{F'(v)}{F(v)}
=\frac{E'(v)}{E(v)}-\frac{Z_{\rm loc}'(v)}{Z_{\rm loc}(v)}
= -\frac14\log\pi + \frac14\frac{\Gamma'}{\Gamma}\Bigl(\frac{1+v}{4}\Bigr)
+\frac12\Biggl(\frac{\zeta'}{\zeta}(s) - \sum_{\rho_s\in Z_s(m)}\frac{m_{\rho_s}}{s-\rho_s}\Biggr).
\]

\smallskip
\noindent\textbf{2) RH-free residual bound for $\zeta'/\zeta$ with local-zero subtraction.}
A standard ``local-zero decomposition'' (unconditional) asserts that there exist absolute constants
$A_\zeta,B_\zeta$ such that for $0\le\sigma\le 1$ and $t\ge 5$,
\[
\Bigl|\frac{\zeta'}{\zeta}(\sigma+it) - \sum_{|\gamma-t|\le 1}\frac{1}{(\sigma+it)-\rho}\Bigr|
\ \le\ A_\zeta\log(t+2) + B_\zeta.
\tag{$\star$}
\]
(For a self-contained route, ($\star$) can be derived from the Hadamard product for $\xi(s)$ plus a
Riemann--von Mangoldt bound for $N(T)$; otherwise cite a standard reference.)

For $v\in\partial B$ we have $|t-\tfrac m2|\le \delta/2<1/10$, hence every zero in $Z_s(m)$
satisfies $|\gamma-t|\le 1$ and is included in the sum in ($\star$).
Thus
\[
\frac{\zeta'}{\zeta}(s)-\sum_{\rho_s\in Z_s(m)}\frac{1}{s-\rho_s}
=
\Bigl(\frac{\zeta'}{\zeta}(s)-\sum_{|\gamma-t|\le 1}\frac{1}{s-\rho}\Bigr)
+\sum_{\substack{|\gamma-t|\le 1\\|\gamma-\tfrac m2|>1/2}}\frac{1}{s-\rho}.
\]
In the remaining sum we have $|\gamma-t|\ge 1/2-|t-\tfrac m2|\ge 2/5$, hence $|s-\rho|\ge 2/5$ and each term has modulus $\le 5/2$.
The number of zeros with $|\gamma-t|\le 1$ is bounded by the manuscript's explicit local window majorant
(Lemma~\ref{lem:Nloc-logm}) at height $\asymp m$, so this difference-of-windows sum is $\ll \log m$.

Combining these bounds yields absolute constants $A_{\rm res},B_{\rm res}$ such that
\[
\Bigl|\frac{\zeta'}{\zeta}(s)-\sum_{\rho_s\in Z_s(m)}\frac{1}{s-\rho_s}\Bigr|
\ \le\ A_{\rm res}\log m + B_{\rm res},
\]
uniformly for all $v\in\partial B$ and all $\delta\in(0,\delta_0]$.

\smallskip
\noindent\textbf{3) Gamma factor bound (Stirling, uniform in $\delta$).}
For $z=(1+v)/4$ we have $\RePart(z)\in[1/4,3/4]$ and $|\ImPart(z)|\asymp m$.
A uniform Stirling-type bound gives
\[
\Bigl|\frac{\Gamma'}{\Gamma}(z)\Bigr| \le \log(|\ImPart(z)|+2) + C_\Gamma
\ \le\ \log(m+2)+C_\Gamma,
\]
with an absolute constant $C_\Gamma$.

\smallskip
\noindent\textbf{4) Conclusion.}
Insert the bounds from (2)--(3) into the identity in (1), and absorb harmless constants into $(C_1,C_2)$.
All constants are independent of $(\alpha,\delta,\eta,\kappa)$ because:
(i) $\sigma$ ranges over a fixed compact interval $[0,1]$,
(ii) $t\asymp m$ with $m\ge 10$ uniformly for $\delta\le\delta_0$, and
(iii) the difference-of-windows sum is controlled by Lemma~\ref{lem:Nloc-logm}, which is unconditional.
\end{proof}

\begin{remark}[Hard gate / certificates (v37)]
The tail harness in Appendix~\ref{app:certificate} uses explicit numerical interval enclosures for the
constant ledger (e.g.\ $C_1,C_2,C_{\rm up},C_h'',\kappa$) stored in
\texttt{v36\_repro\_pack/v36\_constants\_m10.json}.
It evaluates the tail inequality for a pinned parameter choice and records the UE exponent $p$ explicitly.
This is an \emph{audit harness} only: it does not certify that the constants file is correct, and it does not,
by itself, yield a uniform tail closure.
An unconditional proof therefore still requires a referee-acceptable certification of the analytic constant ledger,
and a resolution of the UE--Gate (Remark~\ref{rem:UE-gate}).
\end{remark}

\section{Short-side forcing}
\label{sec:forcing}

Assume an off-axis pair at height $m$ with displacement $a>0$ exists. On an aligned box with
$\alpha=a$, the two upper zeros in the centered $v$--plane are at $v=\pm a+im$. The pair factor
\begin{equation}
  Z_{\mathrm{pair}}(v):=(v-(a+im))(v-(-a+im))
\end{equation}
produces a large phase rotation on the near vertical side.

\begin{lemma}[Short-side forcing lower bound]
\label{lem:short-side-forcing}
Let $I_+:=\{\alpha+iy: \abs{y-m}\le\delta\}$ with $\abs{\alpha-a}\le\delta$. Then
\begin{equation}
  \Delta_{I_+}\arg Z_{\mathrm{pair}}
  = 2\arctan\!\left(\frac{\delta}{\abs{\alpha-a}}\right)
    +2\arctan\!\left(\frac{\delta}{\alpha+a}\right)
  \ge \frac{\pi}{2}.
\end{equation}
\end{lemma}


\begin{lemma}[Single-box forcing is constant-limited]
\label{lem:force-constant-limited}
In the forcing setup of Lemma~\ref{lem:short-side-forcing}, the total phase variation of the pair factor along $I_+$ satisfies
\[
\Delta_{I_+}\arg Z_{\mathrm{pair}}\le 2\pi,
\]
uniformly in the height $m$.
Consequently the forcing constant $c$ appearing in the tail inequality (Theorem~\ref{thm:tail-inequality}) is an absolute constant,
independent of $m$; in particular the forcing side cannot grow like $\log m$ (or any unbounded function of $m$) as $m\to\infty$.
\end{lemma}

\begin{proof}
On $I_+=\{\alpha+iy:\ |y-m|\le\delta\}$ one has
\[
Z_{\mathrm{pair}}(\alpha+iy)=\bigl((\alpha-a)+i(y-m)\bigr)\bigl((\alpha+a)+i(y-m)\bigr).
\]
Along $y\in[m-\delta,m+\delta]$ the argument of each linear factor varies by at most $\pi$ (it is an $\arctan$ function whose range lies in an interval of length $\le\pi$),
so the argument of the product varies by at most $2\pi$, uniformly in $m$.
The forcing chain converts a fixed positive portion of $\Delta_{I_+}\arg Z_{\rm pair}$ into the constant $c$ with fixed conversion scalars,
so $c$ is necessarily $O(1)$.
\end{proof}

\section{Outer factorization and the inner quotient (Bridge 1)}
\label{sec:bridge1}

We work on a fixed box $B=B(\alpha,m,\delta)$ and write $B^\circ$ for its interior.
Assume boundary-contact: $E\neq 0$ on $\partial B$ (this will be enforced later by $\kappa$--admissibility; see Definition~\ref{def:collar-admissible} and Lemma~\ref{lem:collar-existence}).

\begin{lemma}[Dirichlet outer factor on a box]
\label{lem:outer_dirichlet_box}
Let $B=B(\alpha,m,\delta)$ be the closed rectangle and $B^\circ$ its interior.
Assume $E$ is holomorphic on a neighborhood of $\overline{B}$ and $E\neq 0$ on $\partial B$.
Then $\log|E|\in C(\partial B)$.
Let $U\in C(\overline{B})\cap \mathrm{Harm}(B^\circ)$ be the unique solution of the Dirichlet problem
with boundary data $U|_{\partial B}=\log|E|$.
Since $B^\circ$ is simply connected, there exists a harmonic conjugate $V$ on $B^\circ$
(unique up to an additive constant) such that $U+iV$ is holomorphic on $B^\circ$.
Define
\[
G_{\mathrm{out}}(v):=\exp(U(v)+iV(v)),\qquad v\in B^\circ.
\]
Then $G_{\mathrm{out}}$ is holomorphic and zero-free on $B^\circ$, satisfies $|G_{\mathrm{out}}(v)|=e^{U(v)}$ for $v\in B^\circ$, and
\[
\lim_{z\to\xi,\ z\in B^\circ}|G_{\mathrm{out}}(z)| = |E(\xi)|\qquad(\xi\in\partial B).
\]
\end{lemma}

\begin{proof}
Continuity of $\log|E|$ on $\partial B$ follows from $E\neq 0$ on $\partial B$.
Existence and uniqueness of $U$ on a rectangle are standard.
Since $B^\circ$ is simply connected, $U$ admits a harmonic conjugate $V$ on $B^\circ$, unique up to an additive constant.
The function $U+iV$ is holomorphic, hence so is $G_{\mathrm{out}}=\exp(U+iV)$, and it is zero-free.
Finally $|G_{\mathrm{out}}|=e^U$ on $B^\circ$, and by continuity of $U$ on $\overline{B}$ we have
$e^{U(\xi)}=|E(\xi)|$ on $\partial B$, yielding the boundary modulus identity in interior-limit form.
\end{proof}

Define on $B^\circ$ the inner quotient
\[
W(v):=\frac{E(v)}{G_{\mathrm{out}}(v)}.
\]
Then $W$ is holomorphic on $B^\circ$ and $|W|=1$ on $\partial B$ in the sense of interior limits in modulus.

\begin{proposition}[Bridge 1: zero-free inner collapse]
\label{prop:bridge1}
Assume the setup of Lemma~\ref{lem:outer_dirichlet_box} and define $W=E/G_{\mathrm{out}}$ on $B^\circ$.
If $W$ is zero-free on $B^\circ$ (equivalently, $E$ is zero-free on $B^\circ$), then $W$ is constant on $B^\circ$; in fact
$W\equiv e^{i\theta_B}$ for some $\theta_B\in\R$.
\end{proposition}

\begin{proof}
Since $W$ is zero-free on $B^\circ$ and $G_{\mathrm{out}}$ is zero-free, the function $E$ is zero-free on $B^\circ$.
Because $B^\circ$ is simply connected, $E$ admits a holomorphic logarithm on $B^\circ$, so $\log|E|$ is harmonic on $B^\circ$.
By construction $U$ is harmonic on $B^\circ$, continuous on $\overline{B}$, and equals $\log|E|$ on $\partial B$.
Thus $U-\log|E|$ is harmonic on $B^\circ$ with zero boundary values, so by Dirichlet uniqueness $U\equiv \log|E|$ on $B^\circ$.
Therefore for $v\in B^\circ$,
\[
|W(v)|=\frac{|E(v)|}{|G_{\mathrm{out}}(v)|}=\frac{|E(v)|}{e^{U(v)}}=\frac{|E(v)|}{e^{\log|E(v)|}}=1.
\]
An analytic function of constant modulus on a connected open set is constant, hence $W\equiv e^{i\theta_B}$.
\end{proof}

\begin{remark}[Boundary modulus convention]
\label{rem:boundary_modulus}
Under boundary-contact, $U$ extends continuously to $\partial B$ and satisfies $U|_{\partial B}=\log|E|$.
Hence $|G_{\mathrm{out}}|=|E|$ holds pointwise on $\partial B$ as interior limits in modulus, and therefore
$|W|=1$ holds pointwise in modulus on $\partial B$.
In boundary integral estimates this may be used in the a.e.\ sense without change.
\end{remark}


\begin{remark}[No converse: boundary modulus does not exclude interior zeros]
\label{rem:no-converse}
Lemma~\ref{lem:outer_dirichlet_box} implies that under boundary-contact the quotient $W:=E/G_{\mathrm{out}}$ satisfies
$|W|=1$ on $\partial B$ (in the interior boundary-limit sense of Remark~\ref{rem:boundary_modulus}).
This condition alone does \emph{not} imply that $W$ is zero-free or constant on $B^\circ$:
nonconstant holomorphic functions on $B^\circ$ can have $|W|=1$ on $\partial B$ and still possess prescribed interior zeros
(e.g.\ via conformal transport of finite Blaschke products from the unit disc).
Thus Proposition~\ref{prop:bridge1} is strictly one-directional: the additional hypothesis ``$W$ is zero-free on $B^\circ$'' is essential.
\end{remark}

\section{Shape-only invariance and the envelope constants}
\label{sec:shape-only}

Let $T(v):=(v-(\alpha+im))/\delta$, mapping $\partial B$ affinely onto the fixed square boundary
$\partial Q$ with $Q=[-1,1]^2$.

\begin{lemma}[Shape-only invariance]
\label{lem:shape-only}
Any constant arising solely from geometric or boundary-operator estimates on $\partial B$ that are
invariant under affine rescaling depends only on $\partial Q$ and is independent of $(\alpha,m,\delta)$.
\end{lemma}

\begin{proof}
Under $T$, arclength scales by $\delta$ and tangential derivatives by $1/\delta$. After normalization,
all purely geometric quantities and operator norms reduce to fixed quantities on $\partial Q$.
\end{proof}

\begin{lemma}[Boundary-to-center evaluation in $L^2$ (sharp $\delta^{-1/2}$)]
\label{lem:eval-L2}
Let $B=B(\alpha,m,\delta)$ be a box and let $v_0$ be its center.
Let $u$ be harmonic on $B^\circ$ and assume its boundary trace lies in $L^2(\partial B,ds)$.
Then, writing $P_B(v_0,\xi)=d\omega^B_{v_0}/ds(\xi)$ for the Poisson kernel of $B$ at $v_0$,
\[
\abs{u(v_0)}\le \|P_B(v_0,\cdot)\|_{L^2(\partial B,ds)}\;\|u\|_{L^2(\partial B,ds)}.
\]
Under the similarity $T(\xi)=(\xi-v_0)/\delta$ mapping $\partial B$ onto $\partial Q$,
\[
\|P_B(v_0,\cdot)\|_{L^2(\partial B,ds)}=\delta^{-1/2}\,\|P_Q(0,\cdot)\|_{L^2(\partial Q,ds)}.
\]
In particular the exponent $\delta^{-1/2}$ is sharp (witnessed by the constant harmonic function $u\equiv 1$).
\end{lemma}

\begin{proof}
For harmonic $u$ on $B^\circ$ with $L^2$ trace on $\partial B$, the Poisson representation gives
\[
u(v_0)=\int_{\partial B}u(\xi)\,d\omega^B_{v_0}(\xi)
=\int_{\partial B}u(\xi)\,P_B(v_0,\xi)\,ds(\xi).
\]
Cauchy--Schwarz yields $\abs{u(v_0)}\le \|P_B(v_0,\cdot)\|_{L^2}\|u\|_{L^2}$.

For the scaling: under $T$, arclength scales by $ds=\delta\,ds_Q$ and Poisson kernels scale by
$P_B(v_0,\xi)=\delta^{-1}P_Q(0,T(\xi))$. Hence
\[
\int_{\partial B}P_B(v_0,\xi)^2\,ds(\xi)
=\int_{\partial Q}\delta^{-2}P_Q(0,\zeta)^2\,\delta\,ds_Q(\zeta)
=\delta^{-1}\int_{\partial Q}P_Q(0,\zeta)^2\,ds_Q(\zeta),
\]
giving $\|P_B(v_0,\cdot)\|_{L^2}=\delta^{-1/2}\|P_Q(0,\cdot)\|_{L^2}$.

Sharpness: for $u\equiv 1$ we have $\abs{u(v_0)}=1$ and $\|u\|_{L^2(\partial B)}=\sqrt{|\partial B|}\asymp \delta^{1/2}$,
so the inequality forces $\|P_B(v_0,\cdot)\|_{L^2}\gtrsim \delta^{-1/2}$.
\end{proof}

\begin{lemma}[Upper envelope bound (outer-aligned form)]
\label{lem:upper-envelope}
Let $B=B(\pm a,m,\delta)$ be an aligned box and let $G_{\rm out}$ be the outer factor on $B$
constructed from $\log|E|$ on $\partial B$ (Section~\ref{sec:bridge1}). Define the inner quotient
\[
W(v):=\frac{E(v)}{G_{\rm out}(v)}.
\]
Assume the boundary-contact convention: $E$ has no zeros on $\partial B$ (hence $W$ has unimodular
boundary values a.e.). For each sign $\pm$ let $v_\pm:=\pm a+im$ and let $e^{i\varphi_0^\pm}\in\mathbb T$
be an $L^2(\partial B,ds)$-best constant phase,
\[
e^{i\varphi_0^\pm}\in\arg\min_{|c|=1}\int_{\partial B}|W(v)-c|^2\,ds(v).
\]
Then there exists a \emph{shape-only} constant $C_{\rm up}>0$ (depending only on the normalized square
$Q=[-1,1]^2$) such that
\begin{equation}
\label{eq:UE-EoverE}
\sum_{\pm}\bigl|W(v_\pm)-e^{i\varphi_0^\pm}\bigr|
\ \le\ 2\,C_{\rm up}\,\delta\,\sup_{v\in\partial B}\Bigl|\frac{E'(v)}{E(v)}\Bigr|.
\end{equation}
One admissible explicit definition is
\[
C_{\rm up}
:=\Big(\sup_{\xi\in\partial Q}P_Q(0,\xi)\Big)^{1/2}\cdot \frac{4}{\pi}\cdot \sqrt{8}\cdot\bigl(1+\|H_{\partial Q}\|_{L^2\to L^2}\bigr),
\]
where $P_Q(0,\xi)=d\omega^Q_0/ds(\xi)$ is the Poisson kernel of $Q$ at the center $0$ with respect to
arclength on $\partial Q$, and $H_{\partial Q}$ is the boundary conjugation (Hilbert/Cauchy) operator
on $\partial Q$.
\end{lemma}

\begin{remark}[No residual proxying in the upper envelope]
\label{rem:no-proxying}
Lemma~\ref{lem:upper-envelope} controls the inner quotient $W=E/G_{\rm out}$ and therefore depends on
$\sup_{\partial B}|E'/E|$.
Residual bounds for $F=E/Z_{\rm loc}$ control $\sup_{\partial B}|F'/F|$ and do \emph{not} by themselves
bound $\sup_{\partial B}|E'/E|$.
Whenever the residual envelope is used to control dial deviation, it must be routed through the
log-derivative split $E'/E=F'/F+Z'_{\rm loc}/Z_{\rm loc}$ (Lemma~\ref{lem:logder-split}) together with
the collar bound (Lemma~\ref{lem:Zloc-logder-collar}), yielding Corollary~\ref{cor:UE-residual-local}.
\end{remark}

\begin{proof}
Fix one sign and write $v_0=v_\pm$ and $B=B(\pm a,m,\delta)$.
We record the (RH-free) chain and indicate the scale factors explicitly.
\begin{enumerate}[leftmargin=1.5em]
\item \textbf{Evaluation from the boundary (harmonic measure; produces $\delta^{-1/2}$).}
For any constant $c\in\mathbb T$, subharmonicity of $|W-c|^2$ implies
\[
|W(v_0)-c|^2\le \int_{\partial B}|W(\xi)-c|^2\,d\omega_{v_0}^B(\xi)
=\int_{\partial B}|W(\xi)-c|^2\,P_B(v_0,\xi)\,ds(\xi),
\]
so
\[
|W(v_0)-c|\le \|P_B(v_0,\cdot)\|_{L^\infty(\partial B)}^{1/2}\,\|W-c\|_{L^2(\partial B,ds)}.
\]
Under the similarity $T(\xi)=(\xi-v_0)/\delta$ mapping $\partial B$ onto $\partial Q$,
Poisson kernels scale by
$\|P_B(v_0,\cdot)\|_\infty^{1/2}=\delta^{-1/2}\,\|P_Q(0,\cdot)\|_\infty^{1/2}$.
\item \textbf{Poincar\'e/Wirtinger on $\partial B$ (produces $\delta$).}
For the $L^2$-best constant $c=e^{i\varphi_0^\pm}$ and $|\partial B|=8\delta$,
periodic Poincar\'e on a loop of length $8\delta$ gives
\[
\|W-c\|_{L^2(\partial B)}\le \frac{|\partial B|}{2\pi}\,\|\partial_s W\|_{L^2(\partial B)}
=\frac{4\delta}{\pi}\,\|\partial_s W\|_{L^2(\partial B)}.
\]
\item \textbf{Outer factor control (no $\delta$; uses bounded boundary conjugation).}
Write $\log G_{\rm out}=U+i\widetilde U$ with $U|_{\partial B}=\log|E|$ and $\widetilde U=H_{\partial B}U$.
Differentiating tangentially,
$\partial_s\log G_{\rm out}=\partial_s U+i\,H_{\partial B}(\partial_s U)$.
Since $\log W=\log E-\log G_{\rm out}$,
\[
\|\partial_s\log W\|_{L^2(\partial B)}
\le \bigl(1+\|H_{\partial B}\|_{L^2\to L^2}\bigr)\,\|\partial_s\log E\|_{L^2(\partial B)}
\le \bigl(1+\|H_{\partial B}\|_{L^2\to L^2}\bigr)\,\Big\|\frac{E'}{E}\Big\|_{L^2(\partial B)}.
\]
On $\partial B$ we have $|W|=1$ a.e., hence $|\partial_s W|=|\partial_s\log W|$.
\item \textbf{$L^2$ to $\sup$ (produces $\delta^{1/2}$).}
Using $|\partial B|=8\delta$,
\[
\Big\|\frac{E'}{E}\Big\|_{L^2(\partial B)}\le \sqrt{|\partial B|}\,\sup_{\partial B}\Big|\frac{E'}{E}\Big|
=\sqrt{8\delta}\,\sup_{\partial B}\Big|\frac{E'}{E}\Big|.
\]
\end{enumerate}
Combining the four steps yields
\[
|W(v_0)-e^{i\varphi_0^\pm}|
\le \|P_Q(0,\cdot)\|_\infty^{1/2}\cdot \frac{4}{\pi}\cdot \sqrt{8}\cdot\bigl(1+\|H_{\partial Q}\|_{L^2\to L^2}\bigr)
\cdot \delta\sup_{\partial B}\Big|\frac{E'}{E}\Big|,
\]
where we used the similarity invariance $\|H_{\partial B}\|_{L^2\to L^2}=\|H_{\partial Q}\|_{L^2\to L^2}$.
Summing over $\pm$ gives \eqref{eq:UE-EoverE}.
\end{proof}


\subsection{Local factor split and collar control}
\label{subsec:local-split}

\begin{definition}[Collar-admissible aligned boxes]
\label{def:collar-admissible}
Fix once and for all a collar parameter $\kappa\in(0,1/10)$.
An aligned box $B=B(\alpha,m,\delta)$ is called \emph{$\kappa$--admissible} if
\[
\dist\bigl(\partial B,\,\mathcal Z(E)\bigr)\ge \kappa\delta.
\]
Given any nominal scale $\delta_0>0$ and any center, there exists some $0<\delta\le \delta_0$ for which
$\kappa$--admissibility holds (Lemma~\ref{lem:collar-existence}).
Whenever a chosen box is not $\kappa$--admissible, we shrink $\delta$ until $\kappa$--admissibility holds.
Moreover the assembled tail inequality is monotone-safe under such $\delta$--shrinking
(Lemma~\ref{lem:delta-shrink-monotone}).
\end{definition}

\begin{lemma}[Existence of a $\kappa$--admissible shrink]
\label{lem:collar-existence}
Fix $\kappa\in(0,1/10)$ and a center $v_0\in\C$.
For every $\delta_0>0$ there exists $\delta'\in(0,\delta_0]$ such that the closed box
\[
B(v_0,\delta'):=\{\,v\in\C:\ \|v-v_0\|_\infty\le \delta'\,\}
\]
satisfies
\[
\dist\bigl(\partial B(v_0,\delta'),\,\mathcal Z(E)\bigr)\ge \kappa\delta'.
\]
In particular, given $(\alpha,m)$ and nominal $\delta_0=\eta\alpha/(\log m)^2$, one may always choose a scale
$0<\delta\le\delta_0$ for which $B(\alpha,m,\delta)$ is $\kappa$--admissible.
\end{lemma}

\begin{proof}
Zeros of the entire function $E$ are isolated.
Choose $\varepsilon>0$ such that $\mathcal Z(E)\cap \{\,0<\|v-v_0\|_\infty\le \varepsilon\,\}$ is empty
(if $E(v_0)=0$) or such that $\mathcal Z(E)\cap \{\,\|v-v_0\|_\infty\le \varepsilon\,\}$ is empty
(if $E(v_0)\neq 0$).
Set $\delta':=\min\{\delta_0,\varepsilon/(1+\kappa)\}$.
Then every boundary point satisfies $\|v-v_0\|_\infty=\delta'$.
Any zero $\rho\in\mathcal Z(E)$ is either $\rho=v_0$ (in which case $\dist(v,\rho)=\delta'\ge \kappa\delta'$)
or satisfies $\|\rho-v_0\|_\infty\ge \varepsilon$ (in which case $\dist(v,\rho)\ge \varepsilon-\delta'\ge \kappa\delta'$).
Therefore $\dist(\partial B(v_0,\delta'),\mathcal Z(E))\ge \kappa\delta'$.
\end{proof}



\begin{lemma}[Log-derivative decomposition]
\label{lem:logder-split}
With $Z_{\rm loc}$ and $F$ as in \eqref{eq:zloc-def} and \eqref{eq:F-def}, one has on any region where
$E$ and $Z_{\rm loc}$ are holomorphic and nonvanishing (in particular on $\partial B$ under the boundary-contact convention)
\[
\frac{E'}{E}=\frac{F'}{F}+\frac{Z_{\rm loc}'}{Z_{\rm loc}}.
\]
\end{lemma}

\begin{lemma}[Buffered local factor bound on $\partial B$]
\label{lem:Zloc-logder-collar}
Let $B=B(\alpha,m,\delta)$ be $\kappa$--admissible in the sense of Definition~\ref{def:collar-admissible}.
Then
\[
\sup_{v\in\partial B}\left|\frac{Z_{\rm loc}'(v)}{Z_{\rm loc}(v)}\right|
\le \frac{N_{\rm loc}(m)}{\kappa\,\delta},
\]
where $N_{\rm loc}(m)$ counts zeros of $E$ in the local window used to define $Z_{\rm loc}$, with multiplicity.
\end{lemma}


\begin{lemma}[Local log-derivative bound in $L^2(\partial B)$]
\label{lem:Zloc-L2-collar}
Let $B=B(\alpha,m,\delta)$ be $\kappa$--admissible (Definition~\ref{def:collar-admissible}), and let
$Z_{\rm loc}$ be the local factor with local zero-count $N_{\rm loc}(m)$ as in Section~\ref{sec:local-factor}.
Then
\[
\left\|\frac{Z'_{\rm loc}}{Z_{\rm loc}}\right\|_{L^2(\partial B)}
\le \frac{\sqrt{8}\,N_{\rm loc}(m)}{\kappa\,\delta^{1/2}}.
\]
More generally, for any $1\le r\le\infty$,
\[
\left\|\frac{Z'_{\rm loc}}{Z_{\rm loc}}\right\|_{L^r(\partial B)}
\le \frac{8^{1/r}N_{\rm loc}(m)}{\kappa\,\delta^{1-1/r}}.
\]
\end{lemma}

\begin{proof}
Lemma~\ref{lem:Zloc-logder-collar} gives $\|Z'_{\rm loc}/Z_{\rm loc}\|_{L^\infty(\partial B)}\le N_{\rm loc}(m)/(\kappa\delta)$.
Since $|\partial B|=8\delta$, we have $\|f\|_{L^r(\partial B)}\le |\partial B|^{1/r}\|f\|_{L^\infty(\partial B)}$ for every $1\le r\le\infty$,
which yields the stated bounds.
\end{proof}



\begin{lemma}[Explicit local window zero count]
\label{lem:Nloc-logm}
Let $N(T)$ denote the number of nontrivial zeros $\rho=\beta+i\gamma$ of $\zeta(s)$ with $0<\gamma\le T$,
counted with multiplicity.
Then for every $T\ge 5$,
\begin{equation}
\label{eq:two-unit-window}
N(T+1)-N(T-1)\le 1.01\log T + 17.
\end{equation}
Consequently, for every $m\ge 10$,
\begin{equation}
\label{eq:Nloc-explicit}
N_{\rm loc}(m)\le 1.01\log m + 17.
\end{equation}
\end{lemma}

\begin{proof}
By \cite[Theorem~1.1]{BellottiWongZeta2024}, for every $x\ge e$,
\[
\Bigl|\,N(x)-\frac{x}{2\pi}\log\!\Bigl(\frac{x}{2\pi e}\Bigr)\Bigr|
\le 0.10076\log x + 0.24460\log\log x + 8.08344.
\]
Let $M(x):=\frac{x}{2\pi}\log(\frac{x}{2\pi e})$, so $M'(x)=\frac{1}{2\pi}\log(\frac{x}{2\pi})$.
For $T\ge 5$ we have $\log(T\pm 1)\le \log(2T)$ and $\log\log x\le \log x$ for $x\ge e$, hence
\[
N(T+1)-N(T-1)
\le (M(T+1)-M(T-1)) + 2(0.10076+0.24460)\log(2T) + 2\cdot 8.08344.
\]
Moreover
\[
M(T+1)-M(T-1)=\int_{T-1}^{T+1}M'(x)\,dx
\le \int_{T-1}^{T+1}\frac{1}{2\pi}\log x\,dx
\le \frac{1}{\pi}\log(2T).
\]
Combining these bounds gives $N(T+1)-N(T-1)\le 1.00903\log T + 16.8663 \le 1.01\log T + 17$,
establishing \eqref{eq:two-unit-window}.
Finally, in width--2 one has $m=2T$.
The local window $|\ImPart\rho-m|\le 1$ corresponds to $|\gamma-T|\le 1/2$ in the $s$--plane, so
$N_{\rm loc}(m)=N(T+\tfrac12)-N(T-\tfrac12)\le N(T+1)-N(T-1)$, yielding \eqref{eq:Nloc-explicit}.
\end{proof}



\begin{corollary}[Outer-aligned upper envelope in residual+local form]
\label{cor:UE-residual-local}
Let $B$ be $\kappa$--admissible.
Assume the residual envelope bound of Lemma~\ref{lem:residual-envelope}, i.e.
\(\sup_{\partial B}|F'/F|\le L(m):=C_1\log m+C_2\).
Then
\[
\sum_{\pm}|W(v_\pm)-e^{i\varphi_0^\pm}|
\le 2C_{\rm up}\left(\delta L(m) + \frac{N_{\rm loc}(m)}{\kappa}\right)
\le 2C_{\rm up}\left(\delta L(m) + \frac{1.01\log m + 17}{\kappa}\right).
\]
\end{corollary}

\begin{remark}[UE gate = exponent budget at the local interface]
\label{rem:UE-gate}
Lemma~\ref{lem:upper-envelope} is the \emph{only} step in the envelope chain that generates a positive power
of $\delta$ in front of a boundary log-derivative endpoint.
Abstractly, suppose an upper-envelope mechanism yields, for some $p>0$,
\[
\sum_{\pm}\bigl|W(v_\pm)-e^{i\varphi_0^\pm}\bigr|
\le 2C_{\rm up}\,\delta^{p}\,\sup_{\partial B}\Bigl|\frac{E'}{E}\Bigr|,
\]
and suppose the collar/local split yields, for some $\theta>0$,
\[
\sup_{\partial B}\Bigl|\frac{E'}{E}\Bigr|
\le L(m)\ +\ \frac{N_{\rm loc}(m)}{\kappa\,\delta^{\theta}}.
\]
Then the local contribution in the envelope side scales as $\delta^{p-\theta}N_{\rm loc}(m)/\kappa$.
Under the nominal choice $\delta_0(m,\alpha)=\eta\alpha/(\log m)^2$ and the unconditional majorant
$N_{\rm loc}(m)\ll\log m$, uniform $\eta$--shrinking tail closure is possible only if
\[
p-\theta\ \ge\ \tfrac12
\]
(Theorem~\ref{thm:exponent-budget}).

In the \emph{proved} pointwise/sup architecture one has $p=1$ (Lemma~\ref{lem:upper-envelope}) and $\theta=1$ (Lemma~\ref{lem:Zloc-logder-collar}),
so $p-\theta=0$ and the local term is $\delta$--inert; $\eta$--shrinking cannot suppress it (Lemma~\ref{lem:UE-d1-obstruction}).
Moreover, within this same endpoint class, a strengthened exponent $p>1$ is impossible with shape--only constants
(Lemma~\ref{lem:UE-scaling-nogo}).
Thus the former $\eta$--absorption closure route based on the pointwise/sup UE endpoint is a formal NO--GO and is recorded as discarded
(Appendix~\ref{app:discarded}).
\end{remark}

\begin{theorem}[Exponent budget for $\eta$--shrinking under $\delta_0(m,\alpha)=\eta\alpha/(\log m)^2$]
\label{thm:exponent-budget}
Let $m\ge 10$, $\alpha\in(0,1]$ and $\eta\in(0,1]$, and set the nominal scale
\[
\delta_0(m,\alpha):=\frac{\eta\alpha}{(\log m)^2}.
\]
Assume that for all $0<\delta\le \delta_0(m,\alpha)$ one has:
\begin{enumerate}[leftmargin=1.5em]
\item[(UE$_p$)] (\emph{UE exponent}) for some $p>0$,
\[
\mathrm{UE}(\delta)\ \le\ 2C_{\rm up}\,\delta^{p}\,\sup_{\partial B}\Bigl|\frac{E'}{E}\Bigr|;
\]
\item[(COL$_\theta$)] (\emph{Collar/local exponent}) for some $\theta>0$,
\[
\sup_{\partial B}\Bigl|\frac{E'}{E}\Bigr|
\ \le\ L(m)\ +\ \frac{N_{\rm loc}(m)}{\kappa\,\delta^{\theta}},
\]
with fixed $\kappa\in(0,1/10)$;
\item[(GROW)] (\emph{Majorants}) $L(m)\le A_L\log m+B_L$ and $N_{\rm loc}(m)\le A_N\log m+B_N$ for all $m\ge 10$;
\item[(FORCE)] (\emph{Forcing side}) the forcing-vs-envelope tail inequality has a fixed positive forcing constant $c>0$ and only $\delta$--helpful subtractive terms on the RHS.
\end{enumerate}
Then at $\delta=\delta_0(m,\alpha)$ one has the explicit bound
\begin{equation}
\mathrm{UE}(\delta_0)\ \le\ 
2C_{\rm up}\Bigl(\delta_0^{p}L(m)\ +\ \delta_0^{p-\theta}\frac{N_{\rm loc}(m)}{\kappa}\Bigr).
\tag{BUDGET}
\end{equation}
Moreover, \emph{uniform tail closure by $\eta$--shrinking} (i.e.\ there exists $\eta_\star>0$ such that for every $\eta\le \eta_\star$ the tail inequality holds for all $m\ge 10$) is possible only if
\begin{equation}
p-\theta\ \ge\ \tfrac12.
\tag{B1}
\end{equation}
\end{theorem}

\begin{proof}
Insert (COL$_\theta$) into (UE$_p$) at $\delta=\delta_0$ to obtain (BUDGET).
At $\alpha=1$ one has $\delta_0(m,1)=\eta/(\log m)^2$, so the local term behaves as
\[
\delta_0^{p-\theta}N_{\rm loc}(m)\ \ll\ \Bigl(\frac{\eta}{(\log m)^2}\Bigr)^{p-\theta}\log m
=\eta^{p-\theta}(\log m)^{1-2(p-\theta)}.
\]
If $p-\theta<1/2$ then $1-2(p-\theta)>0$, so the local contribution grows without bound as $m\to\infty$,
while the forcing side tends to the fixed constant $c$ because all RHS corrections are $\delta$--helpful and vanish as $\delta_0\to 0$.
Hence uniform tail closure is impossible.
If $p-\theta\ge 1/2$ then the local contribution is uniformly bounded by $O(\eta^{p-\theta})$ and tends to $0$ as $\eta\downarrow 0$,
enabling uniform absorption once all constants are $\delta$--uniform.
\end{proof}



\begin{lemma}[$\eta$--absorption obstruction under the pointwise UE exponent $p=1$]
\label{lem:UE-d1-obstruction}
Assume the hypotheses of Corollary~\ref{cor:UE-residual-local}.
Then for every $\delta\le \delta_0(m,\alpha)=\eta\alpha/(\log m)^2$,
\[
\sum_{\pm}\bigl|W(v_\pm)-e^{i\varphi_0^\pm}\bigr|
\le 2C_{\rm up}\Bigl(\delta L(m)+\frac{N_{\rm loc}(m)}{\kappa}\Bigr).
\]
In particular, letting $\eta\downarrow 0$ (hence $\delta\downarrow 0$) only suppresses the residual term
$\delta L(m)$; the local term $N_{\rm loc}(m)/\kappa$ does \emph{not} decay with $\eta$.
Therefore any absorption-style closure that attempts to force the envelope side small by choosing $\eta$
must additionally verify a \emph{separate} inequality of the form
\[
\frac{2C_{\rm up}}{\kappa}\,N_{\rm loc}(m)\ <\ c
\]
at the relevant anchor height(s), where $c$ is the forcing constant in \eqref{eq:tail-ineq}.
\end{lemma}

\begin{lemma}[UE scaling NO--GO for pointwise/sup endpoints]
\label{lem:UE-scaling-nogo}
Assume an upper-envelope bound of the form
\[
\sum_{\pm}\bigl|W(v_\pm)-e^{i\varphi_0^\pm}\bigr|
\ \le\ 2C_{\rm up}\,\delta^{p}\,\sup_{\partial B}\Bigl|\frac{E'}{E}\Bigr|
\qquad (p>0),
\]
where the constant $C_{\rm up}$ depends only on the normalized shape (Lemma~\ref{lem:shape-only}) and is independent of $\delta$.
Then necessarily $p\le 1$.
In particular, no pointwise/sup envelope mechanism with shape--only constants can yield any exponent $p>1$.
\end{lemma}

\begin{proof}
Under the affine rescaling $T(v)=(v-(\alpha+im))/\delta$, the boundary $\partial B$ maps to the fixed square boundary $\partial Q$.
If $\widetilde E(z):=E(T^{-1}(z))$, then by the chain rule
\[
\frac{E'}{E}(T^{-1}(z))=\frac{1}{\delta}\,\frac{\widetilde E'(z)}{\widetilde E(z)}.
\]
Hence $\sup_{\partial B}|E'/E|=\delta^{-1}\sup_{\partial Q}|\widetilde E'/\widetilde E|$.
The left-hand side of the upper-envelope bound is dimensionless (it is a sum of moduli of complex numbers), and under the normalization it may be $O(1)$
for admissible configurations on the fixed shape.
Therefore the bound forces
\[
O(1)\ \le\ 2C_{\rm up}\,\delta^{p-1}\,\sup_{\partial Q}\Bigl|\frac{\widetilde E'}{\widetilde E}\Bigr|
\qquad\text{as }\delta\downarrow 0.
\]
Since the normalized endpoint $\sup_{\partial Q}|\widetilde E'/\widetilde E|$ is not forced to blow up as $\delta\downarrow 0$ (it depends only on the normalized data),
the factor $\delta^{p-1}$ cannot tend to $0$. Thus $p-1\le 0$, i.e.\ $p\le 1$.
\end{proof}



\begin{proof}
The displayed bound is exactly Corollary~\ref{cor:UE-residual-local} with the corrected UE exponent $p=1$.
As $\eta\to 0$ one has $\delta_0\to 0$ and hence $\delta L(m)\to 0$, while $N_{\rm loc}(m)/\kappa$ is unchanged.
Since the forcing lower bound in the tail inequality tends to $c$ as $\delta\downarrow 0$, the strict inequality
requires the stated necessary condition at the anchor.
\end{proof}


\subsection{Horizontal non-forcing budget in residual form}
\label{subsec:horizontal-budget}

\begin{definition}[Horizontal non-forcing phase budget]
\label{def:Delta-nonforce}
Let $B=B(\pm a,m,\delta)$ be an aligned box and let $F=E/Z_{\rm loc}$ be the residual factor.
Assume $F$ is holomorphic and zero-free on a neighborhood of $\partial B$.
Let $H_\pm$ denote the top and bottom edges of $\partial B$:
\[
H_+:=\{x+i(m+\delta):\ x\in[\pm a-\delta,\pm a+\delta]\},\qquad
H_-:=\{x+i(m-\delta):\ x\in[\pm a-\delta,\pm a+\delta]\}.
\]
Define
\[
\Delta_{\rm nonforce}(B)
:=
\int_{H_+}|\partial_s\arg F|\,ds + \int_{H_-}|\partial_s\arg F|\,ds.
\]
\end{definition}

\begin{lemma}[Horizontal budget (residual form; audit-grade)]
\label{lem:horizontal-budget}
In the setting of Definition~\ref{def:Delta-nonforce},
\[
\Delta_{\rm nonforce}(B)\le 4\delta\,\sup_{v\in\partial B}\left|\frac{F'(v)}{F(v)}\right|.
\]
Consequently, if $\sup_{\partial B}|F'/F|\le C_1\log m+C_2$, then
\[
\Delta_{\rm nonforce}(B)\le C_h''\,\delta\,(\log m+1),\qquad
C_h'':=4\max\{C_1,\,C_2\}.
\]
\end{lemma}

\begin{proof}
On either horizontal edge, $|\partial_s\arg F|\le |F'/F|$ pointwise.
Each edge has length $2\delta$, hence each integral is bounded by $2\delta\sup_{\partial B}|F'/F|$.
Summing top and bottom gives the first inequality, and the second follows from $\sup_{\partial B}|F'/F|\le C_1\log m+C_2\le \max\{C_1,C_2\}(\log m+1)$.
\end{proof}

\section{The explicit tail inequality (post-pivot)}
\label{sec:tail}
For $m\ge 10$ we use the growth surrogate
\[
L(m):=C_1\log m + C_2,
\]
with constants as in Lemma~\ref{lem:residual-envelope}.
For the local window term we use the explicit majorant from Lemma~\ref{lem:Nloc-logm}:
\[
N_{\rm up}(m):=1.01\log m + 17\ \ \text{so that}\ \ N_{\rm loc}(m)\le N_{\rm up}(m)\quad(m\ge 10).
\]

For a parameter $\eta\in(0,1)$ and a dial displacement $\alpha\in(0,1]$ define the \emph{nominal} scale
\[
\delta_0:=\delta_0(m,\alpha):=\frac{\eta\alpha}{(\log m)^2}.
\]
Fix a collar parameter $\kappa\in(0,1/10)$ as in Definition~\ref{def:collar-admissible}.
For each $(m,\alpha)$ we choose any scale $0<\delta\le\delta_0$ such that the aligned boxes
$B=B(\pm\alpha,m,\delta)$ are $\kappa$--admissible; existence is guaranteed by Lemma~\ref{lem:collar-existence}.
By Lemma~\ref{lem:delta-shrink-monotone}, shrinking $\delta$ only helps in the tail inequality, so it is safe to
treat $\delta_0$ as the worst-case scale in one-height reductions.

\begin{theorem}[Tail inequality (criterion form; pointwise UE exponent $p=1$)]
\label{thm:tail-inequality}
Fix $m\ge 10$ and $\eta\in(0,1)$.
Assume:
\begin{enumerate}[leftmargin=1.5em]
\item the forcing lemma producing the positive constant
\[
 c_0:=\frac{3\log 2}{8\pi},\qquad c:=\frac{3\log 2}{16},\qquad K_{\rm alloc}:=3+8\sqrt3;
\]
\item the residual envelope bound (Lemma~\ref{lem:residual-envelope}) providing $C_1,C_2$;
\item the audit-grade horizontal budget bound (Lemma~\ref{lem:horizontal-budget}), giving a constant
$C_h''$ independent of $(\alpha,m,\delta)$;
\item the explicit local window bound (Lemma~\ref{lem:Nloc-logm}) providing the majorant $N_{\rm up}(m)=1.01\log m+17$.
\end{enumerate}
Then for every $\alpha\in(0,1]$ and every $\kappa$--admissible aligned box $B=B(\pm\alpha,m,\delta)$,
absence of off--axis quartets at height $m$ follows from the strict inequality
\begin{equation}
\label{eq:tail-ineq}
2C_{\mathrm{up}}\Bigl(\delta L(m)\; +\; \frac{N_{\rm up}(m)}{\kappa}\Bigr)
\ <\
 c\ -\ \delta\Bigl(K_{\rm alloc}\,c_0\,L(m)+C_h''\,(\log m+1)\Bigr).
\end{equation}
\end{theorem}

\begin{proof}[Proof sketch / bookkeeping]
The forcing side is unchanged from v31. The only post-pivot modification is on the upper-envelope
side: Lemma~\ref{lem:upper-envelope} bounds dial deviation in terms of $\sup_{\partial B}|E'/E|$.
Applying the log-derivative split (Lemma~\ref{lem:logder-split}), the residual envelope for
$\sup_{\partial B}|F'/F|\le L(m)$ (Lemma~\ref{lem:residual-envelope}), and the collar bound
$\sup_{\partial B}|Z_{\rm loc}'/Z_{\rm loc}|\le N_{\rm loc}(m)/(\kappa\delta)$
(Lemma~\ref{lem:Zloc-logder-collar}) yields
\[
\sup_{\partial B}\Big|\frac{E'}{E}\Big|\le L(m)+\frac{N_{\rm loc}(m)}{\kappa\delta}
\le L(m)+\frac{N_{\rm up}(m)}{\kappa\delta}.
\]
Plugging this into Lemma~\ref{lem:upper-envelope} gives the left-hand side of
\eqref{eq:tail-ineq}.
The right-hand side is the forcing lower bound, with the horizontal non-forcing term bounded by
Lemma~\ref{lem:horizontal-budget}.
\end{proof}

\begin{lemma}[Monotonicity under $\delta$--shrinking]
\label{lem:delta-shrink-monotone}
Fix $m\ge 10$, $\alpha\in(0,1]$, and constants $C_{\rm up},\kappa,c,c_0,K_{\rm alloc},C_h'',C_1,C_2$.
Let $L(m)=C_1\log m + C_2$ and $N_{\rm up}(m)=1.01\log m +17$.
For $\delta\in(0,1]$ define
\[
\mathrm{LHS}(\delta):=
2C_{\mathrm{up}}\Bigl(\delta L(m)\; +\; \frac{N_{\rm up}(m)}{\kappa}\Bigr),
\qquad
\mathrm{RHS}(\delta):=
c\ -\ \delta\Bigl(K_{\rm alloc}\,c_0\,L(m)+C_h''\,(\log m+1)\Bigr).
\]
Then $\mathrm{LHS}(\delta)$ is (weakly) increasing in $\delta$ and $\mathrm{RHS}(\delta)$ is (weakly) decreasing.
Consequently, if $\mathrm{LHS}(\delta_0)<\mathrm{RHS}(\delta_0)$ for some $\delta_0\in(0,1]$, then
$\mathrm{LHS}(\delta)<\mathrm{RHS}(\delta)$ holds for every $\delta\in(0,\delta_0]$.
\end{lemma}

\begin{proof}
For $\delta>0$, the map $\delta\mapsto \delta L(m)$ is increasing and the term $N_{\rm up}(m)/\kappa$ is independent of $\delta$, hence $\mathrm{LHS}(\delta)$ is (weakly) increasing.
The bracketed factor in $\mathrm{RHS}(\delta)$ is nonnegative and independent of $\delta$, so $\mathrm{RHS}(\delta)$ decreases linearly in $\delta$.
\end{proof}



\begin{lemma}[Worst case in $\alpha$ is $\alpha=1$ at the nominal scale]
\label{lem:worst-alpha}
Fix $m\ge 10$ and $\eta\in(0,1)$. Define the nominal scale $\delta_0(m,\alpha)=\eta\alpha/(\log m)^2$.
Consider the tail inequality \eqref{eq:tail-ineq} evaluated at $\delta=\delta_0(m,\alpha)$.
Then the left-hand side is (weakly) \emph{increasing} in $\alpha\in(0,1]$, while the right-hand side is (weakly)
\emph{decreasing}.
Therefore it suffices to verify \eqref{eq:tail-ineq} at $\alpha=1$ and $\delta=\delta_0(m,1)$.
If one later shrinks $\delta\le\delta_0(m,\alpha)$ to enforce $\kappa$--admissibility, the inequality only becomes easier
(Lemma~\ref{lem:delta-shrink-monotone}).
\end{lemma}

\begin{proof}
With $\delta=\delta_0(m,\alpha)=\eta\alpha/(\log m)^2$, the only $\alpha$-dependence in the left-hand side is through the factor $\delta L(m)$, which is increasing in $\alpha$,
so the left-hand side increases.
The right-hand side equals $c-\delta\cdot\Xi(m)$ for a nonnegative factor $\Xi(m)$ independent of $\alpha$, hence it decreases.
\end{proof}


\begin{remark}[No one-height reduction in $m$ under the pointwise UE exponent $p=1$]
\label{rem:no-one-height}
In v33, the (claimed) $\delta^{3/2}$ prefactor in Lemma~\ref{lem:upper-envelope} made the local contribution
scale like $\delta^{1/2}N_{\rm up}(m)$ at the nominal choice $\delta_0(m,\alpha)=\eta\alpha/(\log m)^2$,
leading to an expression essentially independent of $m$ and enabling a one-height reduction.
After the UE--Gate audit, Lemma~\ref{lem:upper-envelope} provides only the pointwise exponent $p=1$,
so the tail left-hand side contains the $\delta$--inert term $(2C_{\rm up}/\kappa)\,N_{\rm up}(m)$.
With the explicit majorant $N_{\rm up}(m)=1.01\log m+17$, this term is \emph{increasing} in $m$.
Therefore a one-height reduction in $m$ is not available under the current pointwise envelope mechanism:
the tail criterion must be controlled as a family in $m$, or the UE--Gate must be cleared by a strengthened
envelope mechanism (Remark~\ref{rem:UE-gate}).
\end{remark}

\section{S5$'$ frontier: phase/winding endpoint redesign (open)}
\label{sec:S5-frontier}



% ============================================================
% OPEN (Missing Lever): budget-eligible S5' endpoint (v37)
% ============================================================
\begin{center}
\fbox{\begin{minipage}{0.96\linewidth}
\textbf{OPEN (Missing Lever): budget--eligible S5$'$ endpoint.}\par
The v37 S5$'$ toolkit supplies: (i) a forceable \emph{phase witness} on a buffered contour (Definition~\ref{def:Db-tilde-phase}, Lemma~\ref{lem:phase-forcing-argprinciple}),
(ii) a residual phase budget $O(\delta\log m)$ (Corollary~\ref{cor:residual_phase_budget}),
and (iii) a local phase bound that is $\delta$--inert with exponent $\theta=0$ \emph{in the phase class} (Lemma~\ref{lem:local_phase_delta_inert}).
What is missing is a single analytic lever that makes the envelope side shrink below an $O(1)$ forcing margin \emph{uniformly in $m$} at the nominal scale
$\delta_0(m,\alpha)=\eta\alpha/(\log m)^2$.

\smallskip
\noindent\textbf{Required deliverable (any one is decisive):}
\begin{enumerate}[leftmargin=1.5em]
\item a forceable endpoint class $\Phi_B$ and an upper-envelope inequality with exponent $p\ge \tfrac12$, together with a local interface bound in the \emph{same} endpoint class with exponent $\theta\le p-\tfrac12$ (ideally $\theta=0$);
\item or a micro-window clustering bound $N_{\rm micro}(m,\delta)=O(1)$ that reduces the effective local growth $q$ in the exponent budget (Theorem~\ref{thm:S5prime-closure});
\item or a pair-isolation mechanism showing only the forced pair contributes $O(1)$ to the phase endpoint while the remaining local cluster contributes $O(\delta\log m)$.
\end{enumerate}

\smallskip
\noindent\textbf{No--GO reminders (binding):} raw phase increment endpoints with no $\delta^p$ prefactor have $p=0$ and cannot close under constant-limited forcing; any endpoint whose proof reduces to an \emph{absolute} $L^r(\partial B)$ collar estimate is rejected by the v36 NO--GO class.
\end{minipage}}
\end{center}


\begin{remark}[S5 acceptance criterion (budget calculus; no drift)]
\label{rem:S5-accept}
Any proposed S5 redesign must specify a boundary functional $\Phi_B$ (shape--invariant; $\delta$--normalized)
and prove two explicit inequalities uniformly for all $m\ge 10$, all $\alpha\in(0,1]$, and all
$\kappa$--admissible $0<\delta\le \delta_0(m,\alpha)=\eta\alpha/(\log m)^2$:

\begin{enumerate}[leftmargin=1.5em]
\item \textbf{(S5--UE)} a forceable upper-envelope bound
\[
D_B(W)\le C_{\rm up}\,\delta^{p}\,\Phi_B(E'/E)
\]
with an explicit exponent $p>0$ and $\delta$--uniform constant $C_{\rm up}$;

\item \textbf{(S5--LOC)} a local/collar bound in the same endpoint class
\[
\Phi_B(Z'_{\rm loc}/Z_{\rm loc})\le C_{\rm loc}\,\delta^{-\theta}\,G(N_{\rm loc}(m),\kappa)
\]
with explicit $\theta\ge 0$ and an explicit growth model for $G$ (e.g.\ $G(n,\kappa)\ll \kappa^{-u}n^q$).
\end{enumerate}

The redesign is budget--viable for uniform $\eta$--shrinking closure under $\delta_0$
only if the S5 Budget Theorem yields $2(p-\theta)\ge q$ (and $p-\theta>0$ for shrinkability).
If $p-\theta<0$, the standard $\kappa$--admissible shrink policy is unsafe (shrinking $\delta$ can
increase the envelope term) and must be redesigned.

Finally, the forcing chain remains phrased in terms of $D_B(W)$; therefore S5 must include either
$\Phi_B\ge D_B(W)$ on all admissible boxes or a new forcing lemma that lower-bounds $\Phi_B$
directly (Remark~\ref{rem:s5-forceability-gate}).
\end{remark}


\begin{theorem}[S5 Budget Theorem (general endpoint functional)]
\label{thm:S5-budget}
Fix $\eta\in(0,1]$ and $\kappa\in(0,1/10)$ and define the nominal scale $\delta_0(m,\alpha)=\eta\alpha/(\log m)^2$.
Let $\Phi_B$ be a boundary functional and assume that for every $m\ge 10$, $\alpha\in(0,1]$, and every
$\kappa$--admissible $0<\delta\le \delta_0(m,\alpha)$ one has:
\begin{enumerate}[leftmargin=1.5em]
\item[(i)] \textbf{(S5--UE)} $D_B(W)\le C_{\rm up}\,\delta^{p}\,\Phi_B(E'/E)$ for some $p>0$ and $\delta$--uniform constant $C_{\rm up}$;
\item[(ii)] \textbf{(S5--SPLIT)} $\Phi_B(E'/E)\le \mathrm{Res}(m)+\Phi_B(Z'_{\rm loc}/Z_{\rm loc})$;
\item[(iii)] \textbf{(S5--LOC)} $\Phi_B(Z'_{\rm loc}/Z_{\rm loc})\le C_{\rm loc}\,\delta^{-\theta}\,G(N_{\rm loc}(m),\kappa)$ for some $\theta\ge 0$ and $\delta$--uniform $C_{\rm loc}$.
\end{enumerate}
Assume moreover that $N_{\rm loc}(m)\le A_N\log m+B_N$ and $\mathrm{Res}(m)\le A_L(\log m)^{r_L}+B_L$ for absolute constants,
and that for some $q,u\ge 0$ one has the growth model
\[
G(n,\kappa)\le C_G\,\kappa^{-u}\,n^q\qquad (n\ge 1),
\]
with $C_G$ independent of $(m,\alpha,\delta)$.

Then at the nominal choice $\delta=\delta_0(m,\alpha)$,
\begin{equation}
\label{eq:S5-budget}
D_B(W)\ \le\
C_{\rm up}\Bigl(\delta_0^{p}\,\mathrm{Res}(m)\ +\ C_{\rm loc}\,\delta_0^{p-\theta}\,G(N_{\rm loc}(m),\kappa)\Bigr).
\end{equation}
Furthermore:
\begin{enumerate}[leftmargin=1.5em]
\item \textbf{(Uniformity in $m$)} The local contribution in \eqref{eq:S5-budget} is uniformly bounded in $m\ge 10$ only if
\begin{equation}
\label{eq:S5-budget-condition}
2(p-\theta)\ \ge\ q.
\end{equation}
\item \textbf{($\eta$--shrinkability)} If \eqref{eq:S5-budget-condition} holds and $p-\theta>0$, then
\[
\sup_{m\ge 10}\ \delta_0(m,1)^{p-\theta}\,G(N_{\rm loc}(m),\kappa)\ =\ O\big(\eta^{p-\theta}\big),
\]
so the local penalty can be made arbitrarily small by choosing $\eta$ sufficiently small.
\item \textbf{($\delta$--shrink monotonicity)} If $p\ge 0$ and $p-\theta\ge 0$, then the right-hand side of \eqref{eq:S5-budget}
is nondecreasing in $\delta$ (for fixed $m,\alpha$); hence replacing $\delta_0$ by a smaller $\kappa$--admissible $\delta\le\delta_0$
can only improve the inequality. If $p-\theta<0$, $\kappa$--shrinking can \emph{worsen} the envelope term.
\end{enumerate}
\end{theorem}

\begin{proof}
Combine (S5--UE) with (S5--SPLIT) and (S5--LOC) to obtain
\[
D_B(W)\le C_{\rm up}\,\delta^{p}\,\mathrm{Res}(m)+C_{\rm up}\,C_{\rm loc}\,\delta^{p-\theta}\,G(N_{\rm loc}(m),\kappa).
\]
Set $\delta=\delta_0(m,\alpha)$ to obtain \eqref{eq:S5-budget}.

For the local contribution at $\alpha=1$ use $\delta_0=\eta/(\log m)^2$, the growth model $G(n,\kappa)\le C_G\kappa^{-u}n^q$,
and $N_{\rm loc}(m)\ll\log m$ to get
\[
\delta_0^{p-\theta}\,G(N_{\rm loc}(m),\kappa)
\ \ll\
\kappa^{-u}\,\eta^{p-\theta}\,(\log m)^{-2(p-\theta)}\,(\log m)^{q}
=\kappa^{-u}\,\eta^{p-\theta}\,(\log m)^{q-2(p-\theta)}.
\]
This is uniformly bounded in $m$ only if $q-2(p-\theta)\le 0$, i.e.\ \eqref{eq:S5-budget-condition}.
If additionally $p-\theta>0$, the factor $\eta^{p-\theta}$ yields $\eta$--shrinkability.

Finally, the monotonicity claim follows because $\delta\mapsto \delta^{p}$ and $\delta\mapsto \delta^{p-\theta}$ are
nondecreasing on $(0,\infty)$ exactly when $p\ge 0$ and $p-\theta\ge 0$.
\end{proof}

% ============================================================
% v37: S5' closure criterion (budget spine) + acceptance gate
% ============================================================

\begin{theorem}[S5$'$ closure from a forceable phase endpoint]\label{thm:S5prime-closure}
Fix $\kappa\in(0,1/10)$ and $\eta\in(0,1)$ and define $\delta_0(m,\alpha)=\eta\alpha/(\log m)^2$.
Let $\widetilde D_B$ be a boundary phase endpoint functional assigned to each $\kappa$--admissible
aligned box $B=B(\pm a,m,\delta)$ and its boundary quotient $W$.
Assume:
\begin{enumerate}[leftmargin=1.5em]
  \item[(S5$'$--FORCE)] Under an off-axis quartet at height $m$ with displacement $a>0$,
  there exists an aligned $\kappa$--admissible box $B$ (with $\alpha\approx a$) such that
  $\widetilde D_B(W)\ge c_{\rm force}-\delta\,\Xi(m)$ with $c_{\rm force}>0$ absolute and
  $\Xi(m)\ge 0$ explicit.
  \item[(S5$'$--UE+SPLIT)] For every $\kappa$--admissible aligned box,
  \[
    \widetilde D_B(W)\le C_{\rm up}\,\delta^p\Big(\mathrm{Res}(m)+C_{\rm loc}\,\delta^{-\theta}\,G(N_{\rm loc}(m),\kappa)\Big),
  \]
  where $p>0$, $\theta\ge 0$, and $C_{\rm up},C_{\rm loc}$ are $\delta$--uniform, and
  $G(n,\kappa)\le C_G\,\kappa^{-u}n^q$ for fixed $u,q\ge 0$.
\end{enumerate}
Suppose additionally that $2p\ge 1$, $2(p-\theta)\ge q$, and $p-\theta>0$.
Then there exists $\eta_\star\in(0,1)$ (depending on the displayed constants and $\kappa$) such that
for every $\eta\in(0,\eta_\star]$ the S5$'$ tail inequality holds at $\delta=\delta_0(m,\alpha)$
for all $m\ge 10$ and all $\alpha\in(0,1]$, and hence no off-axis quartet exists at any height $m\ge 10$.
Combined with any finite-height front-end, this implies RH.
\end{theorem}

\begin{remark}[S5$'$ acceptance gate for phase endpoints (no drift)]\label{rem:S5prime-gate}
Any proposed S5$'$ endpoint built from boundary phase data (e.g.\ $\Delta\arg$ or an integral of
$\Im(\log\text{-derivative})$) must declare its exponent budget data $(p,\theta,q)$ in the
schematic bound
\[
  \widetilde D_B(W)\le C_{\rm up}\,\delta^p\Big(\mathrm{Res}(m)+C_{\rm loc}\,\delta^{-\theta}\,G(N_{\rm loc}(m),\kappa)\Big),
\]
and must satisfy the uniformity/shrink conditions of Theorem~\ref{thm:S5prime-closure}:
$2p\ge 1$, $2(p-\theta)\ge q$, and $p-\theta>0$.
Pure $\Delta\arg$ endpoints have $p=0$ and are rejected.
Any phase endpoint whose proof reduces to an absolute $L^r(\partial B)$ estimate for $E'/E$
is also rejected by Lemma~\ref{lem:S5_Lp_collapse} and Proposition~\ref{prop:S5_Lp_nogo}.
\end{remark}




At fixed $(m,\alpha)$ the tail inequality \eqref{eq:tail-ineq} is a strict forcing--vs--envelope condition.
In v37 (inherited from v36) the combination of Theorem~\ref{thm:exponent-budget}, Lemma~\ref{lem:UE-scaling-nogo}, and Lemma~\ref{lem:force-constant-limited}
formally rules out the former ``$\eta$--absorption'' closure route based on the pointwise/sup endpoint
$\sup_{\partial B}|E'/E|$ together with the pointwise collar bound.

\medskip
\noindent\textbf{What must change.}
The forcing chain produces a lower bound for the \emph{dial deviation}
\[
D_B(W)\ :=\ \sum_{\pm}\bigl|W(v_\pm)-e^{i\varphi_0^\pm}\bigr|
\]
appearing in Lemma~\ref{lem:upper-envelope}.
In the current architecture this deviation is upper-bounded by a pointwise endpoint
$\delta\,\sup_{\partial B}|E'/E|$, which (via the collar) introduces the sharp $\delta^{-1}$ blow--up.
To obtain a tail closure mechanism one must redesign the envelope endpoint and/or the local interface so that the exponent budget
$p-\theta\ge \tfrac12$ is met \emph{uniformly in $m$}.

\begin{remark}[Forcing compatibility for redesigned endpoints]
\label{rem:forceability}
The existing forcing chain lower-bounds $D_B(W)$ (via the pair-factor phase rotation) by a fixed constant $c$ up to $\delta$--small corrections.
If one proposes a redesigned envelope endpoint $\Phi_B$ (non-pointwise, e.g.\ an $L^2$ or energy functional),
then the current forcing lower bound is useful only if it implies a corresponding lower bound for $\Phi_B$.
A sufficient (and simplest) compatibility condition is:
\[
\Phi_B\ \ge\ D_B(W)\qquad\text{for all admissible boxes and quotients }W,
\]
so that the forcing lower bound propagates unchanged.
If this domination fails, then a \emph{new forcing lemma} must be proved that lower-bounds $\Phi_B$ directly.
\end{remark}




\begin{lemma}[Forceability transfer by domination]
\label{lem:forceability-domination}
Let $B$ be a $\kappa$--admissible aligned box and $W$ the associated boundary quotient.
Suppose a boundary endpoint functional $\Phi_B$ satisfies
\[
\Phi_B \ge D_B(W)
\quad\text{for all admissible }(B,W).
\]
Then the existing single--box forcing lower bound for $D_B(W)$ implies the same forcing lower bound
for $\Phi_B$ with no change in the forcing constants.
\end{lemma}


\begin{remark}[Forceability gate for S5 endpoints (NO--GO unless met)]
\label{rem:s5-forceability-gate}
The current forcing architecture (Section~\ref{sec:forcing}) forces only the dial deviation $D_B(W)$
by an $O(1)$ constant up to $\delta$--small deductions (Lemma~\ref{lem:force-constant-limited}).
Consequently, any S5 redesign that replaces $D_B(W)$ by a different endpoint $\widetilde D_B$ (or $\Phi_B$)
is \emph{invalid} unless it proves either:
\begin{enumerate}[leftmargin=1.5em]
\item[(i)] $\widetilde D_B \ge D_B(W)$ for all admissible boxes/quotients (domination transfer), or
\item[(ii)] a new forcing lemma that lower--bounds $\widetilde D_B$ directly under an off--axis quartet.
\end{enumerate}
Without (i) or (ii), the forcing half of the tail inequality becomes logically disconnected from the envelope half.
\end{remark}

% ============================================================
% v37: S5' phase endpoint toolkit + buffered phase forcing
% Inserted after Remark~\ref{rem:s5-forceability-gate}.
% ============================================================

\subsection{S5$'$ phase endpoints: winding / argument-increment toolkit}\label{subsec:S5prime-phase-toolkit}

\begin{definition}[Phase increment on a boundary arc]\label{def:phase_increment}
Let $\Gamma\subset\mathbb{C}$ be a piecewise $C^1$ oriented curve and let $f$ be holomorphic
on an open neighborhood of $\Gamma$ with $f(v)\neq 0$ for all $v\in\Gamma$.
Define the phase increment of $f$ along $\Gamma$ by
\[
\Delta_{\Gamma}\arg f \ :=\ \Im\int_{\Gamma}\frac{f'(v)}{f(v)}\,dv.
\]
(Equivalently, $\Delta_{\Gamma}\arg f$ is the total change of a continuous branch of $\arg f$ along $\Gamma$.)
\end{definition}

\begin{lemma}[Phase increment identity and vertical specialization]\label{lem:phase_increment_identity}
Under the hypotheses of Definition~\ref{def:phase_increment}, the phase increment is additive under concatenation of curves and satisfies:
\begin{enumerate}[leftmargin=1.5em]
\item If $\Gamma=\Gamma_1\cup\Gamma_2$ (oriented concatenation), then $\Delta_{\Gamma}\arg f=\Delta_{\Gamma_1}\arg f+\Delta_{\Gamma_2}\arg f$.
\item If $\Gamma$ is the vertical segment $I_+:=\{\alpha+i y:\ |y-m|\le \delta\}$ oriented upward, then
\[
\Delta_{I_+}\arg f \ =\ \Im\int_{m-\delta}^{m+\delta}\frac{f'(\alpha+i y)}{f(\alpha+i y)}\,i\,dy
\ =\ \int_{m-\delta}^{m+\delta}\Re\!\left(\frac{f'(\alpha+i y)}{f(\alpha+i y)}\right)\,dy.
\]
\end{enumerate}
\end{lemma}

\begin{remark}[Parentheses hygiene for phase endpoints]\label{rem:phase_parentheses_hygiene}
For non-horizontal arcs, one must distinguish
\[
\Im\int_{\Gamma}\frac{f'}{f}\,dv
\qquad\text{from}\qquad
\int_{\Gamma}\Im\!\left(\frac{f'}{f}\right)\,dv.
\]
Only the former is a phase increment. This distinction is essential on vertical segments where $dv=i\,dy$.
\end{remark}

\begin{definition}[Shifted near-vertical segment]\label{def:Iplus_lambda}
Let $B=B(\alpha,m,\delta)$ be an aligned box and let $\lambda\in(0,1)$.
Define the shifted segment
\[
I_{+,\lambda}:=\{\alpha+\lambda\delta+i y:\ |y-m|\le \delta\},
\]
oriented upward. (This lies strictly inside $B$ and avoids the boundary by a distance $\asymp\lambda\delta$.)
\end{definition}

\begin{lemma}[Phase split on $I_{+,\lambda}$]\label{lem:phase_split_Iplus_lambda}
Let $B=B(\alpha,m,\delta)$ be $\kappa$--admissible and aligned, and let $I_{+,\lambda}$ be as in Definition~\ref{def:Iplus_lambda}.
Assume $E$, $Z_{\rm loc}$ and $F:=E/Z_{\rm loc}$ are holomorphic and nonvanishing on an open neighborhood of $I_{+,\lambda}$.
Then
\[
\Delta_{I_{+,\lambda}}\arg E \ =\ \Delta_{I_{+,\lambda}}\arg F\ +\ \Delta_{I_{+,\lambda}}\arg Z_{\rm loc}.
\]
Moreover,
\[
\bigl|\Delta_{I_{+,\lambda}}\arg F\bigr|
\ \le\ 2\delta\ \sup_{v\in I_{+,\lambda}}\left|\frac{F'(v)}{F(v)}\right|
\ \le\ 2\delta\ \sup_{v\in\partial B}\left|\frac{F'(v)}{F(v)}\right|.
\]
\end{lemma}

\begin{corollary}[Residual phase budget ($\delta$--uniform)]\label{cor:residual_phase_budget}
Assume the residual envelope bound of Lemma~\ref{lem:residual-envelope}, i.e.
$\sup_{\partial B}|F'/F|\le C_1\log m+C_2$ on every $\kappa$--admissible aligned box.
Then, for every $\lambda\in(0,1)$,
\[
\bigl|\Delta_{I_{+,\lambda}}\arg F\bigr|\ \le\ 2\delta\,(C_1\log m+C_2).
\]
\end{corollary}

\begin{lemma}[Local phase is $\delta$--inert: per-zero contribution is $O(1)$]\label{lem:local_phase_delta_inert}
Let $I_+=\{\alpha+i y:\ |y-m|\le \delta\}$ be oriented upward and let $\rho\notin I_+$.
Then
\[
\left|\Im\int_{I_+}\frac{dv}{v-\rho}\right|
=\left|\arg(\alpha+i(m+\delta)-\rho)-\arg(\alpha+i(m-\delta)-\rho)\right|
\le \pi.
\]
Consequently, writing $Z_{\rm loc}(v)=\prod_{\rho\in Z_{\rm loc}(m)}(v-\rho)^{m_\rho}$,
\[
\bigl|\Delta_{I_+}\arg Z_{\rm loc}\bigr|
=\left|\Im\int_{I_+}\frac{Z'_{\rm loc}(v)}{Z_{\rm loc}(v)}\,dv\right|
\le \pi\,N_{\rm loc}(m).
\]
\end{lemma}

\begin{corollary}[Prototype phase upper bound (residual + local)]\label{cor:prototype_phase_upper}
Under the hypotheses of Lemma~\ref{lem:phase_split_Iplus_lambda} and Corollary~\ref{cor:residual_phase_budget},
\[
\bigl|\Delta_{I_{+,\lambda}}\arg E\bigr|
\le 2\delta\,(C_1\log m+C_2)\ +\ \bigl|\Delta_{I_{+,\lambda}}\arg Z_{\rm loc}\bigr|.
\]
In particular, the residual contribution is $O(\delta\log m)$ while the local contribution is $\delta$--inert in the phase class.
\end{corollary}

% ------------------------------------------------------------
% Buffered boundary phase endpoint + forcing lemma (π/2)
% ------------------------------------------------------------

\begin{definition}[Buffered boundary phase endpoint]\label{def:Db-tilde-phase}
Let $B=B(\alpha,m,\delta)$ be an aligned box and assume $\kappa$--admissibility:
$\dist(\partial B, Z(E))\ge \kappa\delta$.
Let $G_{\rm out}$ be the Dirichlet outer factor on $B^\circ$ and $W:=E/G_{\rm out}$ the inner quotient.
Define the buffered inner rectangle
\[
B_{\kappa/2}:=\{v\in B:\dist(v,\partial B)\ge \tfrac{\kappa\delta}{2}\},
\]
and write its oriented boundary as $\partial B_{\kappa/2}=\bigcup_{j=1}^4 S_j$ (counterclockwise).
Define the sidewise phase increment
\[
\Delta_{S_j}\arg W \ :=\ \Im\int_{S_j}\frac{W'(v)}{W(v)}\,dv,
\]
and the phase endpoint
\[
\widetilde D_B(W):=\max_{1\le j\le 4}\Bigl|\Delta_{S_j}\arg W\Bigr|.
\]
\end{definition}

\begin{lemma}[Phase forcing from an interior zero]\label{lem:phase-forcing-argprinciple}
Assume the setup of Definition~\ref{def:Db-tilde-phase}. If $W$ has at least one zero in
$B_{\kappa/2}^\circ$ (equivalently $E$ has at least one zero there), then
\[
\widetilde D_B(W)\ge \frac{\pi}{2}.
\]
\end{lemma}

\begin{proof}
Since $W$ is holomorphic and nonvanishing on a neighborhood of $\partial B_{\kappa/2}$,
the argument principle gives
\[
\oint_{\partial B_{\kappa/2}} \frac{W'(v)}{W(v)}\,dv = 2\pi i\,N,
\]
where $N\ge 1$ is the number of zeros of $W$ in $B_{\kappa/2}^\circ$, counted with multiplicity.
Taking imaginary parts and decomposing $\partial B_{\kappa/2}$ into four sides yields
\[
\sum_{j=1}^4 \Delta_{S_j}\arg W = 2\pi N.
\]
Hence
\[
\widetilde D_B(W)\ \ge\ \frac{1}{4}\left|\sum_{j=1}^4 \Delta_{S_j}\arg W\right|
\ =\ \frac{\pi N}{2}\ \ge\ \frac{\pi}{2}.
\]
\end{proof}

\begin{remark}[Forceability gate for phase endpoints]\label{rem:forceability-phase-gate}
The v36 forcing chain lower-bounds the dial deviation $D_B(W)$.
The phase endpoint $\widetilde D_B(W)$ in Definition~\ref{def:Db-tilde-phase} is not known to dominate
$D_B(W)$ or vice-versa. Therefore any redesign that replaces $D_B(W)$ by $\widetilde D_B(W)$ must
either:
(i) rewrite the tail inequality so that $\widetilde D_B(W)$ is the forced object (using
Lemma~\ref{lem:phase-forcing-argprinciple}), or
(ii) prove a transfer inequality relating $\widetilde D_B(W)$ and $D_B(W)$ on all admissible boxes.
Without (i) or (ii), forcing and envelope are logically disconnected.
\end{remark}




\subsection{Baseline NO--GO results for naive non-pointwise endpoints}
\label{subsec:S5-nogo-baseline}

The S5 goal is to replace the pointwise/sup endpoint in Lemma~\ref{lem:upper-envelope}
by a non--pointwise functional that still controls the same dial deviation $D_B(W)$ appearing in the forcing chain.
The next two results prevent drift into two large endpoint classes that cannot work under the present $D_B(W)$ target
and the v36 local split/collar interface (unchanged from v35).


\begin{lemma}[Absolute $L^r$ endpoint scaling collapse]
\label{lem:S5_Lp_collapse}
Let $B=B(\pm a,m,\delta)$ be an aligned box and let $G_{\rm out}$ and $W=E/G_{\rm out}$ be as in
Lemma~\ref{lem:upper-envelope}. Assume boundary contact so that $W$ has unimodular boundary values a.e.
Fix $r\in[1,\infty]$ and write $L^r(\partial B)$ for $L^r(\partial B,ds)$.
Then there exists a shape-only constant $C_r>0$ (depending only on the normalized square $Q=[-1,1]^2$)
such that for each sign $\pm$,
\begin{equation}
\label{eq:S5-Lr-UE}
\bigl|W(v_\pm)-e^{i\varphi_0^\pm}\bigr|
\ \le\ C_r\,\delta^{\,1-1/r}\,
\Bigl\|\frac{E'}{E}\Bigr\|_{L^r(\partial B)}.
\end{equation}
In particular, any upper-envelope mechanism whose endpoint is an \emph{absolute} $L^r(\partial B)$ norm of $E'/E$
cannot have a $\delta$--prefactor exponent exceeding $p(r)=1-1/r$ within this endpoint class.
\end{lemma}

\begin{proof}
Repeat the proof of Lemma~\ref{lem:upper-envelope} with $L^2$ replaced by $L^r$ throughout.
Evaluation from the boundary gives $|W(v_\pm)-c|\le \|P_B(v_\pm,\cdot)\|_{L^q}\,\|W-c\|_{L^r}$ for $1/r+1/q=1$,
and under affine rescaling $\|P_B\|_{L^q}\asymp \delta^{-1/r}$.
Boundary Poincar\'e in $L^r$ yields $\|W-c\|_{L^r}\le C_r'\,\delta\,\|\partial_s W\|_{L^r}$ with a shape-only constant $C_r'$,
and outer factor control bounds $\|\partial_s W\|_{L^r}$ by a shape-only constant times $\|E'/E\|_{L^r}$.
Choosing $c=e^{i\varphi_0^\pm}$ gives \eqref{eq:S5-Lr-UE}, with overall factor $\delta^{-1/r}\cdot \delta=\delta^{1-1/r}$.
\end{proof}


\begin{proposition}[NO--GO: absolute $L^r$ log-derivative endpoints cannot clear the UE--Gate]
\label{prop:S5_Lp_nogo}
Assume in addition that $B$ is $\kappa$--admissible and hence the pointwise collar bound holds:
$\sup_{\partial B}|Z'_{\rm loc}/Z_{\rm loc}|\le N_{\rm loc}(m)/(\kappa\delta)$ (Lemma~\ref{lem:Zloc-logder-collar}).
Then for every $r\in[1,\infty]$,
\[
\delta^{\,1-1/r}\left\|\frac{Z'_{\rm loc}}{Z_{\rm loc}}\right\|_{L^r(\partial B)}
\ \le\ 8^{1/r}\,\frac{N_{\rm loc}(m)}{\kappa},
\]
independent of $\delta$.
In particular, under the nominal scale $\delta_0(m,\alpha)=\eta\alpha/(\log m)^2$ and the unconditional majorant
$N_{\rm loc}(m)\ll\log m$, uniform $\eta$--shrinking cannot suppress the local term within any envelope mechanism
whose endpoint is an absolute $L^r(\partial B)$ norm of $E'/E$.
\end{proposition}

\begin{proof}
Use $|\partial B|=8\delta$ and $\|f\|_{L^r}\le |\partial B|^{1/r}\|f\|_{L^\infty}$ to get
\[
\Bigl\|\frac{Z'_{\rm loc}}{Z_{\rm loc}}\Bigr\|_{L^r(\partial B)}
\le (8\delta)^{1/r}\cdot \frac{N_{\rm loc}(m)}{\kappa\delta}
=8^{1/r}\,\frac{N_{\rm loc}(m)}{\kappa\,\delta^{\,1-1/r}}.
\]
Multiply by $\delta^{\,1-1/r}$.
\end{proof}


\begin{remark}[Implication for S5 endpoint design]
\label{rem:S5_endpoint_implication}
Lemmas~\ref{lem:S5_Lp_collapse}--\ref{prop:S5_Lp_nogo} rule out the entire family of S5 proposals that attempt
to replace $\sup_{\partial B}|E'/E|$ by an \emph{absolute} $L^r(\partial B)$ norm of $E'/E$ while keeping the same
$\kappa$--collar local control. Any viable S5 redesign must instead (i) exploit cancellation (argument-principle style
\emph{signed} endpoints) and/or (ii) move to a less singular boundary object (e.g.\ endpoints built from $\log|E|$ / BMO-type control).
\end{remark}


\begin{lemma}[NO--GO: local-kernel projection endpoints cannot control $D_B(W)$ without a new forcing link]
\label{lem:S5-projection-nogo}
Fix an aligned box $B$ and consider an endpoint functional of the form
\[
\Phi_B(E)\ :=\ \|(I-\Pi_B)(E'/E)\|_{X(\partial B)}
\]
for some normed boundary space $X(\partial B)$ and a bounded projection $\Pi_B$ satisfying
$\Pi_B(Z'_{\rm loc}/Z_{\rm loc})=Z'_{\rm loc}/Z_{\rm loc}$ whenever $Z_{\rm loc}$ is the local factor
associated to $B$ (so that the local term is annihilated under the split $E'/E=F'/F+Z'_{\rm loc}/Z_{\rm loc}$).
Then there is no universal inequality of the form
\[
D_B(W)\ \le\ C\,\delta^{p}\,\Phi_B(E)
\]
(valid for all forcing-aligned boxes under the boundary-contact convention), for any fixed $p>0$ and constant $C$,
unless one supplies a new forcing link that lower-bounds $\Phi_B$ directly under an off-axis quartet.
\end{lemma}

\begin{proof}
This is a structural counterexample: in the class of holomorphic functions $E$ obeying the boundary-contact convention,
take $E=Z_{\rm loc}$ on a box for which $Z_{\rm loc}$ has a zero at one of the dial points $v_\pm$.
Then $F\equiv 1$ and $E'/E=Z'_{\rm loc}/Z_{\rm loc}$, so by assumption $(I-\Pi_B)(E'/E)=0$ and hence $\Phi_B(E)=0$.
However $G_{\rm out}$ is zero-free, so $W=E/G_{\rm out}$ shares the same interior zeros as $E$ and
$W(v_\pm)=0$ for at least one sign, giving $D_B(W)\ge 1$.
Thus no inequality $D_B(W)\le C\delta^{p}\Phi_B(E)$ can hold from these hypotheses alone; any attempt to use such an endpoint
must replace $D_B(W)$ as the forced object and provide a forcing-transfer lemma (Remark~\ref{rem:s5-forceability-gate}).
\end{proof}


\begin{remark}[Consequence for S5 searches]
\label{rem:S5-nogo-consequence}
Lemmas~\ref{prop:S5_Lp_nogo} and \ref{lem:S5-projection-nogo} close two broad endpoint classes:
(i) absolute $L^r(\partial B)$ norms of $E'/E$ (including $L^2$) under the current collar interface,
and (ii) endpoints that annihilate the local kernel span while still targeting the forced dial deviation $D_B(W)$.
Any viable S5 redesign must introduce a genuinely new local-interface input and/or a new forcing-compatible endpoint.
\end{remark}


\medskip
\noindent\textbf{S5 design targets (open).}
A future closure route (S5) should provide a non--pointwise endpoint $\Phi_B$ and a UE--type inequality of the schematic form
\[
D_B(W)\ \le\ C_{\rm up}\,\delta^{p}\,\Phi_B(E)
\qquad (p>0),
\]
together with a local/residual split of $\Phi_B(E)$ whose local contribution scales as $\delta^{-\theta}$ with $\theta<p-\tfrac12$,
or more generally satisfies the exponent budget of Theorem~\ref{thm:exponent-budget}.
The point is \emph{not} to recover the specific exponent $\tfrac32$ from older drafts, but to obtain any effective gain $p-\theta>\tfrac12$
with proof--grade uniformity.

\begin{remark}[Recorded open lemmas (S5 checklist)]
A proof--grade S5 implementation would minimally require:
\begin{enumerate}[leftmargin=1.5em]
\item \textbf{(S5--UE)} a redesigned upper-envelope inequality with a forceable endpoint $\Phi_B$;
\item \textbf{(S5--RES)} a $\delta$--uniform residual envelope bound in the same endpoint class;
\item \textbf{(S5--LOC)} a collar/local bound in the same endpoint class that avoids the pointwise $\delta^{-1}$ blow--up;
\item \textbf{(S5--FORCE)} either $\Phi_B\ge D_B(W)$ or a new forcing lemma as in Remark~\ref{rem:forceability}.
\end{enumerate}
\end{remark}


\section{Global RH from a finite front-end + the tail criterion family}
\label{sec:global}

\begin{theorem}[Global closure (criterion-first logical form)]
\label{thm:global}
Assume:
\begin{enumerate}[leftmargin=1.5em]
\item (Front-end) All nontrivial zeros with $0<\ImPart(s)\le 5$ lie on the critical line.
\item (Tail criterion) Fix some $\eta\in(0,1)$ and $\kappa\in(0,1/10)$, and assume the analytic inputs
Lemmas~\ref{lem:upper-envelope}--\ref{lem:Nloc-logm} and Lemma~\ref{lem:horizontal-budget} with finite constants.
Assume moreover that for every $m\ge 10$ and every $\alpha\in(0,1]$ there exists a $\kappa$--admissible
scale $0<\delta\le \delta_0(m,\alpha)=\eta\alpha/(\log m)^2$ such that the strict tail inequality \eqref{eq:tail-ineq} holds.
\end{enumerate}
Then all nontrivial zeros of $\zeta(s)$ lie on the critical line.
\end{theorem}

\begin{proof}
For each $m\ge 10$, Theorem~\ref{thm:tail-inequality} turns the strict inequality \eqref{eq:tail-ineq}
into exclusion of off-axis quartets at height $m$.
By the tail criterion hypothesis, no off-axis quartets exist at any height $m\ge 10$.
By the front-end hypothesis, there are no off-axis zeros below height $5$.
Hence there are no off-axis zeros at any height, so every nontrivial zero lies on the critical line.
\end{proof}

\begin{remark}[Role of computations and the repro pack (v37)]
Appendix~\ref{app:certificate} provides a small interval-arithmetic harness that evaluates the tail inequality
for pinned parameters and a pinned constant ledger. In v36 this is used only for audit purposes (e.g.\ exponent tracking),
not as a proof substitute.
\end{remark}

\appendix

\section{Discarded closure routes (as of v37)}
\label{app:discarded}

This appendix records closure routes that were explored in earlier iterations (v32--v34) but are now ruled out \emph{under the currently proved inputs}.
The purpose is to prevent future drift: these routes should not be re-opened unless a genuinely new analytic input (e.g.\ an S5 endpoint redesign) is supplied.

\subsection{D1: Pointwise UE endpoint $\,\sup_{\partial B}|E'/E|$ + collar + $\eta$--absorption (S1/S1$'$)}
The former absorption narrative attempted to close the tail family by shrinking $\eta$ in the nominal scale
$\delta_0(m,\alpha)=\eta\alpha/(\log m)^2$.
In the pointwise/sup architecture the UE step has exponent $p=1$ (Lemma~\ref{lem:upper-envelope}) and the collar/local split has exponent $\theta=1$ (Lemma~\ref{lem:Zloc-logder-collar}),
so the local contribution is $\delta$--inert and cannot be suppressed by $\eta$ (Lemma~\ref{lem:UE-d1-obstruction}).
More strongly, the exponent budget (Theorem~\ref{thm:exponent-budget}) shows that uniform $\eta$--shrinking requires $p-\theta\ge \tfrac12$,
while the scaling NO--GO (Lemma~\ref{lem:UE-scaling-nogo}) forbids any $p>1$ within this endpoint class.
Finally, the forcing margin is constant--limited in the single--box architecture (Lemma~\ref{lem:force-constant-limited}), so one cannot compensate by ``making forcing grow with $m$''.

\begin{proposition}[Historical record: formal anchor absorption under a hypothetical strengthened UE exponent]
\label{prop:eta-absorption-conditional-appx}
This proposition is \emph{not used} in v36. It is recorded only to document the logical shape of the discarded absorption idea.

Assume that, for some $p>1$, an upper-envelope step admits the strengthened form
\[
D_B(W)\ \le\ 2C_{\rm up}\,\delta^{p}\,\sup_{\partial B}\Bigl|\frac{E'}{E}\Bigr|
\]
with the same constant ledger, and that all other constants in \eqref{eq:tail-ineq} are finite.
Fix an anchor height $m_\star\ge 10$ and evaluate \eqref{eq:tail-ineq} at $(m,\alpha)=(m_\star,1)$ with the nominal scale
$\delta_0(m_\star,1)=\eta/(\log m_\star)^2$.
Then there exists $\eta_\star(m_\star,p)>0$ such that \eqref{eq:tail-ineq} holds at $(m_\star,1)$ for every $\eta\in(0,\eta_\star]$.

\smallskip
\noindent\emph{Warning:} within the pointwise/sup endpoint class, Lemma~\ref{lem:UE-scaling-nogo} forbids any $p>1$, so this proposition cannot be invoked without an S5 redesign.
\end{proposition}

\begin{proof}
Under a strengthened exponent $p>1$, the envelope side becomes
$A\,\eta^{p}+B\,\eta^{p-1}$ for finite constants $A,B$ depending on $(m_\star,p)$ and the constant ledger,
while the forcing side equals $c-D\,\eta$ for a finite $D$.
Since $p>1$, one has $\eta^{p}\to 0$, $\eta^{p-1}\to 0$, and $\eta\to 0$ as $\eta\downarrow 0$,
so the strict inequality holds for all sufficiently small $\eta$.
\end{proof}

\subsection{D2: Shrinking the local window / short-interval zero counts}
A tempting workaround is to replace the fixed local window $|\gamma-t|\le 1$ in the residual/collar interface by a shrinking window
$|\gamma-t|\le \delta^\beta$ to reduce the local term.
However, without additional analytic input, available RH--free methods control $N(t+1)-N(t-1)$ at unit scale and do \emph{not} provide a proof--grade bound for
$N(t+\delta^\beta)-N(t-\delta^\beta)$ as $\delta\downarrow 0$.
Thus v36 does not pursue window-shrinking as a substitute for the missing UE gain.

\subsection{D3: ``Make forcing grow with $m$'' within single-box forcing}
Because $\Delta_{I_+}\arg Z_{\rm pair}\le 2\pi$ uniformly (Lemma~\ref{lem:force-constant-limited}), the forcing constant $c$ in the tail inequality is $O(1)$.
Any attempt to obtain a forcing side that grows like $\log m$ (or any unbounded function of $m$) would require a different forcing architecture (not the v36 single-box forcing chain).

\subsection{D4: ``Boundary modulus implies interior zero-freeness'' converse}
Under boundary-contact, the quotient $W=E/G_{\rm out}$ satisfies $|W|=1$ on $\partial B$ (Remark~\ref{rem:boundary_modulus}),
but this has no converse implication toward zero-freeness or constancy (Remark~\ref{rem:no-converse}).
Therefore, any closure route that implicitly treats $|W|=1$ as ``almost zero-free'' is invalid.





\subsection{D5: Absolute $L^r$ log-derivative endpoints (NO--GO)}
Replacing the pointwise endpoint $\sup_{\partial B}|E'/E|$ by an \emph{absolute} boundary $L^r(\partial B)$ norm of $E'/E$
does not improve the exponent budget: Lemma~\ref{lem:S5_Lp_collapse} forces the UE prefactor exponent to be
$p(r)=1-1/r$, while Proposition~\ref{prop:S5_Lp_nogo} shows the local/collar contribution has the same exponent
$\theta(r)=1-1/r$, hence $p(r)-\theta(r)=0$ and the local leakage is $\delta$--inert.

\subsection{D6: Projecting out the local kernel span (NO--GO)}
A tempting idea is to define an endpoint by projecting $E'/E$ off the span of local Cauchy kernels so that the local term vanishes.
Lemma~\ref{lem:S5-projection-nogo} shows this cannot control the forced dial deviation $D_B(W)$ without changing the contradiction endpoint
or supplying a new forcing link.

\medskip
\noindent\emph{Supporting documentation for D6 (not a viable endpoint under current forcing).}
The next definition and lemmas formalize the projection setup and the exact cancellation of the local term.
They are recorded only to document the mechanism behind the NO--GO.



\begin{definition}[Local Cauchy subspace and $L^2$ projection (supporting documentation)]
\label{def:KB-projection}
Let $B=B(\alpha,m,\delta)$ be $\kappa$--admissible and let $Z_{\rm loc}(m)$ denote the multiset
of zeros of $E$ used to define $Z_{\rm loc}$ (counted with multiplicity).
Define the finite-dimensional subspace
\[
K_B:=\mathrm{span}\{\,k_\rho:\partial B\to\mathbb{C},\ k_\rho(v)=(v-\rho)^{-1}\ :\ \rho\in Z_{\rm loc}(m)\,\}
\subset L^2(\partial B,ds),
\]
and let $\Pi_B:L^2(\partial B)\to K_B$ be the orthogonal projection.
\end{definition}


\begin{lemma}[Projection kills the local log-derivative (supporting documentation)]
\label{lem:proj-kills-Zloc}
With notation as in Definition~\ref{def:KB-projection},
\[
\frac{Z'_{\rm loc}}{Z_{\rm loc}}(v)=\sum_{\rho\in Z_{\rm loc}(m)}\frac{m_\rho}{v-\rho}\in K_B
\quad (v\in\partial B).
\]
Hence $\Pi_B(Z'_{\rm loc}/Z_{\rm loc})=Z'_{\rm loc}/Z_{\rm loc}$ and
$(I-\Pi_B)(Z'_{\rm loc}/Z_{\rm loc})=0$ in $L^2(\partial B)$ (and thus pointwise on $\partial B$).
Consequently, using Lemma~\ref{lem:logder-split},
\[
(I-\Pi_B)\!\left(\frac{E'}{E}\right)=(I-\Pi_B)\!\left(\frac{F'}{F}\right)
\quad\text{on }\partial B.
\]
Moreover $\|\Pi_B\|_{L^2\to L^2}=1$.
\end{lemma}


\begin{remark}[Conditioning caveat for coefficient representations (supporting documentation)]
\label{rem:proj-conditioning}
Lemma~\ref{lem:proj-kills-Zloc} uses only the abstract orthogonal projection $\Pi_B$ (a contraction).
No uniform bound on the inverse Gram matrix of the spanning kernels $k_\rho$ is available without
a lower bound on pairwise zero separations in $Z_{\rm loc}(m)$.
Therefore any coefficient-level formula for $\Pi_B$ must be treated as non-uniform unless additional
spacing structure is proved.
\end{remark}



\section{S6 harness: explicit-formula interpretation (non-closure)}
\label{app:S6-harness}

This appendix is an \emph{interpretation harness only}.  It is not used in any implication in the manuscript.
Its purpose is to connect the v--frame ``off-axis'' language to the classical explicit formula for prime-counting functions.

\subsection{Frame mapping: v--displacement and the real part $\beta$}
A nontrivial zero $\rho=\beta+i\gamma$ in the s--frame corresponds to
\[
u_\rho=2\rho=2\beta+i\,2\gamma,
\qquad
v_\rho=u_\rho-1=(2\beta-1)+i\,2\gamma.
\]
Thus an off-critical-line zero ($\beta\neq \tfrac12$) is exactly an off-axis v--zero ($\Re v_\rho\neq 0$),
with displacement $a:=\Re v_\rho=2\beta-1$.

\subsection{Explicit formula: off-axis zeros as amplitude leaks}
In a standard explicit formula (e.g.\ for $\psi(x)=\sum_{n\le x}\Lambda(n)$), nontrivial zeros enter through terms of the form
$x^\rho/\rho$ (or $\operatorname{Li}(x^\rho)$).  If $\rho=\beta+i\gamma$, then
\[
x^\rho = x^\beta e^{i\gamma\log x},
\]
so the amplitude is governed by $x^\beta$.  In v--variables, $\beta=\tfrac12+\tfrac{a}{2}$, so any $a>0$
corresponds to an $x^{1/2+a/2}$-scale contribution (an ``amplitude leak'' beyond the square-root scale).

\subsection{What S5$'$ would mean in this language}
A successful S5$'$ closure would exclude all off-axis v--zeros, hence prove RH and thereby eliminate all amplitude leaks with $\beta>1/2$.
However, the present manuscript does \emph{not} claim any new prime-error bounds directly: the S6 harness is only a translation layer
for interpreting off-axis zeros in the classical explicit-formula setting.


\section{Tail harness bundle and reproducibility (v37)}
\label{app:certificate}

\subsection{What the tail checks prove (and what they do not)}
\label{app:what-proves}

Each tail check records the statement:
\begin{quote}
Given a constants file that provides interval enclosures for
$(C_1,C_2,C_{\mathrm{up}},C_h'',\kappa)$, the chosen parameters $(m,\eta,\alpha)$,
and the recorded UE exponent $p$,
the harness computes interval bounds for the left-hand side $\mathrm{LHS}$ and right-hand side $\mathrm{RHS}$ in
\eqref{eq:tail-ineq} and reports whether the strict separation $\mathrm{LHS}_{\mathrm{hi}}<\mathrm{RHS}_{\mathrm{lo}}$ holds.
\end{quote}


It does \emph{not} certify that the constants file is correct.

\subsection{SHA--256 table (exact artifacts)}
\label{app:sha}

The file \texttt{v36\_repro\_pack/SHA256SUMS.txt} is the canonical hash list.

\lstinputlisting{v37_repro_pack/SHA256SUMS.txt}

\subsection{Commands}
\label{app:commands}

From the directory \texttt{v36\_repro\_pack/}:
\begin{enumerate}[leftmargin=1.5em]
\item \texttt{sha256sum -c SHA256SUMS.txt}
\item \texttt{python3 v36\_verify\_tail\_check.py --constants v36\_constants\_m10.json --certificate v36\_tail\_check\_m10.json}
\item \texttt{python3 v36\_verify\_frontend\_certificate.py --certificate v36\_frontend\_certificate.json}
\end{enumerate}

\subsection{Expected verifier output: $m=10$ (verbatim; may report strict inequality as false)}
\label{app:verifier-output-10}
\lstinputlisting{v37_repro_pack/v37_tail_check_verifier_output_m10.txt}

\subsection{Bundle files (verbatim)}
\label{app:bundle-files}

\paragraph{v36 low-anchor constants (intervals; audit harness).}
\lstinputlisting{v37_repro_pack/v37_constants_m10.json}

\paragraph{v36 low-anchor tail check ($m=10$).}
\lstinputlisting{v37_repro_pack/v37_tail_check_m10.json}

\paragraph{Generator implementation (directed rounding).}
\lstinputlisting{v37_repro_pack/v37_generate_tail_check.py}

\paragraph{Verifier implementation.}
\lstinputlisting{v37_repro_pack/v37_verify_tail_check.py}

\section{Finite-height front-end certificate (literature-based)}
\label{app:frontend}

The required front-end is RH up to height $H_0=5$.
We record a discharge using Platt--Trudgian's published verification of RH up to
$3\cdot 10^{12}$.

\paragraph{Pinned front-end certificate JSON.}
\lstinputlisting{v37_repro_pack/v37_frontend_certificate.json}

\paragraph{Front-end verifier output (internal logic only).}
\lstinputlisting{v37_repro_pack/v37_frontend_verifier_output.txt}

\paragraph{Generator/verifier scripts.}
\lstinputlisting{v37_repro_pack/v37_generate_frontend_certificate.py}
\lstinputlisting{v37_repro_pack/v37_verify_frontend_certificate.py}
\section*{References}
\addcontentsline{toc}{section}{References}

\begin{thebibliography}{99}

\bibitem{CMM82}
R.~Coifman, A.~McIntosh, and Y.~Meyer,
\emph{L'int\'egrale de Cauchy d\'efinit un op\'erateur born\'e sur $L^2$ pour les courbes lipschitziennes},
Annals of Mathematics (2) \textbf{116} (1982), no.~2, 361--387.

\bibitem{DriscollTrefethenSC}
T.~A. Driscoll and L.~N. Trefethen,
\emph{Schwarz--Christoffel Mapping},
Cambridge Monographs on Applied and Computational Mathematics, Cambridge University Press, 2002.

\bibitem{DurenHp}
P.~L. Duren,
\emph{Theory of $H^p$ Spaces},
Academic Press, 1970.

\bibitem{GarnettBaf}
J.~B. Garnett,
\emph{Bounded Analytic Functions},
Graduate Texts in Mathematics, Springer, 2007.

\bibitem{Ivic}
A.~Ivi\'c,
\emph{The Riemann Zeta-Function: Theory and Applications},
Wiley-Interscience, 1985.

\bibitem{Titchmarsh}
E.~C. Titchmarsh,
\emph{The Theory of the Riemann Zeta-Function}, 2nd ed., revised by D.~R. Heath-Brown,
Oxford University Press, 1986.

\bibitem{BellottiWongZeta2024}
A.~Bellotti and T.~Wong,
\emph{An improved explicit bound on the argument of the Riemann zeta function on the critical line},
arXiv:2412.15470v2 (2024).

\bibitem{PlattTrudgian2021}
D.~Platt and T.~Trudgian,
\emph{The Riemann hypothesis is true up to $3\cdot 10^{12}$},
Bulletin of the London Mathematical Society \textbf{53} (2021), no.~3, 792--797.
\end{thebibliography}


\end{document}
