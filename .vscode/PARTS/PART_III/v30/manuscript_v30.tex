\documentclass[11pt]{article}

\usepackage{amsmath,amssymb,amsthm}
\usepackage{geometry}
\usepackage{hyperref}
\usepackage{booktabs}
\usepackage{longtable}
\usepackage{listings}
\usepackage{xcolor}
\usepackage{graphicx}
\usepackage{enumitem}

\geometry{margin=1in}

% --- Listings (Python/JSON) ---
\lstdefinestyle{auditcode}{
  basicstyle=\ttfamily\footnotesize,
  columns=fullflexible,
  breaklines=true,
  keepspaces=true,
  showstringspaces=false,
  upquote=true
}
\lstset{style=auditcode}

% --- Theorem environments ---
\newtheorem{theorem}{Theorem}[section]
\newtheorem{lemma}[theorem]{Lemma}
\newtheorem{proposition}[theorem]{Proposition}
\newtheorem{corollary}[theorem]{Corollary}
\theoremstyle{definition}
\newtheorem{definition}[theorem]{Definition}
\theoremstyle{remark}
\newtheorem{remark}[theorem]{Remark}

% --- Notation helpers ---
\newcommand{\C}{\mathbb{C}}
\newcommand{\R}{\mathbb{R}}
\newcommand{\RePart}{\operatorname{Re}}
\newcommand{\ImPart}{\operatorname{Im}}
\newcommand{\abs}[1]{\left|#1\right|}
\newcommand{\norm}[1]{\left\|#1\right\|}

\title{A Width--2 Boundary Program for Excluding Off--Axis Quartets\\
with a Reproducible Tail Certificate (v30)}
\author{Dylan Anthony Dupont}
\date{v30 compiled: 2026-01-17}

\begin{document}
\maketitle

\begin{abstract}
This manuscript develops a width--2 ``boundary program'' designed to exclude off--axis quartets of
nontrivial zeros of the Riemann zeta function. The analytic core reduces the ``tail'' (heights above a
published verified band) to a single explicit inequality at one height, together with monotonicity and
worst--$\alpha$ reductions.

\medskip
\noindent\textbf{Claim hygiene (v30 posture).}
The \,\emph{interval certificate} included here proves that the one--height tail inequality holds\,
\emph{given} specific interval enclosures for a short list of constants
$(C_1,C_2,C_{\mathrm{up}},C_h'')$.
However, this certificate does \emph{not} by itself prove that these constants are valid bounds for the
analytic and geometric quantities asserted in the lemmas.
Accordingly, this document is submission--grade as a \emph{certified criterion}:
\emph{RH follows provided the constants ledger is independently certified} by explicit quantitative
analysis or by separate certified computation.
\end{abstract}

\tableofcontents

\section*{Executive Proof Status}
\addcontentsline{toc}{section}{Executive Proof Status}

\begin{enumerate}[leftmargin=1.2em]
\item \textbf{What is proved analytically (paper spine).}
  Part~\ref{part:core} develops a height--local mechanism: if an off--axis quartet exists at a width--2
  height $m$, then on an associated aligned box $B(\alpha,m,\delta)$ one must simultaneously have
  (i) a forcing lower bound of size $\pi/2$ on a short side from the local pair factor, and
  (ii) an upper bound on the remaining boundary variation in terms of a residual envelope
  $\sup_{\partial B}\abs{F'/F}$ plus shape--only constants.
  This yields a concrete tail inequality (Theorem~\ref{thm:tail-inequality}).
\item \textbf{What is discharged by the tail certificate.}
  Appendix~\ref{app:certificate} provides a deterministic interval-arithmetic generator/verifier
  that checks the tail inequality at one height $m=m_{\mathrm{band}}$ and worst parameter $\alpha=1$.
  In this chat, the verifier was executed and returned \texttt{PASS}; the exact stdout is recorded
  verbatim in Appendix~\ref{app:verifier-output}.
\item \textbf{What remains conditional (hard gate).}
  The constants ledger $(C_1,C_2,C_{\mathrm{up}},C_h'')$ is currently supplied as an interval JSON file
  (Appendix~\ref{app:bundle-files}).
  The tail verifier only proves: \emph{if those enclosures are correct}, then the one--height inequality
  holds with strict separation.
  To claim an \emph{unconditional} (computer-assisted) proof of RH, one must additionally provide an
  independent certification of each ledger constant (Section~\ref{sec:ledger}).
\item \textbf{Separation of concerns.}
  Supplementary numerics (tick-generator audits) are not used anywhere in the analytic proof spine.
  They may be included only as illustrative validation and must not be construed as proof input.
\end{enumerate}

\paragraph{One-shot audit path (tail).}
From the accompanying folder \texttt{v30_repro_pack/}, run:
\begin{enumerate}[leftmargin=1.5em]
\item \texttt{sha256sum -c SHA256SUMS.txt}
\item \texttt{python3 v30\_verify\_tail\_certificate.py --constants v29\_constants.json --certificate v29\_tail\_certificate.json}
\end{enumerate}
The verifier prints \texttt{CHECK: lhs.hi < rhs.lo : True} on success.

\part{Reader's Guide / Definitions and Reduction}
\label{part:guide}

\section{Width--2 normalization}
\label{sec:width2}

Define the width--2 objects
\begin{equation}
  u := 2s,\qquad \zeta_2(u) := \zeta\!\left(\tfrac{u}{2}\right),\qquad
  \Lambda_2(u) := \pi^{-u/4}\,\Gamma\!\left(\tfrac{u}{4}\right)\,\zeta\!\left(\tfrac{u}{2}\right).
\end{equation}
Then $\Lambda_2$ is entire and satisfies the functional equation
\begin{equation}
  \Lambda_2(u) = \Lambda_2(2-u).
\end{equation}
We recenter at $u=1$:
\begin{equation}
  v := u-1,\qquad E(v) := \Lambda_2(1+v).
\end{equation}
The functional equation becomes the evenness relation
\begin{equation}
  E(v)=E(-v),
\end{equation}
and complex conjugation gives $E(\overline{v})=\overline{E(v)}$.

\section{Heights and horizontal displacement (RH--free)}
\label{sec:heights}

Let $\rho=\beta+i\gamma$ be any nontrivial zero of $\zeta(s)$ (no assumption on $\beta$). In width--2
we write
\begin{equation}
  u_\rho := 2\rho = (1+a)+im,\qquad a:=2\beta-1\in(-1,1),\qquad m:=2\gamma>0.
\end{equation}
Thus RH is equivalent to $a=0$ for every nontrivial zero.

\section{Quartet symmetry in width--2}
\label{sec:quartets}

The functional equation and conjugation imply that any off--axis zero with parameters $(a,m)$
produces a quartet
\begin{equation}
  \{\,1\pm a\pm im\,\} \subset \{u\in\C: \Lambda_2(u)=0\}.
\end{equation}
In the centered $v$--coordinate this becomes $\{\pm a\pm im\}\subset\{v\in\C:E(v)=0\}$.

\section{Verified band (external input)}
\label{sec:verified-band}

This manuscript treats as an external input a published rigorous computation establishing RH up to
some height $H_0$ in the classical $s$-plane. In width--2 this corresponds to the band
\begin{equation}
  m_{\mathrm{band}} := 2H_0.
\end{equation}
In the v29--v30 chain, $H_0$ is taken to be $3\cdot 10^{12}$ (hence $m_{\mathrm{band}}=6\cdot 10^{12}$),
consistent with the stated Platt--Trudgian verification.
We isolate this as an explicit assumption so that the analytic tail argument is logically separate.
See Appendix~\ref{app:band}.

\part{Self-Contained Boundary Program and Tail Closure}
\label{part:core}

\section{Aligned boxes and the $\delta(m)$ scale}
\label{sec:boxes}

Let $m>0$ and $\alpha\in(0,1]$. Fix a parameter $\eta\in(0,1)$ and set
\begin{equation}
  \delta=\delta(m,\alpha):=\frac{\eta\alpha}{(\log m)^2}.
\end{equation}
Define the (width--2) box centered at $\alpha+im$ by
\begin{equation}
  B(\alpha,m,\delta) := \{\,v\in\C: \abs{\RePart v-\alpha}\le \delta,\ \abs{\ImPart v-m}\le \delta\,\}.
\end{equation}
We will also use the symmetric dial centers $v_\pm:=\pm\alpha+im$.

\begin{lemma}[Geometric containment]
\label{lem:delta-less-alpha}
If $\delta<\alpha$, then $B(\alpha,m,\delta)\subset\{\RePart v>0\}$.
\end{lemma}

\begin{proof}
If $\abs{\RePart v-\alpha}\le\delta<\alpha$, then $\RePart v\ge \alpha-\delta>0$.
\end{proof}

\section{Local factor and finiteness}
\label{sec:local-factor}

For a fixed $m>0$, let
\begin{equation}
  Z(m):=\{\,\rho: E(\rho)=0,\ \abs{\ImPart \rho-m}\le 1\,\}
\end{equation}
(zeros counted with multiplicity). Define the local zero factor and residual:
\begin{equation}
  Z_{\mathrm{loc}}(v):=\prod_{\rho\in Z(m)} (v-\rho)^{m_\rho},
  \qquad
  F(v):=\frac{E(v)}{Z_{\mathrm{loc}}(v)}.
\end{equation}

\begin{lemma}[Finiteness of $Z_{\mathrm{loc}}$]
\label{lem:zloc-finite}
For each fixed $m>0$ the set $Z(m)$ is finite; hence $Z_{\mathrm{loc}}$ is a finite product and $F$ is
meromorphic globally and analytic in any neighborhood of $\partial B(\alpha,m,\delta)$ that contains
no zeros of $E$.
\end{lemma}

\begin{proof}
Nontrivial zeros of $\zeta$ satisfy $0<\beta<1$, hence in the $v$--coordinate one has $\RePart v\in(-1,1)$
for all nontrivial zeros. Therefore the set
\(\{\abs{\ImPart v-m}\le 1\}\cap\{\abs{\RePart v}\le 1\}\)
is compact. Since $E$ is entire and its zeros are discrete, only finitely many zeros can lie in this
compact set.
\end{proof}

\section{Residual envelope bound and the constants ledger}
\label{sec:ledger}

\begin{lemma}[Residual envelope inequality]
\label{lem:residual-envelope}
There exist absolute constants $C_1,C_2>0$ such that for all $m\ge 10$, all $\alpha\in(0,1]$, and
$\delta=\eta\alpha/(\log m)^2$, one has
\begin{equation}
  \sup_{v\in\partial B(\alpha,m,\delta)} \abs{\frac{F'(v)}{F(v)}} \le C_1\log m + C_2.
\end{equation}
\end{lemma}

\begin{remark}[What v30 assumes vs what must be certified]
The tail certificate in Appendix~\ref{app:certificate} uses explicit numerical interval enclosures
for $C_1$ and $C_2$ (stored in \texttt{v29\_constants.json}).
The certificate verifies the tail inequality \emph{conditional on} these enclosures being correct.
An unconditional claim requires a separate certification of $C_1,C_2$ (Section~\ref{sec:ledger}).
\end{remark}

\section{Short-side forcing}
\label{sec:forcing}

Assume an off-axis pair at height $m$ with displacement $a>0$ exists. On an aligned box with
$\alpha=a$, the two upper zeros in the centered $v$--plane are at $v=\pm a+im$. The pair factor
\begin{equation}
  Z_{\mathrm{pair}}(v):=(v-(a+im))(v-(-a+im))
\end{equation}
produces a large phase rotation on the near vertical side.

\begin{lemma}[Short-side forcing lower bound]
\label{lem:short-side-forcing}
Let $I_+:=\{\alpha+iy: \abs{y-m}\le\delta\}$ with $\abs{\alpha-a}\le\delta$. Then
\begin{equation}
  \Delta_{I_+}\arg Z_{\mathrm{pair}}
  = 2\arctan\!\left(\frac{\delta}{\abs{\alpha-a}}\right)
    +2\arctan\!\left(\frac{\delta}{\alpha+a}\right)
  \ge \frac{\pi}{2}.
\end{equation}
\end{lemma}

\section{Outer factorization and the inner quotient (Bridge 1)}
\label{sec:bridge1}

Let $B=B(\alpha,m,\delta)$ and assume $E$ has no zeros on $\partial B$. Let $U$ be the harmonic solution
to the Dirichlet problem on $B$ with boundary data $\log\abs{E}$. Let $V$ be a harmonic conjugate on $B$
(chosen so that $U+iV$ is analytic). Define the outer function
\begin{equation}
  G_{\mathrm{out}}(v):=\exp(U(v)+iV(v)).
\end{equation}
Then $G_{\mathrm{out}}$ is analytic and zero-free on $B$ and satisfies $\abs{G_{\mathrm{out}}}=\abs{E}$
on $\partial B$. Define the inner quotient
\begin{equation}
  W(v):=\frac{E(v)}{G_{\mathrm{out}}(v)}.
\end{equation}
Then $W$ is analytic on $B$ and satisfies $\abs{W}=1$ on $\partial B$.

\begin{proposition}[Bridge 1: boundary modulus $1$ forces constancy if zero-free]
\label{prop:bridge1}
Assume $W$ is analytic and zero-free on $B$, continuous on $\overline{B}$, and satisfies $\abs{W}=1$ on
$\partial B$. Then $W$ is constant on $B$.
\end{proposition}

\begin{proof}
Since $W$ is continuous on $\overline{B}$ and analytic on $B$, the maximum modulus principle gives
$\abs{W}\le 1$ on $B$. Since $W$ is zero-free, $1/W$ is analytic on $B$ and continuous on $\overline{B}$,
with $\abs{1/W}=1$ on $\partial B$. Applying the maximum modulus principle to $1/W$ yields $\abs{1/W}\le 1$
on $B$, i.e. $\abs{W}\ge 1$ on $B$. Thus $\abs{W}\equiv 1$ on $B$, and an analytic function of constant modulus
is constant.
\end{proof}

\section{Shape-only invariance and the envelope constants}
\label{sec:shape-only}

Let $T(v):=(v-(\alpha+im))/\delta$, mapping $\partial B$ affinely onto the fixed square boundary
$\partial Q$ with $Q=[-1,1]^2$.

\begin{lemma}[Shape-only invariance]
\label{lem:shape-only}
Any constant arising solely from geometric or boundary-operator estimates on $\partial B$ that are
invariant under affine rescaling depends only on $\partial Q$ and is independent of $(\alpha,m,\delta)$.
\end{lemma}

\begin{proof}
Under $T$, arclength scales by $\delta$ and tangential derivatives by $1/\delta$. After normalization,
all purely geometric quantities and operator norms reduce to fixed quantities on $\partial Q$.
\end{proof}

\begin{lemma}[Upper envelope bound (residual form)]
\label{lem:upper-envelope}
There exists a shape-only constant $C_{\mathrm{up}}>0$ such that on aligned boxes $\alpha=\pm a$ one has
\begin{equation}
  \sum_{\pm}\abs{W(v_\pm)-e^{i\varphi_0^{\pm}}}
  \le 2C_{\mathrm{up}}\,\delta^{3/2}\,\sup_{v\in\partial B}\abs{\frac{F'(v)}{F(v)}},
\end{equation}
where $e^{i\varphi_0^{\pm}}$ are fixed boundary phase anchors for the two dial centers.
\end{lemma}

\begin{lemma}[Horizontal budget]
\label{lem:horizontal-budget}
There exists a shape-only constant $C_h''>0$ such that, after removing the residual factor $F$, the
remaining non-forcing boundary phase contribution satisfies
\begin{equation}
  \Delta_{\mathrm{nonforce}}\le C_h''\,\delta\,(\log m + 1)
\end{equation}
on aligned boxes.
\end{lemma}

\begin{remark}[Ledger status]
The tail certificate uses explicit interval enclosures for $C_{\mathrm{up}}$ and $C_h''$ (stored in
\texttt{v29\_constants.json} under keys \texttt{C\_up} and \texttt{C\_hpp}).
As with $(C_1,C_2)$, an unconditional claim requires independent certification of these geometric
constants.
\end{remark}

\section{The explicit tail inequality}
\label{sec:tail}

Define
\begin{equation}
  L(m):=C_1\log m + C_2.
\end{equation}
Fix the numerical constants
\begin{equation}
  c_0:=\frac{3\log 2}{8\pi},\qquad c:=\frac{3\log 2}{16},\qquad K_{\mathrm{alloc}}:=3+8\sqrt{3}.
\end{equation}
(These are exact quantities; no certification is required.)

\begin{remark}[Why $c_0$ and $c$ appear]
Lemma~\ref{lem:short-side-forcing} yields a raw phase rotation $\pi/2$ on a short side in the aligned
case. In the boundary program this forcing is compared against envelope costs in a normalized metric.
The scalar $c_0=(4\pi)^{-1}\log(2\sqrt{2})$ converts the forcing size $\pi/2$ into the constant
$c=c_0\cdot(\pi/2)=\tfrac{1}{8}\log(2\sqrt{2})=\tfrac{3\log 2}{16}$.
The factor $K_{\mathrm{alloc}}$ is the explicit coefficient produced by a fixed allocation scheme on the
normalized square boundary (the v25--v27 chain used the parameter choice $\lambda=12$).
\end{remark}

\begin{theorem}[Tail inequality (certified form)]
\label{thm:tail-inequality}
Fix $\eta\in(0,1)$ and set $\delta=\eta\alpha/(\log m)^2$.
Let $C_{\mathrm{up}},C_h''>0$ be the shape-only constants from Lemma~\ref{lem:upper-envelope} and
Lemma~\ref{lem:horizontal-budget}, and let $C_1,C_2>0$ be residual constants from
Lemma~\ref{lem:residual-envelope}.
If the inequality
\begin{equation}
\label{eq:tail-ineq}
  2C_{\mathrm{up}}\,\delta^{3/2}\,L(m)
  <
  c-\delta\,\Bigl(K_{\mathrm{alloc}}\,c_0\,L(m) + C_h''\,(\log m + 1)\Bigr)
\end{equation}
holds for a given $m\ge 10$ and all $\alpha\in(0,1]$, then there is no off--axis quartet at width--2 height
$m$.
\end{theorem}

\begin{lemma}[Worst--$\alpha$ reduction]
\label{lem:worst-alpha}
For fixed $m$ and fixed admissible constants, the left-hand side of \eqref{eq:tail-ineq} is increasing in
$\alpha\in(0,1]$ and the right-hand side is decreasing in $\alpha\in(0,1]$. Therefore it suffices to
verify \eqref{eq:tail-ineq} at $\alpha=1$.
\end{lemma}

\begin{proof}
With $\delta(\alpha)=\eta\alpha/(\log m)^2$, the left-hand side is proportional to
$\delta(\alpha)^{3/2}\propto\alpha^{3/2}$.
The right-hand side equals $c-\delta(\alpha)\cdot(\cdots)$, hence is affine decreasing in $\alpha$.
\end{proof}

\begin{lemma}[Monotonicity for one-height verification]
\label{lem:monotone-m}
Fix $\eta\in(0,1)$ and any admissible constants.
For all $m>1$ and fixed $\alpha\in(0,1]$, the left-hand side of \eqref{eq:tail-ineq} is non-increasing in
$m$, and the right-hand side is non-decreasing in $m$.
Consequently, verifying \eqref{eq:tail-ineq} at one height $m=m_\star$ implies it for all $m\ge m_\star$.
\end{lemma}

\begin{proof}
Write $x=\log m>0$.
Since $\delta(m)=\eta\alpha/x^2$, we have
\(\delta(m)^{3/2}L(m)=\eta^{3/2}\alpha^{3/2}\bigl(C_1x^{-2}+C_2x^{-3}\bigr)\), which is strictly decreasing
in $x$ because its derivative is
\(\eta^{3/2}\alpha^{3/2}(-2C_1x^{-3}-3C_2x^{-4})<0\).
Thus the left-hand side decreases in $m$.

For the right-hand side, the subtractive term equals
\(\eta\alpha\bigl(Ax^{-1}+Bx^{-2}\bigr)\) with
$A:=K_{\mathrm{alloc}}c_0C_1+C_h''>0$ and $B:=K_{\mathrm{alloc}}c_0C_2+C_h''>0$.
Its derivative in $x$ is negative:
\(-\eta\alpha(Ax^{-2}+2Bx^{-3})<0\), hence the subtractive term decreases in $m$ and the right-hand side
increases.
\end{proof}

\begin{theorem}[Tail closure from one certified check]
\label{thm:tail-closure}
Fix $\eta\in(0,1)$ and suppose \eqref{eq:tail-ineq} holds at one height $m=m_\star$ for $\alpha=1$.
Then \eqref{eq:tail-ineq} holds for all $m\ge m_\star$ and all $\alpha\in(0,1]$.
In particular, there are no off--axis quartets at any width--2 height $m\ge m_\star$.
\end{theorem}

\begin{proof}
Combine Lemma~\ref{lem:worst-alpha} and Lemma~\ref{lem:monotone-m}.
\end{proof}

\section{Global RH conditional on the ledger and the verified band}
\label{sec:global}

\begin{theorem}[Global RH from a verified band + a certified ledger]
\label{thm:global}
Assume:
\begin{enumerate}[leftmargin=1.5em]
\item[(A)] \textbf{Band.} RH holds for all nontrivial zeros with $0<\ImPart s\le H_0$.
\item[(B)] \textbf{Ledger constants.} The constants ledger $(C_1,C_2,C_{\mathrm{up}},C_h'')$ is independently
  certified to satisfy the interval enclosures in \texttt{v29\_constants.json}.
\item[(C)] \textbf{One-height check.} The tail inequality \eqref{eq:tail-ineq} is verified at
  $m=m_{\mathrm{band}}:=2H_0$ and $\alpha=1$.
\end{enumerate}
Then RH holds for all nontrivial zeros of $\zeta(s)$.
\end{theorem}

\begin{proof}
By (C) and Theorem~\ref{thm:tail-closure}, there are no off-axis quartets for all width--2 heights
$m\ge m_{\mathrm{band}}$, i.e. no off-axis zeros for $\ImPart s\ge H_0$.
By (A), there are no off-axis zeros for $\ImPart s\le H_0$.
Therefore there are no off-axis zeros at any height.
\end{proof}

\begin{remark}[Why v30 does not claim ``unconditional proof'']
Appendix~\ref{app:certificate} demonstrates (C) as a reproducible interval inequality check.
However, (B) is not yet discharged inside this version set: no independent derivation or certified
computation is provided for the ledger constants.
The most conservative correct posture is therefore: Theorem~\ref{thm:global} is a certified criterion.
\end{remark}

\part{Supplementary Numerics (Not Used in the Proof)}
\label{part:supp}

\section{Tick-generator audit (illustrative only)}
\label{sec:tick}

Earlier drafts (v25--v27) included a table/figure titled
``Numerical audit to $j=50$: error--vs--cutoff'' and ``Mean absolute tick error decreases as $C$ grows''.
In v28--v29 these plots were removed from the manuscript body and replaced by an auditable script.

In v30, the tick audit remains strictly supplementary and non-load-bearing.
Appendix~\ref{app:tick} provides:
\begin{itemize}
\item the original v29 audit script (unchanged),
\item an optimized v30 audit script that caches prime powers and supports frozen truth datasets,
\item a frozen ``truth'' dataset produced by \texttt{mpmath.zetazero} (explicitly \emph{not} an external
  verification), pinned by SHA--256.
\end{itemize}

\appendix

\section{Verified band input}
\label{app:band}

This manuscript separates the analytic tail argument from the finite-height verification.
Assumption~(A) of Theorem~\ref{thm:global} should be instantiated by citing a published, rigorous
verification of RH up to height $H_0$.
In the v29--v30 chain this is stated as $H_0=3\cdot 10^{12}$.

\section{Tail certificate bundle and reproducibility}
\label{app:certificate}

\subsection{What the tail certificate proves (and what it does not)}
\label{app:what-proves}

The certificate proves the following statement:

\begin{quote}
Given a constants file that provides interval enclosures for $(C_1,C_2,C_{\mathrm{up}},C_h'')$ and the
chosen parameters $(m_{\mathrm{band}},\eta,\alpha)$, the generated interval bounds satisfy
$\mathrm{LHS}<\mathrm{RHS}$ with strict separation in the sense
$\mathrm{LHS}_{\mathrm{hi}}<\mathrm{RHS}_{\mathrm{lo}}$.
\end{quote}

It does \emph{not} certify that the constants file is correct.
Discharging that requires separate work (analytic derivation and/or certified computation) beyond this
certificate.

\subsection{SHA--256 table (exact artifacts)}
\label{app:sha}

The file \texttt{v30\_repro\_pack/SHA256SUMS.txt} contains the canonical hash list.
For convenience we reproduce its contents here:

\lstinputlisting{v30_repro_pack/SHA256SUMS.txt}

\subsection{Commands}
\label{app:commands}

From the directory \texttt{v30\_repro\_pack/}:
\begin{enumerate}[leftmargin=1.5em]
\item \texttt{sha256sum -c SHA256SUMS.txt}
\item \texttt{python3 v30\_verify\_tail\_certificate.py --constants v29\_constants.json --certificate v29\_tail\_certificate.json}
\end{enumerate}

\subsection{Expected verifier output (verbatim)}
\label{app:verifier-output}

The verifier output recorded in this chat (\texttt{v30\_verifier\_output.txt}) is:

\lstinputlisting{v30_repro_pack/v30_verifier_output.txt}

\subsection{Bundle files (verbatim)}
\label{app:bundle-files}

\paragraph{Constants file (pinned).}
\lstinputlisting{v30_repro_pack/v29_constants.json}

\paragraph{Pinned tail certificate.}
\lstinputlisting{v30_repro_pack/v29_tail_certificate.json}

\paragraph{v29 generator implementation (directed rounding).}
\lstinputlisting{v30_repro_pack/v29_generate_tail_certificate.py}

\paragraph{v29 verifier (historical; shells out to non-prefixed generator).}
\lstinputlisting{v30_repro_pack/v29_verify_tail_certificate.py}

\paragraph{v30 canonical entrypoints.}
\lstinputlisting{v30_repro_pack/v30_generate_tail_certificate.py}
\lstinputlisting{v30_repro_pack/v30_verify_tail_certificate.py}

\section{Tick audit bundle (supplementary)}
\label{app:tick}

\paragraph{Frozen floating-point ``truth'' zeros (illustrative).}
\texttt{v30\_truth\_zeros\_mpmath\_dps80\_J50.json} is a frozen dataset produced by
\texttt{mpmath.zetazero} at \texttt{mp.dps=80}. It is pinned by SHA--256 in the table above.

\paragraph{v29 tick audit (unchanged).}
\texttt{v30\_repro\_pack/v29\_tick\_generator\_audit.py}

\paragraph{v30 tick audit (optimized, still illustrative).}
\texttt{v30\_repro\_pack/v30\_tick\_generator\_audit.py}

\end{document}
