```latex
% ======================================================================
% Master Manuscript — Part I (Reader's Guide) + Part II (Analytic Core) + Part III (Structural Corollaries)
% v26 = v25 + [PATCH 9.0] (restore v24 load-bearing sections; fix height definition; tighten box geometry;
%                          reinstate Schur route + bridge-logs; full Appendix NA script (no placeholders);
%                          expand appendices + references; add executive summary + RH-dependency ledger)
% ======================================================================

\documentclass[11pt]{article}

% ------------------ Basic packages ------------------
\usepackage[a4paper,margin=1in]{geometry}
\usepackage{amsmath,amssymb,amsthm,mathtools}
\usepackage{microtype}
\usepackage{hyperref}
\usepackage{nameref}
\usepackage{tabularx,booktabs,array}
\usepackage{enumitem}
\usepackage{needspace}
\usepackage{caption}
\usepackage{float}
\usepackage{longtable}
\usepackage{pgfplots}
\pgfplotsset{compat=1.18}

% ------------------ Theorem styles ------------------
\numberwithin{equation}{section}
\newtheorem{theorem}{Theorem}[section]
\newtheorem{lemma}[theorem]{Lemma}
\newtheorem{proposition}[theorem]{Proposition}
\newtheorem{corollary}[theorem]{Corollary}
\theoremstyle{remark}
\newtheorem{remark}[theorem]{Remark}

% ------------------ Column types ------------------
\newcolumntype{L}{>{\raggedright\arraybackslash}X}

% ------------------ Macros ------------------
\newcommand{\C}{\mathbb{C}}
\newcommand{\R}{\mathbb{R}}
\newcommand{\Z}{\mathbb{Z}}
\newcommand{\D}{\mathbb{D}}
\newcommand{\Real}{\operatorname{Re}}
\newcommand{\Imag}{\operatorname{Im}}
\newcommand{\Arg}{\operatorname{Arg}}
\newcommand{\sgn}{\operatorname{sgn}}
\newcommand{\ii}{\mathrm{i}}

\newcommand{\zetaTwo}{\zeta_2}
\newcommand{\LamTwo}{\Lambda_2}
\newcommand{\Afac}{A_2}
\newcommand{\chiTwo}{\chi_2}

\newcommand{\Podd}{P_{\mathrm{odd}}}
\newcommand{\Peven}{P_{\mathrm{even}}}
\newcommand{\Ucore}{U}
\newcommand{\UR}{U_{\mathrm{R}}}
\newcommand{\UL}{U_{\mathrm{L}}}

\newcommand{\Zloc}{Z_{\mathrm{loc}}}
\newcommand{\Gout}{G_{\mathrm{out}}}

\newcommand{\abs}[1]{\left|#1\right|}
\newcommand{\norm}[1]{\left\|#1\right\|}

\newenvironment{Overview}{\begin{quote}\itshape}{\end{quote}}

% ------------------ Title page ------------------
\title{\Large A Height--Local Width--2 Program for Excluding Off--Axis Quartets\\[2pt]
\large with an Analytic Tail and a Rigorous Certified Criterion}
\author{Dylan Anthony Dupont}
\date{\today}

\begin{document}
\maketitle

\begin{abstract}
\noindent
The manuscript is organized in three parts.\;
\textbf{Part~I} (Reader’s Guide) introduces the width--2 normalization and the centered \(a\)-parameter measuring horizontal displacement at a given height.\;
\textbf{Part~II} (Analytic Core) develops boundary-only tools on aligned boxes: (i) a hinge--unitarity law for the width--2 functional-equation factor; (ii) a de-singularized residual envelope for logarithmic derivatives; (iii) boundary criteria (including a certified Outer/Rouch\'e route) designed for validated numerics.\;
\textbf{Part~III} records structural corollaries under on-axis collapse and presents a deterministic prime--locked \emph{tick generator} together with a reproducible audit (supplementary).
\end{abstract}

\begin{center}
\fbox{\begin{minipage}{0.94\linewidth}
\textbf{Executive summary (what is proved vs.\ what is certified).}
\begin{itemize}[leftmargin=1.2em]
\item \textbf{Unconditional analytic statements proved in Part II:}
the width--2 setup and symmetries; Hinge--Unitarity (Thm.~\ref{thm:hinge});
residual/log-derivative envelopes (Lem.~\ref{lem:residual}, \ref{lem:bridge-logs});
and the rigorous \emph{Outer/Rouch\'e certification criterion} (Prop.~\ref{prop:rouche-criterion})
together with the inner-collapse bridges (Prop.~\ref{prop:bridge1}, \ref{prop:bridge2}).
\item \textbf{Certified criterion (computational, but rigorous):}
if the Rouch\'e ratio \eqref{eq:rouche-ratio} is validated on $\partial B$,
then the box is zero-free (Prop.~\ref{prop:rouche-criterion}); the protocol is frozen in Appendix~\ref{app:cert}.
\item \textbf{Analytic tail (programmatic):}
Part II provides explicit symbolic inequalities that, once all shape-only constants are pinned
(either from explicit operator bounds or from a certified computation on the normalized square),
yield an explicit threshold $M_0(\eta)$ beyond which off-axis quartets are excluded.
\item \textbf{Part III: supplementary.}
All lattice corollaries follow from on-axis collapse; the tick generator is a deterministic equation
audited against LMFDB data (Appendix~\ref{app:audit-protocol}) but is not used in Part II.
\end{itemize}
\end{minipage}}
\end{center}

\tableofcontents

% ======================================================================
% Part I — Reader’s Guide / Motivation, Reduction & Implications
% ======================================================================
\section*{Part I --- Reader’s Guide / Motivation, Reduction \& Implications}
\phantomsection
\addcontentsline{toc}{section}{Part I --- Reader’s Guide / Motivation, Reduction \& Implications}

\paragraph{What this section is (and is not).}
\emph{What it does.} It introduces the width--2 normalization, defines the centered coordinate \(v\) and the horizontal displacement parameter \(a\) at a given height, and records structural (non-load-bearing) corollaries that become valid once on-axis collapse is established.

\noindent\emph{What it does not do.} It contains no analytic estimates and no proofs. All analytic statements are proved in Part~II.

\subsection*{1) Modulated frames and the width--2 pivot}
For \(f>0\) define \(\zeta_f(s):=\zeta(s/f)\) with completed form
\[
\Lambda_f(s)=\pi^{-\,s/(2f)}\,\Gamma\!\Big(\frac{s}{2f}\Big)\,\zeta_f(s),
\]
so \(\Lambda_f\) is entire and satisfies \(\Lambda_f(s)=\Lambda_f(f-s)\).

\smallskip
\noindent\textbf{Width--2 normalization.} Put \(u:=(2/f)\,s\). Then
\[
\zetaTwo(u):=\zeta(u/2),\qquad
\LamTwo(u):=\pi^{-u/4}\Gamma(u/4)\,\zeta(u/2),\qquad
\LamTwo(u)=\LamTwo(2-u).
\]
The non--completed FE reads \(\zetaTwo(u)=\Afac(u)\,\zetaTwo(2-u)\) where \(\Afac(u)\Afac(2-u)\equiv 1\).

\smallskip
\noindent\textbf{Centered coordinate.} Put \(v:=u-1\) and define
\[
E(v):=\LamTwo(1+v).
\]
Then \(E(v)=E(-v)=\overline{E(\bar v)}\).

\subsection*{2) Heights and the width parameter \(a\) (RH-free definition)}
Let \(s=\beta+\ii t\) be any nontrivial zero of \(\zeta(s)\) with \(0<\beta<1\), \(t>0\). In the width--2 frame \(u=2s\) and centered variable \(v=u-1\),
\[
v = (2\beta-1) + \ii (2t).
\]
Define the \emph{height} and \emph{horizontal displacement} of this zero by
\[
m:=\Imag v = 2t,\qquad a:=\Real v = 2\beta-1.
\]
Thus RH is equivalent to: every nontrivial zero has \(a=0\).

\smallskip
\noindent\textbf{Quartet symmetry.} If \(E(a+\ii m)=0\) with \(a\neq 0\), then by \(E(v)=E(-v)=\overline{E(\bar v)}\) the quartet
\(\{\pm a \pm \ii m\}\) are zeros in the \(v\)-plane.

\subsection*{3) Height--local reduction}
Define the height-local statement:
\[
\mathrm{PHU}(m): \quad \text{every zero of \(E\) with imaginary part \(m\) has real part \(0\).}
\]
Then RH is equivalent to \(\mathrm{PHU}(m)\) holding for every height \(m\) attained by a nontrivial zero.

\subsection*{4) Boxes and hand-off}
For \(x_0\in(0,1]\) and \(m\ge 10\), define the aligned box
\[
B(x_0,m,\delta):=[x_0-\delta,x_0+\delta]\times[m-\delta,m+\delta],
\qquad
\delta=\frac{\eta\,x_0}{(\log m)^2},\ \ \eta\in(0,1).
\]
Part~II develops boundary criteria on \(\partial B\) designed to exclude off-axis quartets by contradiction and/or certification.

\subsection*{5) Toolbox \(\to\) structural consequences (after collapse)}
The parity-lattice constructions and structural corollaries in Part~III become valid once collapse is established. They are not used as inputs in Part~II.

% ======================================================================
% Part II — Analytic Core and Certified Criterion
% ======================================================================
\section*{Part II --- Analytic Core and Certified Criterion on Aligned Boxes}
\phantomsection
\addcontentsline{toc}{section}{Part II --- Analytic Core and Certified Criterion on Aligned Boxes}

\paragraph{Standing objects.}
In the width--2 centered frame:
\[
\LamTwo(u)=\pi^{-u/4}\Gamma(u/4)\zeta(u/2),\qquad E(v)=\LamTwo(1+v).
\]
We work on boxes \(B(x_0,m,\delta)\) with \(m\ge 10\) and \(\delta=\eta x_0/(\log m)^2\).

\Needspace{16\baselineskip}
\section*{Symbols \& provenance (at a glance)}
\phantomsection
\addcontentsline{toc}{section}{Symbols \& provenance (at a glance)}

\small
\begin{center}
\begin{tabularx}{\textwidth}{@{}p{3.9cm} L L@{}}
\toprule
\textbf{Symbol} & \textbf{Definition / role} & \textbf{Provenance / rationale}\\
\midrule
$u=2s$, $v=u-1$ & width--2, centered frame & centers FE symmetry\\
\midrule
$\LamTwo(u)$, $E(v)$ & completed width--2 object; shifted workhorse & standard FE + conjugation\\
\midrule
$\chiTwo(u)$ & inverse FE factor & $\chiTwo(u)=\pi^{u/2-1/2}\dfrac{\Gamma((2-u)/4)}{\Gamma(u/4)}$\\
\midrule
$B(x_0,m,\delta)$ & $[x_0-\delta,x_0+\delta]\times[m-\delta,m+\delta]$ & aligned box around $(x_0,m)$\\
\midrule
$\delta=\eta x_0/(\log m)^2$ & half-side length & balances forcing vs residual terms\\
\midrule
$I_L,I_R,H_\pm$ & left/right vertical sides; horizontals & boundary decomposition\\
\midrule
$\Zloc$ & product of local zeros near height band & de-singularizes logarithmic derivatives\\
\midrule
$F=E/\Zloc$ & residual analytic factor on $\partial B$ & controlled by Lem.~\ref{lem:residual}\\
\midrule
$G(v)=E(1+v)/E(1-v)$ & FE-symmetric quotient & used in Schur route\\
\midrule
$\Gout=e^{U+\ii V}$ & outer function with $|\Gout|=|E|$ on $\partial B$ & Dirichlet + harmonic conjugate\\
\midrule
$W=E/\Gout$ & inner quotient ($|W|=1$ a.e.\ on $\partial B$) & collapses to constant under Rouch\'e\\
\bottomrule
\end{tabularx}
\end{center}
\normalsize

\medskip
\noindent\textit{Primary sources.}
Digamma bounds and reflection: DLMF \S5.5, \S5.11.
Logarithmic derivative $\zeta'/\zeta$ expansions: Titchmarsh \cite[\S14]{Titchmarsh}, Ivi\'c \cite[Ch.~9]{Ivic}.
Lipschitz Cauchy/Hilbert boundedness: Coifman--McIntosh--Meyer \cite{CoifmanMcIntoshMeyer}.
Corner regularity and Dirichlet continuity for rectangles: Kellogg \cite{Kellogg}, Axler--Bourdon--Ramey \cite{AxlerBourdonRamey}.

% ---------------------------------------------------
\section{Frames, symmetry, and the hinge law}\label{sec:frames}
% ---------------------------------------------------

\begin{theorem}[Hinge--Unitarity]\label{thm:hinge}
Let \(\zetaTwo(u)=\zeta(u/2)\) and \(\zetaTwo(u)=A_2(u)\,\zetaTwo(2-u)\) with
\[
\chiTwo(u):=A_2(u)^{-1}=\pi^{u/2-1/2}\frac{\Gamma\big(\frac{2-u}{4}\big)}{\Gamma\big(\frac{u}{4}\big)}.
\]
For each fixed \(t\neq 0\), define \(f(\sigma)=\log|\chiTwo(\sigma+\ii t)|\). Then
\begin{equation}\label{eq:fprime}
f'(\sigma)=\tfrac12\log\pi-\tfrac12\,\Real\psi\!\Big(\tfrac{\sigma+\ii t}{4}\Big)
-\tfrac14\,\Real\!\Big[\pi\cot\!\Big(\tfrac{\pi}{4}(\sigma+\ii t)\Big)\Big].
\end{equation}
Moreover,
\begin{equation}\label{eq:cotbound}
\big|\Real[\pi\cot(x+\ii y)]\big|\le\frac{\pi}{\cosh(2y)-1}.
\end{equation}
For \(|t|\ge t_1\) (Appendix~\ref{app:firstheight-certified}), \(f'(\sigma)<0\) for all real \(\sigma\).
Since \(f(1)=0\), we have \(|\chiTwo(u)|=1\) iff \(\Real u=1\) for all \(|\Imag u|\ge t_1\).
\end{theorem}

\begin{proof}
Differentiate \(\log|\chiTwo|\) using \(\partial_\sigma \log|\Gamma(z)|=\Real\psi(z)\) and the reflection identity
\(\psi(1-z)-\psi(z)=\pi\cot(\pi z)\) (DLMF \S5.5). This yields \eqref{eq:fprime}.
For \eqref{eq:cotbound}, write
\(
\cot(x+\ii y)=\dfrac{\sin(2x)-\ii\sinh(2y)}{\cosh(2y)-\cos(2x)}
\)
so the real part is
\(
\Real\cot(x+\ii y)=\dfrac{\sin(2x)}{\cosh(2y)-\cos(2x)}
\)
and \(|\sin(2x)|\le 1\), \(\cos(2x)\le 1\) give \eqref{eq:cotbound}.
Finally, vertical-strip bounds for \(\psi\) (DLMF \S5.11) imply
\[
\Real\psi\!\Big(\tfrac{\sigma+\ii t}{4}\Big)\ge \log\!\Big(\tfrac{|t|}{4}\Big)-\frac{2}{|t|}
\qquad(|t|\ge 10,\ \sigma\in\R),
\]
so the dominating term in \eqref{eq:fprime} is \(-\tfrac12\log(|t|/4)\) while the cotangent term is exponentially small in \(|t|\). For \(|t|\ge t_1\) this yields \(f'(\sigma)<0\) for all \(\sigma\).
\end{proof}

% ---------------------------------------------------
\section{Boxes, de-singularization, residual control, and forcing}\label{sec:boxes}
% ---------------------------------------------------

Fix \(m\ge 10\), \(x_0\in(0,1]\), and
\begin{equation}\label{eq:box-delta}
B(x_0,m,\delta)=\big[x_0-\delta,x_0+\delta\big]\times\big[m-\delta,m+\delta\big],
\qquad
\delta=\frac{\eta\,x_0}{(\log m)^2},\ \ \eta\in(0,1).
\end{equation}
Write the boundary as:
\[
I_L=\{x_0-\delta+\ii y:\ |y-m|\le \delta\},\quad
I_R=\{x_0+\delta+\ii y:\ |y-m|\le \delta\},
\]
\[
H_-=\{x+\ii(m-\delta):\ |x-x_0|\le \delta\},\quad
H_+=\{x+\ii(m+\delta):\ |x-x_0|\le \delta\}.
\]

\begin{lemma}[Short boxes stay in \(\Real v>0\)]\label{lem:box-right}
For \(m\ge10\) and \(\eta\in(0,1)\), one has \(\delta<x_0\) and \(B(x_0,m,\delta)\subset\{\Real v>0\}\).
\end{lemma}
\begin{proof}
Since \(\eta/(\log m)^2<1\), \(\delta=x_0\eta/(\log m)^2<x_0\), hence \(x_0-\delta>0\).
\end{proof}

\paragraph{De--singularization.}
Let
\begin{equation}\label{eq:Zloc}
\Zloc(v):=\prod_{\rho:\,|\Imag\rho-m|\le 1}(v-\rho)^{m_\rho},
\qquad
F(v):=\frac{E(v)}{\Zloc(v)}.
\end{equation}
By construction, \(F\) is analytic and zero-free on a neighborhood of \(\partial B\) (all zeros/poles of \(E\) in the band \(|\Imag\rho-m|\le 1\) are removed).

\begin{lemma}[Residual envelope]\label{lem:residual}
There exist absolute constants \(C_1,C_2>0\) such that on \(\partial B\),
\begin{equation}\label{eq:residual-sup}
\sup_{\partial B}\Big|\frac{F'}{F}\Big|\ \le\ C_1\log m + C_2,
\end{equation}
and hence
\begin{equation}\label{eq:residual-perimeter}
\big|\Delta_{\partial B}\arg F\big|\ \le\ 8\delta\,\big(C_1\log m+C_2\big).
\end{equation}
\end{lemma}

\begin{proof}
On vertical strips, \(\psi(z)=\log z+O(1)\) uniformly (DLMF \S5.11), so the Gamma-factor contribution to \(E'/E\) is \(O(\log m)\) on \(\partial B\).
For the zeta factor, one uses the classical local expansion (Titchmarsh \cite[\S14]{Titchmarsh}, Ivi\'c \cite[Ch.~9]{Ivic})
\[
\frac{\zeta'}{\zeta}(\sigma+\ii t)=\sum_{|\Imag\rho-t|\le 1}\frac{1}{\sigma+\ii t-\rho}+O(\log t),
\qquad (1/2\le\sigma\le 1,\ t\ge 3),
\]
with the implied constant absolute. Transporting to width--2 and dividing by \(\Zloc\) cancels the local pole sum, leaving an \(O(\log m)\) bound on \(\partial B\).
Collecting the Gamma and zeta contributions yields \eqref{eq:residual-sup}. Then
\(\Delta_{\partial B}\arg F=\int_{\partial B}\partial_\tau\arg F\,ds\) with \(|\partial_\tau\arg F|\le |F'/F|\) and \(|\partial B|=8\delta\) gives \eqref{eq:residual-perimeter}.
\end{proof}

\begin{lemma}[Bridge logs on \(\partial B\)]\label{lem:bridge-logs}
On \(\partial B\),
\[
\frac{E'}{E}=\frac{F'}{F}+\frac{\Zloc'}{\Zloc}.
\]
In particular,
\[
\sup_{\partial B}\Big|\frac{E'}{E}\Big|
\le
\sup_{\partial B}\Big|\frac{F'}{F}\Big|
+\sum_{\rho:\,|\Imag\rho-m|\le 1}\ \sup_{v\in\partial B}\frac{m_\rho}{|v-\rho|}.
\]
\end{lemma}
\begin{proof}
Differentiate \(E=F\Zloc\). The supremum bound is immediate from the triangle inequality.
\end{proof}

\begin{lemma}[Short-side forcing from a conjugate-symmetric pair]\label{lem:short-side}
Let \(a\in(0,1]\) and define the pair factor
\[
Z_{\mathrm{pair}}(v)=(v-(a+\ii m))(v-(-a+\ii m)).
\]
On the right side \(I_R\) of the aligned box \(B(a,m,\delta)\), one has
\begin{equation}\label{eq:short-side}
\Delta_{I_R}\arg Z_{\mathrm{pair}}
=
2\arctan\!\Big(\frac{\delta}{\delta}\Big)+2\arctan\!\Big(\frac{\delta}{2a+\delta}\Big)
\ \ge\ \frac{\pi}{2}.
\end{equation}
\end{lemma}

\begin{proof}
Parametrize \(v(y)=a+\delta+\ii y\), \(y\in[m-\delta,m+\delta]\).
For \(v(y)-(a+\ii m)=\delta+\ii(y-m)\),
\[
\Delta_{I_R}\arg\big(v-(a+\ii m)\big)
=
\arctan\!\Big(\frac{\delta}{\delta}\Big)-\arctan\!\Big(\frac{-\delta}{\delta}\Big)=\frac{\pi}{2}.
\]
For \(v(y)-(-a+\ii m)=(2a+\delta)+\ii(y-m)\),
\[
\Delta_{I_R}\arg\big(v-(-a+\ii m)\big)
=
2\arctan\!\Big(\frac{\delta}{2a+\delta}\Big)\ge 0.
\]
Summing yields \eqref{eq:short-side}.
\end{proof}

% ---------------------------------------------------
\section{Boundary-only criteria: Schur route and certified Rouch\'e route}\label{sec:criteria}
% ---------------------------------------------------

\subsection{Two-point Schur/outer criterion (boundary-only)}\label{subsec:schur}

Define the quotient
\begin{equation}\label{eq:Gdef}
G(v):=\frac{E(1+v)}{E(1-v)}.
\end{equation}
This quotient is designed to compare symmetric points about the hinge. It is analytic on boxes avoiding the singular set of the denominator.

\begin{proposition}[Two-point Schur pinning]\label{prop:schur-pin}
Let \(\varphi:\D\to B\) be conformal with \(\varphi(0)\) the center of \(B\) and nontangential boundary correspondence away from corners.
Let \(M\in C(\partial B)\) satisfy \(M\ge |G|\) a.e.\ on \(\partial B\), and let \(H=e^{U+\ii V}\) be the outer majorant obtained by solving Dirichlet on \(B\) with boundary data \(\log M\) and choosing a harmonic conjugate \(V\).
Then \(\Phi:=(G/H)\circ\varphi\in H^\infty(\D)\) and \(\|\Phi\|_\infty\le 1\).
If two non-corner boundary points \(\zeta_\pm\in\partial\D\) satisfy \(|\Phi(\zeta_\pm)|=1\) and some boundary arc \(A\subset\partial\D\) satisfies \(\operatorname*{ess\,sup}_A|\Phi|\le 1-\varepsilon\), then for any \(z\in\D\) with harmonic measure \(\omega_z(A)\ge \omega_*>0\),
\[
|\Phi(z)|\le 1-\kappa,
\qquad
\kappa=\kappa(\varepsilon,\omega_*)>0.
\]
\end{proposition}

\begin{proof}
This is a standard harmonic-measure/maximum-modulus argument in \(H^\infty\): the subharmonic function \(\log|\Phi|\le 0\) has boundary values \(\le 0\) a.e., equals \(0\) at two points, and is \(\le \log(1-\varepsilon)\) on an arc. Poisson averaging at \(z\) yields a strict deficit \(\kappa\) once \(\omega_z(A)\) is bounded below.
See Duren \cite[\S II.5]{DurenHp} and Garnett \cite[\S II.2]{GarnettBAF}.
\end{proof}

\begin{lemma}[Two-point link between \(G\) and \(\chiTwo\)]\label{lem:G-chi-link}
For \(v=a+\ii m\),
\begin{equation}\label{eq:G-chi-product}
|G(a+\ii m)|\,|G(-a+\ii m)|
=\big|\chiTwo(1+a+\ii m)\big|\,\big|\chiTwo(1-a+\ii m)\big|.
\end{equation}
\end{lemma}

\begin{proof}
By definition \(E(v)=\LamTwo(1+v)\), so
\[
G(v)=\frac{\LamTwo(2+v)}{\LamTwo(2-v)}.
\]
Using \(\LamTwo(u)=\LamTwo(2-u)\) and the non-completed FE for \(\zetaTwo\) in width--2 yields the identity
\(
|G(v)|\,|G(-v)|=|\chiTwo(1+v)|\,|\chiTwo(1-v)|.
\)
Evaluating at \(v=a+\ii m\) gives \eqref{eq:G-chi-product}.
\end{proof}

\subsection{Outer/Rouch\'e certification route (rigorous and checkable)}\label{subsec:rouche}

Let \(U\) solve the Dirichlet problem on \(B\) with boundary data \(\log|E|\) (continuous if \(E\) has no zeros on \(\partial B\)), and let \(V\) be a harmonic conjugate. Define
\[
\Gout:=e^{U+\ii V}.
\]
Then \(\Gout\) is analytic and zero-free in \(B\), and \(|\Gout|=|E|\) on \(\partial B\) in the sense of boundary traces.

\begin{proposition}[Outer/Rouch\'e criterion]\label{prop:rouche-criterion}
If
\begin{equation}\label{eq:rouche-ratio}
\sup_{v\in\partial B}\frac{|E(v)-\Gout(v)|}{|\Gout(v)|}\ <\ 1,
\end{equation}
then \(E\) is zero-free in \(B\).
Consequently, the inner quotient \(W:=E/\Gout\) is analytic and nonvanishing in \(B\) with \(|W|=1\) a.e.\ on \(\partial B\).
\end{proposition}

\begin{proof}
By \eqref{eq:rouche-ratio}, \(|E-\Gout|<|\Gout|\) on \(\partial B\). Since \(\Gout\) is zero-free in \(B\), Rouch\'e's theorem implies that \(E\) and \(\Gout\) have the same number of zeros in \(B\), hence none. The stated properties of \(W\) follow.
\end{proof}

\begin{proposition}[Bridge~1: inner collapse]\label{prop:bridge1}
Under \eqref{eq:rouche-ratio}, \(W\equiv e^{\ii\theta_B}\) on \(B\) for some constant phase \(\theta_B\in\R\).
\end{proposition}
\begin{proof}
Since \(W\) is analytic and nonvanishing, \(\log|W|\) is harmonic. The boundary trace satisfies \(\log|W|=0\) a.e.\ because \(|W|=|E|/|\Gout|=1\) on \(\partial B\). By uniqueness for the Dirichlet problem, \(\log|W|\equiv 0\) on \(B\), hence \(|W|\equiv 1\). The open mapping theorem forces an analytic map of a domain into the unit circle to be constant, so \(W\equiv e^{\ii\theta_B}\).
\end{proof}

\begin{proposition}[Bridge~2: stitching]\label{prop:bridge2}
If \(B_1,B_2\) overlap and \(W\equiv e^{\ii\theta_{B_j}}\) on \(B_j\) for \(j=1,2\), then \(\theta_{B_1}\equiv\theta_{B_2}\pmod{2\pi}\).
\end{proposition}
\begin{proof}
On \(B_1\cap B_2\), both analytic constants agree pointwise, hence globally.
\end{proof}

\subsection{Corner interpolation (regularity reminder)}\label{subsec:corner}
Rectangles are Wiener-regular, so continuous boundary data admit harmonic extensions continuous up to \(\overline B\).
This justifies using continuous boundary moduli in the outer construction and evaluating corner limits; see Kellogg \cite{Kellogg} and Axler--Bourdon--Ramey \cite{AxlerBourdonRamey}.
A detailed note is recorded in Appendix~\ref{app:corner}.

% ---------------------------------------------------
\section{Analytic tail: symbolic inequalities and pinned constants (programmatic)}\label{sec:tail}
% ---------------------------------------------------

\paragraph{Purpose of this section.}
This section records the symbolic inequalities that connect:
(i) short-side forcing (Lemma~\ref{lem:short-side}),
(ii) residual envelopes (Lemma~\ref{lem:residual}),
and (iii) an upper--lower comparison on a covering family of boxes.
Once all shape-only constants are pinned (either from explicit operator-norm bounds or from a certified computation on the normalized square), these inequalities yield an explicit threshold \(M_0(\eta)\) for the tail.
The pinned-constants closure is recorded in Appendix~\ref{app:S3}.

\begin{remark}[Why constants are “shape-only”]\label{rem:shape-only}
Under the affine normalization \(T(v)=(v-(x_0+\ii m))/\delta\), the boundary \(\partial B(x_0,m,\delta)\) maps to the fixed square boundary \(\partial Q\), \(Q=[-1,1]^2\).
Any constant arising solely from boundary singular-integral operator norms (Cauchy/Hilbert), Poisson/trace operators, or geometric decompositions of the square boundary depends only on \(\partial Q\), hence is independent of \(m,x_0,\delta\).
\end{remark}

\begin{theorem}[Tail closure criterion: symbolic form]\label{thm:tail-criterion}
Fix \(\eta\in(0,1)\) and set \(\delta=\eta x_0/(\log m)^2\).
Assume:
\begin{itemize}[leftmargin=1.2em]
\item residual envelope constants \(C_1,C_2\) as in Lemma~\ref{lem:residual};
\item pinned shape-only constants \(C_{\mathrm{up}}, C_h''\) controlling the relevant boundary-to-interior and allocation estimates on the normalized square (Appendix~\ref{app:S1} and Appendix~\ref{app:S3}).
\end{itemize}
Then there exists an explicit computable function \(M_0(\eta)\) (depending only on these constants) such that:
for all \(m\ge M_0(\eta)\) and all \(x_0\in(0,1]\),
any off-axis quartet at height \(m\) would violate at least one of the pinned inequalities, hence is excluded.
\end{theorem}

\begin{remark}[How \(M_0(\eta)\) is used in practice]
In an implementation, one combines:
\begin{enumerate}[label=(\alph*)]
\item a finite verified band \(0<\Imag s\le T_{\mathrm{ver}}\) from Platt/Platt--Trudgian (Appendix~\ref{app:firstheight-certified});
\item a tail threshold \(M_0(\eta)\) for \(\Imag u=\Imag(2s)=2\Imag s\).
\end{enumerate}
If \(M_0(\eta)\le 2T_{\mathrm{ver}}\), then the finite band plus tail imply global RH.
Appendix~\ref{app:S3} records the pinned-constants logic needed to drive \(M_0(\eta)\) down (potentially to \(m_1\)).
\end{remark}

% ======================================================================
% Part III — Structural Corollaries (after collapse)
% ======================================================================
\section*{Part III --- Structural Corollaries (supplementary)}
\phantomsection
\addcontentsline{toc}{section}{Part III --- Structural Corollaries (supplementary)}

\paragraph{Standing basis.}
Part~III assumes the on-axis collapse conclusion (i.e.\ every nontrivial zero has \(a=0\)) and records structural consequences.
Nothing in Part~III is used in Part~II.

\begin{corollary}[Canonical columns]\label{cor:canonical-columns}
Define \(\Podd(n)=(1-\cos\pi n)/2\) and \(\Peven(n)=(1+\cos\pi n)/2\). Let \(k(2j-1)=j\), \(k(2j)=j+1\).
For any \(x\in(0,2)\),
\[
\UR(x,n)=\Podd(n)\,\big(x+\ii\,m_{k(n)}\big)\;-\;4\big(n+1-k(n)\big)\,\Peven(n),
\qquad
\UL(x,n)=\Podd(n)\,\big(2-x+\ii\,m_{k(n)}\big)\;-\;4\big(n+1-k(n)\big)\,\Peven(n).
\]
Under collapse, the canonical choice \(x=1\) gives \(\UR(1,n)=\UL(1,n)\) for all \(n\).
\end{corollary}

\begin{corollary}[Collapsed canonical stream: mod--4 face]\label{cor:collapsed-mod4}
\[
\Ucore(n):=\Podd(n)\,\big(1+\ii\,m_{k(n)}\big)\;-\;4\big(n+1-k(n)\big)\,\Peven(n),
\]
so \(\Ucore(2j-1)=1+\ii m_j\) and \(\Ucore(2j)=-4(j+1)\).
\end{corollary}

\begin{corollary}[Collapsed canonical stream: mod--2 face]\label{cor:collapsed-mod2}
Using \(\sin^2(\pi n/2)=\Podd(n)\) and \(\cos^2(\pi n/2)=\Peven(n)\),
\[
\Ucore(n)=\sin^2\!\Big(\frac{\pi n}{2}\Big)\,\big(1+\ii\,m_{k(n)}\big)\;-\;4\big(n+1-k(n)\big)\,\cos^2\!\Big(\frac{\pi n}{2}\Big).
\]
\end{corollary}

\begin{corollary}[Single--frequency collapse]\label{cor:single-frequency}
There exist functions \(c(n),d(n)\) with
\[
\Ucore(n)=(c+d)\;+\;(c-d)\,\cos(\pi n),\qquad
c=2\big(k(n)-n-1\big),\quad d=\frac{1+\ii\,m_{k(n)}}{2}.
\]
\end{corollary}

\begin{corollary}[Self--indexed recurrence]\label{cor:self-indexed}
With \(\Ucore(0)=-4\) and \(\Ucore(1)=1+\ii m_1\), for all \(n\ge2\),
\[
\Ucore(n)=\Podd(n)\,\Big(1+\ii\,m_{-\Ucore(n-1)/4}\Big)\;-\;\Peven(n)\,\Big(\Ucore(n-2)+4(n+1)\Big).
\]
\end{corollary}

\begin{corollary}[Seed $\to$ rectifier $\to$ physical streams]\label{cor:rectifier}
Let \(\chi_4(n):=(-1)^{\lfloor n/2\rfloor}\). For \(f>0\) and gain \(\lambda\in\R\),
\[
s_{f,k}(n)=f\lambda\Big[\sin\!\Big(\frac{\pi n}{2}\Big)\big(1+\ii\,m_k\big)-4n\,\cos\!\Big(\frac{\pi n}{2}\Big)\Big],
\]
then \(\chi_4(n)\,s_{f,k}(n)=f\lambda\big[\Podd(n)(1+\ii m_k)-4n\,\Peven(n)\big]\).
With \(\lambda=\tfrac12\) and \(k=k(n)\) we get \(S_f(n)=\frac{f}{2}\,\Ucore(n)\).
\end{corollary}

\begin{corollary}[Curvature extractor \& \(\zeta(2)\) disguise]\label{cor:curvature}
Let \(F(n):=\Imag \Ucore(n)\). Then \(F(2j-1)=m_j\), \(F(2j)=0\), and
\[
m_j=\frac{2}{\pi^2}\,\Imag\big(\Ucore''(2j)\big)
=\frac{1}{3\,\zeta(2)}\,\Imag\big(\Ucore''(2j)\big)
=\frac{2}{3\,\zeta(2)}\sum_{\ell\in\Z}\frac{m_\ell}{\big(2(j-\ell)+1\big)^2}.
\]
For \(\Delta^2 U(n):=U(n+1)-2U(n)+U(n-1)\), \(\Imag\Delta^2 U(2j)=m_{j+1}+m_j\).
\end{corollary}

\begin{center}
\fbox{\begin{minipage}{0.94\linewidth}
\textbf{RH–dependency ledger (Part III).}
\begin{itemize}
\item \emph{RH-free:} pure parity-lattice algebra and identities in Cor.~\ref{cor:canonical-columns}--\ref{cor:curvature}.
\item \emph{Uses collapse:} interpreting the odd-lane imaginary parts as the complete set of nontrivial ordinates.
\end{itemize}
\end{minipage}}
\end{center}

% ----------------------------------------------------------------------
% Part III (continued) — Prime-locked tick generator (audited)
% ----------------------------------------------------------------------
\section*{Part III (continued) --- Prime--Locked Tick Generator (audited; supplementary)}
\phantomsection
\addcontentsline{toc}{section}{Part III (continued) --- Prime--Locked Tick Generator (audited; supplementary)}

\paragraph{Notation (true zeros vs.\ generated ticks).}
Let \(\gamma_1<\gamma_2<\cdots\) denote ordinates of zeros on \(\Real s=\tfrac12\) and set \(m_j:=2\gamma_j\).
Define a deterministic \emph{tick sequence} \(\tilde t_1,\tilde t_2,\dots\) by the generator equation below and set \(\tilde m_j:=2\tilde t_j\).
The numerical audit compares \(\tilde m_j\) against the true \(m_j\).
Part~II does not use this section.

Let \(\theta(t)\) be the Riemann--Siegel theta function.

Fix
\begin{equation}\label{eq:PW-choices}
A:=\tfrac32,\qquad
X(t):=C(\log t)^A\qquad (C\ge 1),
\end{equation}
and the smooth cutoff weight \(W\) of Appendix~\ref{app:PW}.

Define for \(t>0\) and \(\Delta>0\):
\[
\mathcal P_{X(t)}(t,\Delta)
:=
-\sum_{p^k\ge1}\frac{1}{k\,p^{k/2}}\,
W\!\Big(\frac{p^k}{X(t)}\Big)
\Big[\sin\!\big((t+\Delta)k\log p\big)-\sin\!\big(tk\log p\big)\Big].
\]

\begin{theorem}[Deterministic prime--locked tick generator]\label{thm:generator}
Fix \(C\ge 1\), \(A=\tfrac32\), and \(W\) as above.
Seed \(\tilde t_1:=\gamma_1\) (Appendix~\ref{app:firstheight-certified}).
Given \(\tilde t_j\), define \(\tilde t_{j+1}\) as the unique solution of
\begin{equation}\label{eq:generator-eqn}
\theta(\tilde t_{j+1})-\theta(\tilde t_j)\;+\;\mathcal P_{X(\tilde t_j)}(\tilde t_j,\tilde t_{j+1}-\tilde t_j)\;=\;\pi.
\end{equation}
For all sufficiently large \(j\), \eqref{eq:generator-eqn} has a unique solution \(\tilde t_{j+1}>\tilde t_j\),
and a deterministic bracketed bisection converges.
\end{theorem}

\begin{proof}
Define \(F_j(\Delta):=\theta(\tilde t_j+\Delta)-\theta(\tilde t_j)+\mathcal P_{X(\tilde t_j)}(\tilde t_j,\Delta)-\pi\).
Then \(F_j(0)=-\pi<0\) and \(\theta(\tilde t_j+\Delta)-\theta(\tilde t_j)\to\infty\) as \(\Delta\to\infty\), while \(\mathcal P\) is bounded for fixed cutoff \(X(\tilde t_j)\). Hence a root exists.
Moreover
\[
F_j'(\Delta)=\theta'(\tilde t_j+\Delta)-\sum_{p^k\le X(\tilde t_j)}\frac{\log p}{p^{k/2}}W\!\Big(\frac{p^k}{X(\tilde t_j)}\Big)\cos\big((\tilde t_j+\Delta)k\log p\big).
\]
As \(t\to\infty\), \(\theta'(t)=\tfrac12\log(t/2\pi)+O(1/t)\).
The prime sum is \(O(\sum_{p^k\le X}\log p/p^{k/2})=O(\sqrt{X})\).
With \(X(\tilde t_j)=C(\log\tilde t_j)^{3/2}\), we have \(\sqrt{X}=O((\log\tilde t_j)^{3/4})=o(\log\tilde t_j)\),
hence \(F_j'(\Delta)>0\) for all large \(j\), so \(F_j\) is strictly increasing and the root is unique.
Bisection converges by monotonicity.
\end{proof}

\subsection*{Numerical audit to \(j=50\): error–vs–cutoff (fixed \(A=\tfrac32\))}
We compare \(\tilde m_j=2\tilde t_j\) to true \(m_j=2\gamma_j\) from LMFDB (Ref.~\cite{LMFDB}; Appendix~\ref{app:audit-protocol}).
Errors exclude the seeded \(j=1\).

\begin{center}
\begin{tabular}{@{}rcccc@{}}
\toprule
$C$ & $\max|\tilde m-m|$ & mean\,$|\tilde m-m|$ & $\max$ rel.\ err & mean rel.\ err\\
\midrule
16 & 0.106406 & 0.028070 & 0.000476 & 0.000165\\
32 & 0.087644 & 0.022884 & 0.000395 & 0.000133\\
48 & 0.057151 & 0.017504 & 0.000323 & 0.000109\\
\bottomrule
\end{tabular}
\end{center}

\begin{figure}[H]
\centering
\begin{tikzpicture}
\begin{axis}[
  width=0.7\linewidth, height=6cm,
  xlabel={$C$}, ylabel={Mean $|\tilde m-m|$},
  ymin=0.015, ymax=0.030, xmin=14, xmax=50,
  xtick={16,32,48}, ytick={0.015,0.020,0.025,0.030},
  grid=both, grid style={densely dotted}
]
\addplot coordinates {(16,0.028070) (32,0.022884) (48,0.017504)};
\end{axis}
\end{tikzpicture}
\caption{Mean absolute tick error decreases as $C$ grows (fixed $A=3/2$; $j=2,\dots,50$).}
\end{figure}

%------------------------------------------------------------------------------------------
% Appendices
%------------------------------------------------------------------------------------------
\appendix

\section{Corner interpolation (detail)}\label{app:corner}
Rectangles are Wiener-regular domains; if boundary data are continuous on \(\partial B\), the Dirichlet solution is continuous on \(\overline B\).
This allows one to treat \(\log|E|\) as boundary data for the outer construction away from boundary zeros and to interpret corner limits in the usual way.
See Kellogg \cite{Kellogg} and Axler--Bourdon--Ramey \cite[Ch.~8]{AxlerBourdonRamey}.

\section{Outer/Rouch\'e certification protocol (rigorous outline)}\label{app:cert}
\begin{itemize}[leftmargin=1.2em]
\item \textbf{Boundary sampling.} Partition each edge of \(\partial B\) into intervals; on each, compute interval enclosures for \(E(v)\) and (optionally) \(E'(v)\).
\item \textbf{Validated Dirichlet.} Solve Dirichlet for \(U\) with boundary data \(\log|E|\) using a validated Poisson solver (map to \(\D\) if desired).
\item \textbf{Phase reconstruction.} Obtain \(V\) as a harmonic conjugate (validated Hilbert transform on the circle after conformal mapping, or a direct validated Cauchy integral on \(\partial B\)).
\item \textbf{Outer function.} Form \(\Gout=e^{U+\ii V}\) with interval arithmetic.
\item \textbf{Rouch\'e check.} Verify \(\sup_{\partial B}|E-\Gout|/|\Gout|<1\) with grid-to-continuum Lipschitz enclosure (using \(E'/E\) bounds).
\end{itemize}

\section{Certified first nontrivial zero and verified band}\label{app:firstheight-certified}
We cite rigorously verified computations:
\begin{theorem}[Platt 2017; Platt--Trudgian 2021]
There are no nontrivial zeros of $\zeta(s)$ with $0<\Imag s<t_1$, and the first nontrivial zero occurs at
$t_1=14.134725141734693790457251983562\ldots$ with rigorous interval bounds.
Moreover, RH has been verified for all zeros with $0<\Imag s\le 3\cdot 10^{12}$.
\end{theorem}
Set $m_1:=2t_1$.

\section*{Appendix S.1. Shape-only operator norms (dependency statement)}\label{app:S1}
On a Lipschitz Jordan curve (in particular, the square boundary), the Cauchy singular integral and boundary Hilbert transform are bounded on \(L^2\) with operator norms depending only on the Lipschitz character (Coifman--McIntosh--Meyer \cite{CoifmanMcIntoshMeyer}).
Under affine normalization of \(\partial B\) to the fixed square \(\partial Q\), these norms become \emph{shape-only}.
A fully explicit numerical enclosure of these norms is possible via certified quadrature on \(\partial Q\).

\section*{Appendix S.3. Pinned constants and tail threshold logic (program)}\label{app:S3}
Theorem~\ref{thm:tail-criterion} reduces the tail threshold \(M_0(\eta)\) to a finite family of constants:
residual constants \(C_1,C_2\) and shape-only constants (trace/Hilbert/geometric allocation).
To turn the tail into a complete proof, one must:
\begin{enumerate}[leftmargin=1.2em,label=(\arabic*)]
\item instantiate \(C_1,C_2\) from explicit literature bounds for \(\zeta'/\zeta\) on vertical strips;
\item certify the shape-only operator norms on the normalized square boundary (Appendix~\ref{app:S1}) or bound them explicitly;
\item compute \(M_0(\eta)\) and verify \(M_0(\eta)\le 2\cdot 3\cdot 10^{12}\) so that the Platt--Trudgian band closes the remainder.
\end{enumerate}
The v24-style numerical pinning example can be carried out here, but for submission it should be presented as a \emph{certified} enclosure rather than floating-point approximations.

\section*{Appendix PW. Smooth cutoff weight}\label{app:PW}
Define \(W:[0,1]\to[0,1]\) by
\[
W(y):=
\begin{cases}
\exp\!\Big(1-\dfrac{1}{1-y}\Big), & 0\le y<1,\\[6pt]
0, & y=1.
\end{cases}
\]
Interpret \(W(y)=0\) for \(y>1\).

\section*{Appendix NA. Deterministic audit protocol and full reference script}\label{app:audit-protocol}
\paragraph{Purpose.}
This appendix freezes the data source, theta function, weight, cutoff rule, solver, and error metrics used in the Part III audit.

\paragraph{Truth ordinates.}
Download \(\gamma_1,\dots,\gamma_{50}\) from the LMFDB plain-text endpoint:
\[
\texttt{https://www.lmfdb.org/zeros/zeta/list?download=yes\&limit=100}.
\]
Parse index/value pairs \(j\ \gamma_j\).

\paragraph{Audit protocol.}
\begin{itemize}[leftmargin=1.2em]
\item Seed: \(\tilde t_1:=\gamma_1\).
\item Theta: \(\theta(t)=\Imag\log\Gamma(\frac14+\frac{\ii t}{2})-\frac{t}{2}\log\pi\) (principal branch).
\item Weight: \(W\) from Appendix~\ref{app:PW}.
\item Cutoff: \(X(t)=C(\log t)^{3/2}\).
\item Root solve: bisection on \(F_j(\Delta)=0\) where \(F_j\) is defined in Thm.~\ref{thm:generator}.
\item Statistics: compare \(\tilde m_j=2\tilde t_j\) to \(m_j=2\gamma_j\), exclude \(j=1\) in summary stats.
\end{itemize}

\paragraph{Reference script (Python 3).}
\begin{verbatim}
#!/usr/bin/env python3
"""
Deterministic audit script for the Part III prime-locked tick generator.

Reproduces the v26 Part III table (A = 3/2; C in {16,32,48}; J = 50)
with deterministic bisection and a fixed smooth cutoff weight W.

Truth ordinates are fetched from LMFDB's plain-text endpoint; if the fetch fails,
the script falls back to an embedded list for j=1..50.

Notation:
  - gamma_j: true ordinates from LMFDB
  - ttilde_j: generated tick ordinates
  - m_j = 2*gamma_j, mtilde_j = 2*ttilde_j

No circularity:
  - X(t) uses the predicted tick ttilde_j at each step.
  - Truth list is used only for reporting errors.
"""

import argparse
import math
import urllib.request

import mpmath as mp
mp.mp.dps = 60  # fixed precision for theta

LMFDB_URL = "https://www.lmfdb.org/zeros/zeta/list?download=yes&limit=100"

FALLBACK_GAMMA_50 = [
"14.1347251417346937904572519835625",
"21.0220396387715549926284795938969",
"25.0108575801456887632137909925628",
"30.4248761258595132103118975305840",
"32.9350615877391896906623689640747",
"37.5861781588256712572177634807053",
"40.9187190121474951873981269146334",
"43.3270732809149995194961221654068",
"48.0051508811671597279424727494277",
"49.7738324776723021819167846785638",
"52.9703214777144606441472966088808",
"56.4462476970633948043677594767060",
"59.3470440026023530796536486749922",
"60.8317785246098098442599018245241",
"65.1125440480816066608750542531836",
"67.0798105294941737144788288965221",
"69.5464017111739792529268575265547",
"72.0671576744819075825221079698261",
"75.7046906990839331683269167620305",
"77.1448400688748053726826648563047",
"79.3373750202493679227635928771161",
"82.9103808540860301831648374947706",
"84.7354929805170501057353112068276",
"87.4252746131252294065316678509191",
"88.8091112076344654236823480795095",
"92.4918992705584842962597252418105",
"94.6513440405198869665979258152080",
"95.8706342282453097587410292192467",
"98.8311942181936922333244201386224",
"101.3178510057313912287854479402924",
"103.7255380404783394163984081086952",
"105.4466230523260944936708324141119",
"107.1686111842764075151233519630860",
"111.0295355431696745246564503099445",
"111.8746591769926370856120787167707",
"114.3202209154527127658909372761910",
"116.2266803208575543821608043120647",
"118.7907828659762173229791397026999",
"121.3701250024206459189455329704998",
"122.9468292935525882008174603307700",
"124.2568185543457671847320079661301",
"127.5166838795964951242793237669060",
"129.5787041999560509857680339061800",
"131.0876885309326567235663724615015",
"133.4977372029975864501304920426407",
"134.7565097533738713313260641571699",
"138.1160420545334432001915551902824",
"139.7362089521213889504500465233824",
"141.1237074040211237619403538184753",
"143.1118458076206327394051238689139",
]

def fetch_lmfdb_gammas(limit: int = 50, url: str = LMFDB_URL, timeout: int = 20):
    """
    Returns a list of decimal strings gamma_1..gamma_limit.
    Endpoint returns plain text: "1 gamma1 2 gamma2 ..."
    """
    try:
        with urllib.request.urlopen(url, timeout=timeout) as f:
            txt = f.read().decode("utf-8", errors="replace").strip()
        parts = txt.split()
        gammas = parts[1::2]
        if len(gammas) < limit:
            raise ValueError("LMFDB response too short")
        return gammas[:limit], "LMFDB"
    except Exception:
        return FALLBACK_GAMMA_50[:limit], "FALLBACK"

def theta_float(t: float) -> float:
    # mpmath's siegeltheta implements the standard Riemann–Siegel theta.
    return float(mp.siegeltheta(t))

def W_cutoff(y: float) -> float:
    # Smooth one-sided cutoff on (0,1): W(0+)=1, W(y)->0 rapidly as y->1-.
    if y <= 0.0 or y >= 1.0:
        return 0.0
    return math.exp(1.0 - 1.0/(1.0 - y))

def primes_upto(n: int):
    if n < 2:
        return []
    sieve = bytearray(b"\x01") * (n + 1)
    sieve[:2] = b"\x00\x00"
    r = int(n ** 0.5)
    for p in range(2, r + 1):
        if sieve[p]:
            start = p * p
            step = p
            sieve[start:n+1:step] = b"\x00" * (((n - start) // step) + 1)
    return [i for i in range(2, n + 1) if sieve[i]]

def prime_power_terms(X: float, t: float):
    """
    Precompute omega=log(p^k), coeff=W(p^k/X)/(k*sqrt(p^k)),
    and sin(t*omega), cos(t*omega) for fast evaluation of P(t,Delta).
    """
    N = int(X)
    ps = primes_upto(N)
    omegas, coeffs, sin0, cos0 = [], [], [], []
    for p in ps:
        n = p
        k = 1
        while n <= N:
            y = n / X
            w = W_cutoff(y)
            if w != 0.0:
                coeff = w / (k * math.sqrt(n))
                omega = math.log(n)
                omegas.append(omega)
                coeffs.append(coeff)
                ang = t * omega
                sin0.append(math.sin(ang))
                cos0.append(math.cos(ang))
            k += 1
            n *= p
    return omegas, coeffs, sin0, cos0

def P_prime_increment(terms, Delta: float) -> float:
    """
    Compute P_{X}(t,Delta) using the precomputed terms at t.
    """
    omegas, coeffs, sin0, cos0 = terms
    s = 0.0
    for omega, coeff, s0, c0 in zip(omegas, coeffs, sin0, cos0):
        # sin((t+Delta)omega)-sin(t omega) = s0*(cos(Delta omega)-1) + c0*sin(Delta omega)
        d = s0 * (math.cos(Delta * omega) - 1.0) + c0 * math.sin(Delta * omega)
        s += coeff * d
    return -s

def next_tick(tj: float, C: float, A: float, tol: float = 1e-12, max_iter: int = 100):
    """
    Solve for the next tick t_{j+1} from t_j by bisection on F_j(Delta)=0.
    """
    X = C * (math.log(tj) ** A)
    terms = prime_power_terms(X, tj)
    theta_tj = theta_float(tj)

    def F(Delta: float) -> float:
        return (theta_float(tj + Delta) - theta_tj) + P_prime_increment(terms, Delta) - math.pi

    # Initial heuristic gap near 2pi/log(t/2pi)
    denom = math.log(max(tj / (2.0 * math.pi), 1.0000001))
    gap0 = 2.0 * math.pi / denom if denom > 0 else 10.0

    lo = 0.0
    hi = max(1.0, 2.0 * gap0)
    f_hi = F(hi)
    it = 0
    while f_hi <= 0.0 and it < 60:
        hi *= 2.0
        f_hi = F(hi)
        it += 1
    if f_hi <= 0.0:
        raise RuntimeError(f"Failed to bracket root at t={tj} (hi={hi}, F(hi)={f_hi})")

    for _ in range(max_iter):
        mid = 0.5 * (lo + hi)
        if F(mid) <= 0.0:
            lo = mid
        else:
            hi = mid
        if hi - lo < tol:
            break

    return tj + 0.5 * (lo + hi)

def generate_ticks(t1: float, J: int, C: float, A: float):
    ts = [t1]
    t = t1
    for _ in range(1, J):
        t = next_tick(t, C=C, A=A)
        ts.append(t)
    return ts

def error_stats(ts_tick, gammas_true, exclude_seed: bool = True):
    start = 1 if exclude_seed else 0
    m_tick = [2.0 * t for t in ts_tick[start:]]
    m_true = [2.0 * float(g) for g in gammas_true[start:len(ts_tick)]]

    abs_err = [abs(a - b) for a, b in zip(m_tick, m_true)]
    rel_err = [ae / abs(mt) for ae, mt in zip(abs_err, m_true)]

    return {
        "max_abs": max(abs_err),
        "mean_abs": sum(abs_err) / len(abs_err),
        "max_rel": max(rel_err),
        "mean_rel": sum(rel_err) / len(rel_err),
    }

def main():
    ap = argparse.ArgumentParser()
    ap.add_argument("--J", type=int, default=50)
    ap.add_argument("--A", type=float, default=1.5)
    ap.add_argument("--Cs", type=str, default="16,32,48")
    ap.add_argument("--no_fetch", action="store_true")
    ap.add_argument("--include_seed_in_stats", action="store_true")
    args = ap.parse_args()

    Cs = [float(x.strip()) for x in args.Cs.split(",") if x.strip()]

    if args.no_fetch:
        gammas, source = (FALLBACK_GAMMA_50[:args.J], "FALLBACK(forced)")
    else:
        gammas, source = fetch_lmfdb_gammas(limit=max(args.J, 50))

    # Seed uses gamma_1 to match the frozen protocol
    t1 = float(gammas[0])

    exclude_seed = not args.include_seed_in_stats

    print(f"[audit] source={source}  J={args.J}  A={args.A}  Cs={Cs}")
    print("[audit] computing...")

    # LaTeX-ready rows
    for C in Cs:
        ts_tick = generate_ticks(t1, args.J, C=C, A=args.A)
        st = error_stats(ts_tick, gammas, exclude_seed=exclude_seed)
        print(
            f"{int(C)} & "
            f"{st['max_abs']:.6f} & {st['mean_abs']:.6f} & "
            f"{st['max_rel']:.6f} & {st['mean_rel']:.6f} \\\\"
        )

if __name__ == "__main__":
    main()
\end{verbatim}

% -----------------------------------------------------------------------------------------
% Bibliography
% -----------------------------------------------------------------------------------------
\clearpage
\phantomsection
\addcontentsline{toc}{section}{References}
\begin{thebibliography}{99}

\bibitem{AxlerBourdonRamey}
S.~Axler, P.~Bourdon, and W.~Ramey, \emph{Harmonic Function Theory}, 2nd ed., Springer, 2001.

\bibitem{CoifmanMcIntoshMeyer}
R.~R.~Coifman, A.~McIntosh, and Y.~Meyer,
L’int\'egrale de Cauchy d\'efinit un op\'erateur born\'e sur $L^2$ pour les courbes lipschitziennes,
\emph{Ann. of Math.} \textbf{116} (1982), 361--387.

\bibitem{DLMF}
NIST Digital Library of Mathematical Functions, \S5.5, \S5.11.
\url{https://dlmf.nist.gov/}

\bibitem{DurenHp}
P.~L.~Duren, \emph{Theory of $H^p$ Spaces}, Academic Press, 1970.

\bibitem{GarnettBAF}
J.~B.~Garnett, \emph{Bounded Analytic Functions}, Springer, 2007.

\bibitem{Ivic}
A.~Ivi\'c, \emph{The Riemann Zeta-Function}, John Wiley \& Sons, 1985.

\bibitem{Kellogg}
O.~D.~Kellogg, \emph{Foundations of Potential Theory}, Dover, 1953.

\bibitem{Platt2017}
D.~J.~Platt, \emph{Isolating some nontrivial zeros of $\zeta(s)$}, \emph{Math. Comp.} \textbf{86} (2017), 2449--2467.

\bibitem{PlattTrudgian2021}
D.~J.~Platt and T.~S.~Trudgian, \emph{The Riemann hypothesis is true up to $3\cdot 10^{12}$},
\emph{Bull. Lond. Math. Soc.} \textbf{53} (2021), 792--797.

\bibitem{Titchmarsh}
E.~C.~Titchmarsh (rev. D.~R.~Heath--Brown),
\emph{The Theory of the Riemann Zeta-Function}, 2nd ed., Oxford, 1986.

\bibitem{LMFDB}
The LMFDB Collaboration, \emph{The L-functions and Modular Forms Database}.\\
Zeros of the Riemann zeta function: \url{https://www.lmfdb.org/zeros/zeta/}.\\
Plain-text endpoint: \url{https://www.lmfdb.org/zeros/zeta/list?download=yes&limit=100}.

\end{thebibliography}

\clearpage
\section*{Authorship and AI--Use Disclosure}
\phantomsection
\addcontentsline{toc}{section}{Authorship and AI--Use Disclosure}
The author designed the framework and validated the mathematics and computations. Generative assistants were used for typesetting assistance, editorial organization, and consistency checks; they are not authors. All claims are the author's responsibility.

\end{document}
```