% ======================================================================
% Master Manuscript — Part I (Reader's Guide) + Part II (Analytic Core) + Part III (Structural Corollaries)
% ======================================================================

\documentclass[11pt]{article}

% ------------------ Basic packages (minimal, used) ------------------
\usepackage[a4paper,margin=1in]{geometry}
\usepackage{amsmath,amssymb,amsthm,mathtools}
\usepackage{microtype}
\usepackage{hyperref}
\usepackage{tabularx,booktabs,array}
\usepackage{enumitem}

% ------------------ Theorem styles ------------------
\newtheorem{theorem}{Theorem}[section]
\newtheorem{lemma}[theorem]{Lemma}
\newtheorem{corollary}[theorem]{Corollary}
\theoremstyle{remark}
\newtheorem{remark}[theorem]{Remark}

% ------------------ Tabularx helper column ------------------
\newcolumntype{L}{>{\raggedright\arraybackslash}X}

% ------------------ Macros: frames, functions, projectors ------------------
\newcommand{\C}{\mathbb{C}}
\newcommand{\R}{\mathbb{R}}
\newcommand{\Z}{\mathbb{Z}}
\newcommand{\D}{\mathbb{D}}
\newcommand{\Real}{\operatorname{Re}}
\newcommand{\Imag}{\operatorname{Im}}
\newcommand{\zetaTwo}{\zeta_2}
\newcommand{\LambdaTwo}{\Lambda_2}
\newcommand{\LamTwo}{\LambdaTwo}
\newcommand{\Afac}{A_2}
\newcommand{\chiTwo}{\chi_2}
\newcommand{\Podd}{P_{\mathrm{odd}}}
\newcommand{\Peven}{P_{\mathrm{even}}}
\newcommand{\Ucore}{U}
\newcommand{\UR}{U_{\mathrm{R}}}
\newcommand{\UL}{U_{\mathrm{L}}}
\newcommand{\Ecomp}{E}
\newcommand{\Gout}{G_{\mathrm{out}}}
\newcommand{\Zloc}{Z_{\mathrm{loc}}}
\newcommand{\Arg}{\operatorname{Arg}}
\newcommand{\sgn}{\operatorname{sgn}}
\newcommand{\ii}{\mathrm{i}}

% ------------------ Compact inline overview block (optional) ------------------
\newenvironment{Overview}{\begin{quote}\itshape}{\end{quote}}

% ------------------ Title page ------------------
\title{\Large A Height--Local Width--2 Program for Excluding Off--Axis Quartets\\[2pt]
\large Analytic Tail \& Certified Outer/Rouch\'e Criterion}
\author{Dylan Anthony Dupont\thanks{%
\textbf{Authorship and AI-use disclosure.} The author designed the framework, chose all constants/normalizations, and validated all mathematics and computations. Generative assistants (from GPT--4o to GPT--5~Pro) were used solely for typesetting assistance, editorial organization, and consistency checks; they are not an author. All claims are the author's responsibility (COPE/ICMJE guidance).}}
\date{\today}

\begin{document}
\maketitle

\begin{abstract}
\noindent
This paper is organized in three parts. 
\textbf{Part~I} (Reader’s Guide) reduces the Riemann Hypothesis (RH) to a height--local statement in the width--2 frame: \emph{RH $\Leftrightarrow$ $a(m)=0$ at each nontrivial height $m$}, while recording non--load--bearing structural scaffolding.
\textbf{Part~II} gives a self--contained, boundary--only analytic proof that the per--height tilt satisfies $a(m)=0$ at every nontrivial height using a disc--based $L^2$ upper envelope and an $L^2$ lower envelope via allocation $+$ restricted contour $+$ Jensen. We also provide a rigorous Outer/Rouch\'e Certification Path with explicit domains and symbolic constants (``shape--only'' vs.\ residual).
\textbf{Part~III} promotes the toolbox identities to structural corollaries once $a(m)=0$ is established.
\end{abstract}

\setcounter{tocdepth}{2}
\tableofcontents

% ======================================================================
% Part I — Reader’s Guide / Motivation, Reduction & Implications
% ======================================================================
\section*{Part I --- Reader’s Guide / Motivation, Reduction \& Implications}
\addcontentsline{toc}{section}{Part I --- Reader’s Guide / Motivation, Reduction \& Implications}

\paragraph{What this section is (and is not).}
\emph{What it does.} It introduces modulated frames and the width--2 normalization, defines the centered $a$--lens that measures horizontal tilt at a fixed height, and reduces RH to the height--local target $a(m)=0$ for each nontrivial height $m$. It also records the structural toolbox (projectors, rectifier, canonical stream, recurrence, curvature extractor, seed$\to$rectifier) and explains how these become consequences once $a(m)=0$ is proved.

\noindent\emph{What it does not do.} It contains no analytic estimates and no proofs. The hinge unitarity fact and all bounds are proved later; this Guide is not used by the analytic part.

\subsection*{1) Modulated frames and the width--2 pivot}
For $f>0$ define the modulated family $\zeta_f(s):=\zeta(s/f)$ with completed form
\[
\Lambda_f(s)=\pi^{-\,s/(2f)}\,\Gamma\!\Big(\frac{s}{2f}\Big)\,\zeta_f(s),
\]
so $\Lambda_f$ is entire and satisfies $\Lambda_f(s)=\Lambda_f(f-s)$. Equivalently, $\zeta_f(s)=A_f(s)\,\zeta_f(f-s)$ with $A_f(s)A_f(f-s)\equiv1$.

\smallskip
\noindent\textbf{Width--2 normalization.} Put $u:=(2/f)\,s$. Then
\[
\zetaTwo(u):=\zeta(u/2),\qquad
\LambdaTwo(u):=\pi^{-u/4}\Gamma(u/4)\,\zeta(u/2),\qquad
\LambdaTwo(u)=\LambdaTwo(2-u).
\]
The non--completed FE reads $\zetaTwo(u)=\Afac(u)\,\zetaTwo(2-u)$.
In the open strip $0<\Real u<2$ and $\Imag u\neq0$, $\Afac$ is analytic and nonvanishing.

\smallskip
\noindent\textbf{Partner map.} On $\Imag u>0$, FE $+$ conjugation gives the involution $J(u)=2-\overline{u}$, swapping the two column points at the same height.

\smallskip
\noindent\textbf{Hinge unitarity (deferred).} The statement ``$|\chiTwo(u)|=|\Afac(u)|^{-1}=1$'' iff $\Real u=1$ is proved in Part~II (Hinge--Unitarity). We do not use it here.

\subsection*{2) Centered $a$--lens and the quartet}
Let $v:=u-1$ and $\Ecomp(v):=\LambdaTwo(1+v)$. Then $\Ecomp(v)=\Ecomp(-v)=\overline{\Ecomp(\overline v)}$.

\smallskip
\noindent\textbf{Nontrivial height.} A ``nontrivial height'' $m>0$ means: $m$ occurs as the imaginary part of a nontrivial zero $s=\tfrac12+\ii m/2$. The reduction shows that whenever such an $m$ occurs, the associated tilt must satisfy $a(m)=0$.

\smallskip
\noindent\textbf{Tilt at height $m$.} At fixed $m>0$, set
\[
\UR(m;a)=1+a+\ii m,\qquad \UL(m;a)=1-a+\ii m,\qquad a\in[0,1).
\]
In the centered frame, the dial points are $\pm(a+\ii m)$. The partner map $J$ swaps $\UR\leftrightarrow \UL$.

\smallskip
\noindent\textbf{Quartet.} Conjugation (top$\leftrightarrow$bottom) and FE reflection generate the quartet $\{\,1\pm a\pm \ii m\,\}$ at height $m$.

\subsection*{3) Why width--2: slope invariance}
If the columns collapse at height $m$ ($a=0$), the point is $u=1+\ii m$ and its slope is $\Imag u/\Real u = m/1=m$. Rescaling to any frame $s=(f/2)\,u$ preserves the slope:
\[
\frac{\Imag s}{\Real s}=\frac{(f/2)\,m}{f/2}=m.
\]
Thus $\{m_k\}$ simultaneously records the imaginary ordinates of the nontrivial zeros and their origin through slopes in every modulated frame---provided the per--height collapse holds.

\subsection*{4) Height--local reduction of RH}
Fix a nontrivial height $m>0$ and write $\UR=1+a+\ii m$, $\UL=1-a+\ii m$. The following are purely algebraic and equivalent:
\begin{itemize}
  \item (PHU--1) Column equality: $\Real \UR=\Real \UL \iff a=0$.
  \item (PHU--2) Ray (slope) lock: $\Imag \UR/\Real \UR=\Imag \UL/\Real \UL$, i.e.\ $m/(1+a)=m/(1-a)\iff a=0$.
  \item (PHU--3) Hinge form: $\UR=\UL=1+\ii m$.
\end{itemize}
\emph{Reduction target.} RH $\iff$ for every nontrivial height $m>0$, $a(m)=0$. Part~II proves this per--height collapse; nothing from this Guide is used there.

\subsection*{5) Box alignment and hand--off (no circularity)}
For later reference, define
\[
B(\alpha,m,\delta)=[\alpha-\delta,\alpha+\delta]\times[m-\delta,m+\delta],\qquad
\delta:=\eta\,\alpha/(\log m)^2,\ \ \eta\in(0,1).
\]
When $\alpha=\pm a$, the dial points $\pm(a+\ii m)$ lie on the box’s horizontal centerline.

\noindent\textbf{What Part~II does.} Using only boundary analysis on such boxes (completed FE symmetry, Cauchy--Riemann transport, three--lines tools, Stirling--class envelopes, explicit control of $\zeta'/\zeta$ away from zeros), Part~II shows that any off--axis quartet forces a boundary lower bound larger than an explicit upper bound, hence $a(m)=0$.

\noindent\textbf{No circularity.} The analytic proof is logically independent of this Guide.

\subsection*{6) Parity gating and selection devices (interpretive only)}
\textbf{Gating from the non--completed FE.} In the width--2 frame the non--completed FE reads
\[
\zetaTwo(u)=\Afac(u)\,\zetaTwo(2-u),\quad
\Afac(u)=2^{u/2}\,\pi^{\,u/2-1}\,\sin\!\Big(\frac{\pi u}{4}\Big)\,\Gamma\!\Big(1-\frac{u}{2}\Big).
\]
On the open strip $0<\Real u<2$ with $\Imag u\neq0$, the prefactor $\Afac(u)$ is nonzero and finite; its sine zeros (the “trivial ladder”) lie on the real axis only. Thus \emph{inside the open strip only $\zetaTwo$ can vanish} (nontrivial zeros), while the \emph{trivial class is confined to the real axis}. This is the basic “odd/even lane” picture: the odd (upper) lane can host nontrivial zeros; the even (real) lane hosts the trivial ladder.

\smallskip
\noindent\textbf{Orthogonal split on the integer lattice.} To model this dichotomy as a clean input--space symmetry, decompose any lattice signal $X:\Z\to\C$ via the orthogonal projectors
\[
\Podd(n)=\tfrac{1-\cos(\pi n)}{2},\qquad
\Peven(n)=\tfrac{1+\cos(\pi n)}{2},
\]
so $X=\Podd X+\Peven X$. We \emph{assign the nontrivial stream to odd slots} (where $\Podd=1$) and the \emph{trivial ladder to even slots} (where $\Peven=1$). This mirrors the FE fact above without using it analytically.

\subsection*{7) Toolbox $\to$ structural consequences (after the theorem)}
The items below are not inputs to the analytic proof. After Part~II proves $a(m)=0$ for all nontrivial heights, they become \emph{Structural Corollaries} describing the collapsed geometry and its lattice faces (brief proofs appear in Part~III).

\begin{itemize}
  \item Pre--collapse columns (projector faces in the $u$--frame): right/left templates place odd--slot samples $x\pm \ii m_k$ and the even ladder $-4(\cdot)$ via $\Podd,\Peven$.
  \item Collapsed canonical stream $U(n)$: when per--height collapse holds ($x=1$ on odd slots), the two columns coincide; parity face (via $\Podd,\Peven$) and an equivalent trigonometric face.
  \item Single--frequency collapse (cosine face): a two--parameter cosine form $U(n)=(c+d)+(c-d)\cos(\pi n)$; $c,d$ simple in the odd--indexer $k(n)$.
  \item Self--indexed recurrence (no explicit $k$): a short recurrence for $U(n)$ pulls the needed odd index from the previous even sample.
  \item Curvature extractor \& the $\zeta(2)$ disguise: the discrete second difference of the imaginary part at even indices recovers $m_j$ and admits an odd--square convolution normalized by $\zeta(2)$.
  \item Seed $\to$ rectifier $\to$ physical streams: two--carrier seeds rectify under a mod--4 factor to yield the physical stream $S_f(n)\propto U(n)$; pre--collapse faces scale analogously.
\end{itemize}

\subsection*{8) Implications and one--sentence hand--off}
The width--2 organization centralizes symmetry at $\Real u=1$; the centered $a$--lens isolates the single per--height degree of freedom; parity--orthogonal scaffolding separates the nontrivial stream from the ladder without entering the proof. With these definitions, RH reduces to: for every nontrivial height $m>0$, $a(m)=0$.

% ======================================================================
% Part II — Analytic Core (self-contained; boundary-only)
% ======================================================================
\section*{Part II --- Self-Contained Boundary--Only Contradiction on Aligned Boxes}
\addcontentsline{toc}{section}{Part II --- Self-Contained Boundary--Only Contradiction on Aligned Boxes}

In the width-2 centered frame $u=2s$, $v=u-1$, let $\LamTwo(u)=\pi^{-u/4}\Gamma(u/4)\zeta(u/2)$ and $E(v)=\LamTwo(1+v)$. We present a boundary-only, height-local program to exclude off-axis quartets $\{\pm a\pm \ii m\}$ via two complementary routes:
\begin{enumerate}[label=(\arabic*)]
\item an analytic tail (uniform in $\alpha\in(0,1]$) using only: (i) explicit short-side forcing $\ge \pi/2$; (ii) a residual bound for $F=E/\Zloc$ with perimeter factor $8\delta$; and (iii) a disc-based, $L^2$ boundary-to-midpoint estimate with \emph{shape-only} constants;
\item a rigorous Outer/Rouch\'e Certification Path: interval arithmetic on $\partial B$ + validated Poisson + Lipschitz grid$\to$continuum enclosure $\Rightarrow \sup_{\partial B}\!\big|E-\Gout\big|/|\Gout|<1 \Rightarrow$ zero-free box, followed by Bridge~1 (inner collapse $W\equiv e^{\ii\theta}$) and Bridge~2 (stitching).
\end{enumerate}
We also prove a corner outer interpolation from continuous Dirichlet data. The tail is stated with symbolic constants: for each fixed $\eta\in(0,1)$ there exists $M_0(\eta)$ such that no off-axis quartet lies in any $B(\alpha,m,\delta)$ with $\delta=\eta\alpha/(\log m)^2$ for all $m\ge M_0(\eta)$, uniformly in $\alpha$. Combined with a certified base range below $m_1$ (first nontrivial height in width-2), this yields the global on-axis theorem. All constants in the upper/lower envelope are \emph{shape-only}; residual constants are kept symbolic in theorems and may be instantiated from classical literature in an appendix.

% ---------------------------------------------------
\section*{Symbols \& Provenance (at a glance)}
% ---------------------------------------------------

\noindent\textit{Notation hygiene.} We reserve $\psi$ for the digamma function and write $\varphi:\D\to B$ for conformal maps.

\medskip
\begin{tabularx}{\textwidth}{@{}p{3.2cm} L L@{}}
\toprule
\textbf{Symbol} & \textbf{Definition / role} & \textbf{Provenance / why this form}\\
\midrule
$u=2s$, $v=u-1$ & Width-2 frame centered at $\Real u=1$ & Centers functional equation symmetry\\[2pt]
$\LamTwo(u)=\pi^{-u/4}\Gamma\!\big(\frac{u}{4}\big)\zeta\!\big(\frac{u}{2}\big)$ & Completed object & Standard; FE for $\LamTwo$; width-2 transport\\[2pt]
$E(v)=\LamTwo(1+v)$ & Workhorse in $v$-plane & Even \& conjugate-symmetric: $E(v)=E(-v)=\overline{E(\bar v)}$\\[2pt]
$\zeta_2(u)=\zeta(u/2)$ & Width-2 zeta & Used in FE and hinge law\\[2pt]
$\chiTwo(u)$ & FE factor inverse & $\chiTwo(u)=\pi^{u/2-1/2}\frac{\Gamma((2-u)/4)}{\Gamma(u/4)}$\\[2pt]
$B(\alpha,m,\delta)$ & $[\alpha-\delta,\alpha+\delta]\times[m-\delta,m+\delta]$ & Square (width \& height $2\delta$) centered at $(\alpha,m)$\\[2pt]
$\alpha\in(0,1]$ & Horizontal center & Left dial handled by reflection $w=-v$\\[2pt]
$m\ge 10$ & Height parameter & Ensures uniform DLMF/Titchmarsh/Ivi\'c regimes\\[2pt]
$\delta=\dfrac{\eta\,\alpha}{(\log m)^2}$, $\eta\in(0,1)$ & Half-side length of $B$ & Balances forcing vs residual $O(\delta\log m)$\\[4pt]
$\partial B$ & Boundary of $B(\alpha,m,\delta)$ & Boundary integrals/suprema\\[2pt]
$I_\pm$ & Short vertical sides of $\partial B$ & Near/far verticals in forcing budgets\\[2pt]
$Q$ & Quiet arcs (horizontal sides of $\partial B$) & Controlled by $L^2$ trace \& Hilbert\\[2pt]
$\Zloc(v)=\prod_{|\Imag\rho-m|\le 1}(v-\rho)^{m_\rho}$ & Local zero/pole factors & De-singularizes $E$ near $\partial B$\\[2pt]
$F=E/\Zloc$ & Residual analytic factor (nonvanishing near $\partial B$) & Lemma~\ref{lem:residual} (constants symbolic)\\[2pt]
$G(v)=\dfrac{E(1+v)}{E(1-v)}$ & Odd-lane quotient & Links to hinge via two-point identity\\[2pt]
$\Gout=e^{U+iV}$ & Outer with $|\,\Gout\,|=|E|$ on $\partial B$ & $U=\log|E|\in C(\overline B)$ solves Dirichlet; $V$ harmonic conj.\\[2pt]
$W=E/\Gout$ & Inner quotient ($|W|=1$ a.e. on $\partial B$) & Collapses to unimodular constant upon certification\\[2pt]
$v_\pm^\star=\pm(a+im)$ & Dial pair on centerline & Points of evaluation in the tail\\[2pt]
$Z_{\rm pair}(v)$ & $(v-(a+im))(v-(-a+im))$ & Short-side forcing on $I_+$\\[2pt]
$\Gamma_\lambda$ & Central $\lambda\delta$ sub-arcs on verticals + tiny joins & Restricted contour (zero forcing)\\[2pt]
$B_{\rm core}(a,m;\lambda)$ & Dial-centred core box & Zero location forced by $\Gamma_\lambda$\\[2pt]
$K_{\rm alloc}^{(\star)}(\lambda)$ & Allocation coefficient & Shape-only; Lemma~\ref{lem:allocL2}\\[2pt]
$c_0=\frac{1}{4\pi}\log(2\sqrt{2})$ & Dial deficit constant ($\lambda=\tfrac12$) & From Jensen at dial; Lemma~\ref{lem:jensen-dial}\\[2pt]
$C_{\mathrm{up}}$ & Upper-envelope constant & Shape-only; Lemma~\ref{lem:upper-disc}\\[2pt]
$C_h''$ & Horizontal budget constant & Shape-only; Lemma~\ref{lem:corezero}\\
\bottomrule
\end{tabularx}

\medskip
\noindent\textit{Sources.} Digamma: DLMF §5.5 (reflection), §5.11 (vertical-strip bounds). $\zeta'/\zeta$: Titchmarsh, \emph{The Theory of the Riemann Zeta-Function}, §14; Ivi\'c, \emph{The Riemann Zeta-Function}, Ch.~9. Lipschitz Hilbert/Cauchy and boundary traces: Coifman--McIntosh--Meyer (1982); Duren; Garnett.

% ---------------------------------------------------
\section{Frames, symmetry, and the hinge law}\label{sec:frames}
% ---------------------------------------------------

We work in the width-2 centered frame $u=2s$, $v=u-1$, with
\[
\LamTwo(u)=\pi^{-u/4}\Gamma\!\Big(\frac{u}{4}\Big)\zeta\!\Big(\frac{u}{2}\Big),
\qquad
E(v):=\LamTwo(1+v).
\]
Then $E(v)=E(-v)=\overline{E(\bar v)}$; off-axis zeros appear as quartets $\{\pm a\pm \ii m\}$. These symmetries follow from $\LamTwo(u)=\LamTwo(2-u)$ and $\overline{\LamTwo(\overline z)}=\LamTwo(z)$ on vertical strips, hence $E(v)=\LamTwo(1+v)=\LamTwo(1-v)=E(-v)$ and conjugation invariance.

\begin{theorem}[Hinge--Unitarity]\label{thm:hinge}
Let $\zeta_2(u)=\zeta(u/2)$ and $\zeta_2(u)=A_2(u)\,\zeta_2(2-u)$ with
\[
\chiTwo(u):=A_2(u)^{-1}=\pi^{u/2-1/2}\frac{\Gamma\big(\frac{2-u}{4}\big)}{\Gamma\big(\frac{u}{4}\big)}.
\]
For each fixed $t\neq 0$, define $f(\sigma)=\log|\chi_2(\sigma+\ii t)|$. Then
\[
f'(\sigma)=\tfrac12\log\pi-\tfrac12\,\Real\psi\!\Big(\tfrac{\sigma+\ii t}{4}\Big)
-\tfrac14\,\Real\!\Big[\pi\cot\!\Big(\tfrac{\pi}{4}(\sigma+\ii t)\Big)\Big].
\]
Moreover,
\[
\big|\Real\!\big[\pi\cot(x+\ii y)\big]\big|\;\le\;\frac{\pi}{\cosh(2y)-1}.
\]
Taking $x=\frac{\pi}{4}\sigma$, $y=\frac{\pi}{4}|t|$, for $|t|\ge m_1/2$ (with $m_1$ defined in Appendix~\ref{app:firstheight-certified}) the cotangent term is $<10^{-8}$. Using vertical-strip bounds,
\[
\Real\psi\!\Big(\frac{\sigma+\ii t}{4}\Big)\ \ge\ \log\!\Big(\frac{|t|}{4}\Big)-\frac{2}{|t|},
\]
hence $f'(\sigma)<0$ on $\R$ for all such $t$. Since $f(1)=0$, we have $|\chi_2(u)|=1$ iff $\Real u=1$. For $|t|<m_1/2$ no monotonicity claim is needed in this paper; the corresponding range is covered by the certified base band in Appendix~\ref{app:firstheight-certified}.
\end{theorem}

\paragraph{(Interpretive; non-load-bearing) $\Omega$-continuum and ray invariance.}
Let $\Omega(z)=z/|z|$ forget scale. FE-symmetric dilations $T_\lambda(u)=1+\lambda(u-1)$ preserve rays; $\tan\theta=\Imag v/\Real v=m/a$. At a nontrivial zero $a=0$, the ray is vertical. This layer is contextual only; the proofs below do not use it.

% ---------------------------------------------------
\section{Boxes, de-singularization, residual control, and forcing}\label{sec:boxes}
% ---------------------------------------------------

Fix $m\ge 10$, $\alpha\in(0,1]$, and
\begin{equation}\label{eq:box-delta}
B(\alpha,m,\delta)=\big[\alpha-\delta,\alpha+\delta\big]\times\big[m-\delta,m+\delta\big],
\qquad
\delta=\frac{\eta\,\alpha}{(\log m)^2},\ \ \eta\in(0,1).
\end{equation}

\paragraph{Why $m\ge 10$.}
This ensures uniform applicability of the vertical-strip digamma bounds (DLMF §5.11) and of the $\zeta'/\zeta$ expansions on $1/2\le\sigma\le1,\ t\ge 3$ (Titchmarsh §14; Ivi\'c Ch.~9) after width-2 transport (since $u=2s$ doubles ordinates, $t\ge3$ corresponds to $m\ge 6$; we take $m\ge10$ for margin).

\paragraph{Why $\delta=\eta\alpha/(\log m)^2$.}
This balances the scale-free forcing ($\ge\pi/2$) against residual budgets $O(\delta\log m)$ and yields an $L^2$ + harmonic-measure upper envelope (in Section~\ref{sec:tail}) that is uniformly small in $\alpha$.

\begin{lemma}[Short boxes stay in $\Real v>0$]\label{lem:box-right}
For $m\ge10$ and any $\eta\in(0,1)$, one has $\delta<\alpha$ and $B(\alpha,m,\delta)\subset\{\Real v>0\}$, uniformly in $\alpha\in(0,1]$.
\end{lemma}
\begin{proof}
Since $\eta\in(0,1)$ and $\log m\ge\log 10>0$, we have $\eta/(\log m)^2<1$, hence $\delta=\alpha\,\eta/(\log m)^2<\alpha$. Therefore the left edge is at $\alpha-\delta>0$, so the entire box lies strictly in $\{\Real v>0\}$.
\end{proof}

\paragraph{De-singularization on $\partial B$.}
Let
\begin{equation}\label{eq:Zloc}
\Zloc(v)=\prod_{\rho:\,|\Imag\rho-m|\le 1}(v-\rho)^{m_\rho},\qquad
F(v):=\frac{E(v)}{\Zloc(v)}.
\end{equation}
Then $F$ is analytic and zero-free on a neighborhood of $\partial B$ (all local zeros/poles within $|\Imag\rho-m|\le 1$ have been removed).

\paragraph{Boundary contact convention.}
If a zero/pole meets $\partial B$, shrink $\delta$ by a factor $1-\varepsilon$ or shift $\alpha$ by $O(\delta)$. All constants/inequalities below (residual envelope, short-side forcing) are stable under $O(\delta)$ changes.

\begin{lemma}[Residual envelope]\label{lem:residual}
On $\partial B$,
\begin{equation}\label{eq:residual-sup}
\sup_{\partial B}\Big|\frac{F'}{F}\Big|\ \le\ C_1\log m + C_2,
\end{equation}
and
\begin{equation}\label{eq:residual-perimeter}
\big|\Delta_{\partial B}\arg F\big|\ \le\ 8\delta\,\big(C_1\log m+C_2\big).
\end{equation}
\emph{Justification.} DLMF §5.11 controls $\psi$ on vertical strips; Titchmarsh §14 and Ivi\'c Ch.~9 control $\zeta'/\zeta$ on $1/2\le\sigma\le 1,\ t\ge 3$. After removing local poles via \eqref{eq:Zloc} and transporting to width-2, we obtain \eqref{eq:residual-sup}. For \eqref{eq:residual-perimeter}, write $\Delta_{\partial B}\arg F=\int_{\partial B}\partial_\tau\arg F\,ds$ as the sum of side integrals (angular limits at the corners); then bound by $|\partial B|\,\sup_{\partial B}|F'/F|=8\delta\,\sup|F'/F|$. The constants $C_1,C_2>0$ are absolute; we keep them symbolic (see Appendix~\ref{app:S2} for an optional instantiation).
\end{lemma}

\begin{lemma}[Logarithmic derivatives on $\partial B$]\label{lem:bridge-logs}
On $\partial B$,
\[
\frac{E'}{E}=\frac{F'}{F}+\frac{(Z_{\rm loc})'}{Z_{\rm loc}},\qquad
\sup_{\partial B}\Big|\frac{E'}{E}\Big|
\ \le\ \sup_{\partial B}\Big|\frac{F'}{F}\Big|+\sum_{\rho:\,|\Imag\rho-m|\le 1}\ \sup_{v\in\partial B}\frac{m_\rho}{|v-\rho|}\,.
\]
In particular, by the boundary-contact convention the right-hand side is finite.
\end{lemma}

\begin{lemma}[Short-side forcing]\label{lem:short-side}
Let $Z_{\rm pair}(v)=(v-(a+\ii m))(v-(-a+\ii m))$. On the near vertical
\[
I_+=\{\alpha+\ii y:\ |y-m|\le \delta\},\quad\text{with }|\alpha-a|\le\delta,
\]
one has
\begin{equation}\label{eq:short-side}
\Delta_{I_+}\arg Z_{\rm pair}
=2\arctan\frac{\delta}{|\alpha-a|}+2\arctan\frac{\delta}{\alpha+a}\ \ge\ \frac{\pi}{2}.
\end{equation}
\end{lemma}

% ---------------------------------------------------
\section{Boundary-only criteria, bridges, and corner interpolation}\label{sec:criteria}
% ---------------------------------------------------

\subsection{Two-point Schur/outer criterion (boundary-only)}\label{subsec:schur-criterion}

Let $\varphi:\D\to B$ be a conformal bijection with $\varphi(0)$ the box center and with the boundary map avoiding corners at the two marked points. Define
\begin{equation}\label{eq:schur-def}
G(v):=\frac{E(1+v)}{E(1-v)},\qquad \Phi:=(G/H)\circ\varphi,
\end{equation}
where $H$ is an \emph{outer majorant} for $G$ on $B$: choose $M\in C(\partial B)$ with $M\ge |G|$ a.e.\ on $\partial B$, let $U$ solve the Dirichlet problem on $B$ with boundary data $\log M$, fix a harmonic conjugate $V$ by an anchor, and set $H=e^{U+\ii V}$. Then $H$ is analytic and zero-free on $B$ with nontangential boundary limits $|H|=M$ a.e.; moreover $\Phi\in H^\infty(\D)$ with $\|\Phi\|_\infty\le 1$ (Duren~\cite[§II.5]{DurenHp}; Garnett~\cite[§II.2]{GarnettBAF}).

\begin{proposition}[Two-point Schur pinning]\label{prop:schur-pin}
Let $\Phi=(G/H)\circ\varphi\in H^\infty(\D)$ as above, $\|\Phi\|_\infty\le 1$. Suppose two non-corner boundary points $\zeta_\pm\in\partial\D$ have nontangential limits with $|\Phi(\zeta_\pm)|=1$, and there exists a boundary arc $A\subset\partial\D$ of positive measure on which $\operatorname*{ess\,sup}_{A}|\Phi|\le 1-\varepsilon$ for some $\varepsilon>0$. Then the angular derivatives of $\Phi$ exist at $\zeta_\pm$ (Julia--Carath\'eodory), and for any interior point $z\in\D$ with harmonic measure $\omega_z(A)\ge\omega_*>0$ one has
\[
|\Phi(z)|\ \le\ 1-\kappa,\qquad \kappa=\kappa(\varepsilon,\omega_*)>0.
\]
Consequently, for $v=\varphi(z)$ one obtains $|G(v)|\le (1-\kappa)\,|H(v)|$.
\end{proposition}

\begin{lemma}[Two-point link for $|G|$ and $|\chi_2|$]\label{lem:G-chi-link}
For $v=a+\ii m$ one has
\begin{equation}\label{eq:G-chi-link}
|G(v)|=\big|\chi_2(1+v)\big|\cdot R(v),\qquad R(-v)=R(v)^{-1},
\end{equation}
hence
\begin{equation}\label{eq:G-chi-product}
|G(a+\ii m)|\,|G(-a+\ii m)|
=\big|\chi_2(1+a+\ii m)\big|\,\big|\chi_2(1-a+\ii m)\big|.
\end{equation}
Here
\[
R(v)=\pi^{-a}\left|\frac{\Gamma\!\Big(\frac{2+v}{4}\Big)}{\Gamma\!\Big(\frac{2-v}{4}\Big)}\right|
\left|\frac{\zeta\!\big(1+\tfrac{v}{2}\big)}{\zeta\!\big(1-\tfrac{v}{2}\big)}\right|,
\qquad R(-v)=R(v)^{-1}.
\]
\end{lemma}

\subsection{Outer/Rouch\'e Certification Path}\label{subsec:rouche-criterion}

Let $U$ be the harmonic solution to the Dirichlet problem on $B$ with boundary data $\log|E|$, and let $V$ be a harmonic conjugate fixed by an anchor. Set
\[
\Gout:=e^{U+\ii V}.
\]
Then $\Gout$ is analytic and zero-free on $B$ and satisfies $|\Gout|=|E|$ nontangentially on $\partial B$ (a.e.). Existence/uniqueness (up to unimodular constant) follows from the Dirichlet solution and harmonic conjugation in simply connected domains; see Duren~\cite[§II.5]{DurenHp} and Garnett~\cite[§II.2]{GarnettBAF}.

\begin{proposition}[Outer/Rouch\'e criterion]\label{prop:rouche-criterion}
If
\begin{equation}\label{eq:rouche-ratio}
\sup_{v\in\partial B}\frac{|E(v)-\Gout(v)|}{|\Gout(v)|}\ <\ 1,
\end{equation}
then $E$ is zero-free in $B$ (Rouch\'e's theorem; Ahlfors~\cite[§§5--6]{Ahlfors}, Conway~\cite[Ch.~VI]{Conway}). Consequently the inner quotient $W:=E/\Gout$ is analytic and nonvanishing on $B$ with $|W|=1$ a.e.\ on $\partial B$.
\end{proposition}

\begin{proposition}[Bridge~1: inner collapse]\label{prop:bridge1}
Under \eqref{eq:rouche-ratio}, $\log|W|$ is harmonic with zero boundary trace on $B$, hence $|W|\equiv 1$ on $B$. By the open mapping theorem, $W\equiv e^{\ii\theta_B}$ on $B$ for some real constant $\theta_B$.
\end{proposition}

\begin{proposition}[Bridge~2: stitching]\label{prop:bridge2}
If $B_1,B_2$ overlap and $W\equiv e^{\ii\theta_{B_j}}$ on $B_j$ $(j=1,2)$, then $e^{\ii\theta_{B_1}}=e^{\ii\theta_{B_2}}$ on $B_1\cap B_2$ by analyticity. Hence a band tiled by certified boxes inherits a single unimodular phase.
\end{proposition}

\begin{remark}[Certification recipe and reproducibility]
The verification of \eqref{eq:rouche-ratio} is performed by a rigorous pipeline (Appendix~\ref{app:cert}):
(i) interval enclosures for $|E|$ and $\arg E$ on $\partial B$; (ii) a validated Poisson solver on $\D$ to reconstruct $U=\log|\Gout|$ and transport to $B$; (iii) an interval reconstruction of $\arg\Gout$; and (iv) a grid$\to$continuum Lipschitz enclosure using $\sup_{\partial B}|E'/E|$ (Lemma~\ref{lem:residual}). Appendix~\ref{app:cert} also pins libraries (e.g., Arb), precisions, and boundary meshes to ensure reproducibility.
\end{remark}

\subsection{Corner outer interpolation (two-point)}\label{subsec:corner-interp}

\begin{theorem}[Corner outer interpolation]\label{thm:corner-outer}
Let $G$ be analytic in a neighborhood of $\overline B$. Let $h\in C(\partial B)$ satisfy $h\ge 0$ and $h\equiv 0$ on small boundary arcs containing the two top corners $C_\pm$. Let $H=e^{U+\ii V}$ be the outer on $B$ with $U|\_{\partial B}=\log|G|+h$. Then the nontangential limits at $C_\pm$ exist and
\[
|H(C_\pm)|=|G(C_\pm)|.
\]
\end{theorem}

\begin{remark}[Two “outers”: roles and notation]
We reserve $H$ for an \emph{outer majorant} attached to an arbitrary analytic datum $G$ on $B$ (used in the Schur pinning), and $\Gout$ for the \emph{modulus-outer} attached to $E$ via the boundary data $\log|E|$ (used in the Rouch\'e route). Both are analytic, zero-free, and determined up to a unimodular factor; their roles are distinct.
\end{remark}

% ===================================================
\section{Analytic tail (uniform in \texorpdfstring{$\alpha$}{alpha})}\label{sec:tail}
% ===================================================

\paragraph{Setup and notation.}
Let $\varphi:\D\to B(\alpha,m,\delta)$ be a conformal bijection with $\varphi(0)=\alpha+\ii m$; define the \emph{dial pair} on the horizontal centerline by
\[
v_\pm^\star=\pm(a+\ii m).
\]
Split the boundary $\partial B$ into the two \emph{quiet arcs} $Q$ (horizontal edges) and the two short vertical sides $I_\pm$.
Write
\[
W:=\frac{E}{\Gout}.
\]
We write $\partial_\tau$ for the unit tangential derivative along $\partial B$. All boundary integrals are taken with respect to arclength $ds$; the perimeter is $|\partial B|=8\delta$.
For the left dial $-a+\ii m$, we either work in the reflected coordinate $w=-v$ with a box centered at $\alpha=a>0$, or equivalently use the reflected aligned box (shape-only constants are unaffected).

% ---------------------------------------------------
\subsection{Upper envelope via a disc-based $L^2$ route}\label{subsec:upper}
% ---------------------------------------------------

\begin{lemma}[Boundary phase $\Rightarrow$ dial deficit; disc-based upper bound]\label{lem:upper-disc}
Let $m\ge 10$ and $\delta=\eta\,\alpha/(\log m)^2$. Let $W=E/\Gout$ be analytic on $B(\alpha,m,\delta)$ with $|W|=1$ a.e.\ on $\partial B$, and assume $v_\pm^\star\in B$ (as in the aligned boxes $\alpha=\pm a$). For each such dial $v_\pm^\star$ on the horizontal centerline, there exists a shape-only constant $C_{\mathrm{up}}>0$ such that
\begin{equation}\label{eq:upper-disc-point}
\big|W(v_\pm^\star)-e^{\ii\phi_0^\pm}\big|
\ \le\ C_{\mathrm{up}}\ \delta^{3/2}\ \Big(\sup_{\partial B}\Big|\frac{E'}{E}\Big|\Big),
\end{equation}
where $\phi_0^\pm$ is the harmonic-measure average of $\arg W$ seen from $v_\pm^\star$. Consequently,
\begin{equation}\label{eq:Uhm-upper-disc}
\sum_{\pm}\big|W(v_\pm^\star)-e^{\ii\phi_0^\pm}\big|
\ \le\ 2\,C_{\mathrm{up}}\ \delta^{3/2}\ \Big(\sup_{\partial B}\Big|\frac{E'}{E}\Big|\Big),
\end{equation}
where the sum is obtained by applying \eqref{eq:upper-disc-point} separately on the two aligned boxes (right and left; or in $v$ and $w=-v$ with the same $\alpha=a$) and adding the bounds.
Moreover,
\begin{equation}\label{eq:Cup-def}
C_{\mathrm{up}}\ =\ C_{\rm tr}\,C_{\mathrm H}\cdot \frac{8\sqrt{8}}{\pi},
\end{equation}
with $C_{\rm tr}$ the $L^2$ conformal trace constant and $C_{\mathrm H}$ the $L^2$ norm of the boundary Hilbert/conjugation on $\partial B$ (both shape-only; see Appendix~\ref{app:S1}).
\end{lemma}

\begin{remark}[Branch and trace conventions]
Since $|W|=1$ a.e.\ on $\partial B$, choose any measurable branch of $\arg W$ on $\partial B$; $\phi_0^\pm$ is defined as the harmonic-measure average seen from $v_\pm^\star$. The bounds are invariant under $2\pi\mathbb Z$ shifts of the branch.
\end{remark}

% ---------------------------------------------------
\subsection{Lower envelope via forcing, $L^2$ allocation, and Jensen}\label{subsec:lower-new}
% ---------------------------------------------------

We quantify how much of the vertical phase gap can be lost to the tails and horizontals, then force a zero in a dial-centred core via a restricted contour, and finally convert that zero into a dial-deficit by Jensen.

\begin{lemma}[Vertical Lipschitz allocation ($L^2$)]\label{lem:allocL2}
Let $\lambda\in(0,1)$, and let $s_{\rm tail}=(2-\lambda)\delta$ be the total tail length on a vertical side (outside the central sub-arc of length $\lambda\delta$). Then on each vertical side
\begin{equation}\label{eq:alloc-one}
\int_{\textup{tails}} \big|\partial_\tau \arg W\big|\,ds
\ \le\ \Big[(2-\lambda)+2\sqrt{2(2-\lambda)}\Big]\,\delta\,\sup_{\partial B}\Big|\frac{E'}{E}\Big|.
\end{equation}
Summing both verticals yields
\begin{equation}\label{eq:alloc-two}
\Delta_{\rm cent}\ \ge\ \Delta_{\rm vert}\ -\ K_{\rm alloc}(\lambda)\,\delta\,\sup_{\partial B}\Big|\frac{E'}{E}\Big|,
\quad
K_{\rm alloc}(\lambda):=2\Big[(2-\lambda)+2\sqrt{2(2-\lambda)}\Big].
\end{equation}
For conservatism we may adopt $K_{\rm alloc}^{\star}(\lambda):=2\big[(2-\lambda)+4\sqrt{2(2-\lambda)}\big]$.
\end{lemma}

\noindent\textit{Retained central gap.} Under $|\alpha-a|\le\delta$ and $\Real v>0$, the near/far vertical forcing gives $\Delta_{\rm vert}\ge \pi/2$ (Lemma~\ref{lem:short-side}). We set
\begin{equation}\label{eq:Delta-cent-ineq}
\Delta_{\rm cent}\ :=\ \Delta_{\rm vert}\ -\ K_{\rm alloc}^{\star}(\lambda)\,\delta\,\sup_{\partial B}\Big|\frac{E'}{E}\Big| \ -\ C_h''\,\delta\,(\log m+1),
\end{equation}
where $C_h''>0$ is a shape-only constant accounting for the horizontal (quiet-arc) budget (Appendix~\ref{app:S1}).

\begin{lemma}[Core zero via restricted contour]\label{lem:corezero}
Align the box by taking $\alpha=a$. Let $\Gamma_\lambda$ be the union of the two central sub-arcs (length $\lambda\delta$) on the vertical sides, joined by vanishing horizontals at heights $m\pm\varepsilon$ as $\varepsilon\downarrow 0$. If $\Delta_{\rm cent}>0$ in the sense of \eqref{eq:Delta-cent-ineq}, then the rectangle bounded by $\Gamma_\lambda$ contains at least one zero of $W$. This zero lies in the dial-centred core
\[
B_{\rm core}(a,m;\lambda)=\big[a-\tfrac{\lambda\delta}{2},a+\tfrac{\lambda\delta}{2}\big]\times \big[m-\tfrac{\lambda\delta}{2},m+\tfrac{\lambda\delta}{2}\big].
\]
The tiny horizontal joins contribute $o(1)$ to the argument change and are absorbed in the horizontal budget.
\end{lemma}

\begin{lemma}[Jensen at the dial]\label{lem:jensen-dial}
With $\alpha=a$, fix one dial $p=a+\ii m$. Then ${\rm dist}(p,\partial B)=\delta$ so $D_p=\{|z-p|<\delta\}\subset B$. If $W$ has a zero $z_k$ in $B_{\rm core}(a,m;\lambda)$, then
\[
-\log|W(p)|\ \ge\ \log\!\Big(\frac{\delta}{|z_k-p|}\Big)\ \ge\ \log\!\Big(\frac{\sqrt{2}}{\lambda}\Big),
\]
hence
\begin{equation}\label{eq:jensen-dial-const}
1-|W(p)|\ \ge\ 1-\frac{\lambda}{\sqrt{2}}.
\end{equation}
\end{lemma}

\begin{lemma}[Bridge to the upper-envelope metric]\label{lem:bridge-metric}
For any unimodular $c=e^{\ii\phi}$ and any $z\in B$, one has $|W(z)-c|\ge 1-|W(z)|$.
\end{lemma}

\begin{corollary}[Lower envelope; aligned boxes]\label{cor:lower}
Pick $\lambda=\tfrac12$ and denote $c_0=\frac{1}{4\pi}\log(2\sqrt{2})$. With $L=\sup_{\partial B}|E'/E|$ and $\delta=\eta\,\alpha/(\log m)^2$,
\[
\varepsilon_+ + \varepsilon_- \ \ge\ c_0\,\frac{\pi}{2}\ -\ \delta\Big( K_{\rm alloc}^{\star}(\tfrac12)\,c_0\,L + C_h''(\log m+1) \Big),
\]
where $K_{\rm alloc}^{\star}(\tfrac12)=3+8\sqrt{3}$ and $C_h''>0$ is shape-only.
\end{corollary}

% ---------------------------------------------------
\subsection{Tail comparison (symbolic constants)}\label{subsec:comparison}
% ---------------------------------------------------

\begin{theorem}[Global on-axis theorem; symbolic constants]\label{thm:tail-symbolic}
Fix $\eta\in(0,1)$ and set $\delta=\eta\,\alpha/(\log m)^2$. Let $C_{\mathrm{up}}>0$ be the shape-only constant in Lemma~\ref{lem:upper-disc}, $C_h''>0$ the horizontal budget constant in Lemma~\ref{lem:corezero}, and $K_{\rm alloc}^{\star}(\tfrac12)=3+8\sqrt{3}$. Assume Lemma~\ref{lem:residual} with constants $C_1,C_2>0$. Then there exists $M_0(\eta)$ such that, for all $m\ge M_0(\eta)$ and all $\alpha\in(0,1]$,
\begin{equation}\label{eq:upper-lower-compare}
\underbrace{\sum_{\pm}\big|W(v_\pm^\star)-e^{\ii\phi_0^\pm}\big|}_{\mathcal U_{hm}(m,\alpha)}
\ <\
\underbrace{c_0\,\frac{\pi}{2}\ -\ \delta\Big( K_{\rm alloc}^{\star}(\tfrac12)\,c_0\,(C_1\log m+C_2) + C_h''(\log m+1) \Big)}_{\mathcal L(m,\alpha)}\,.
\end{equation}
Consequently, no off-axis quartet lies in any $B(\alpha,m,\delta)$ for $m\ge M_0(\eta)$ and all $\alpha\in(0,1]$. Combined with a certified base range “no zeros below $m_1$” (Appendix~\ref{app:firstheight-certified}) and, when $M_0(\eta)>m_1$, certification of the finite band $[m_1,M_0(\eta)]$ via the Outer/Rouch\'e pipeline (Section~\ref{sec:criteria} and Appendix~\ref{app:cert}), all nontrivial zeros lie on $\Real s=\tfrac12$.
\end{theorem}

\paragraph{Choice of $M_0(\eta)$ (explicit criterion).}
A sufficient (symbolic) condition ensuring \eqref{eq:upper-lower-compare} for all $\alpha\in(0,1]$ is
\begin{equation}\label{eq:M0-criterion}
2\,C_{\mathrm{up}}\left(\frac{\eta}{(\log m)^2}\right)^{\!3/2}\!\!(C_1\log m+C_2)\ \le\ \tfrac12\left(c_0\frac{\pi}{2}-\frac{\eta}{(\log m)^2}\Big(K_{\rm alloc}^{\star}(\tfrac12)\,c_0\,(C_1\log m+C_2)+C_h''(\log m+1)\Big)\right).
\end{equation}
Since the left side is $o(1)$ and the right side $\to c_0\pi/4>0$ as $m\to\infty$, there exists $M_0(\eta)$ with \eqref{eq:M0-criterion} holding for all $m\ge M_0(\eta)$.

\begin{remark}[Numerical check; illustrative only]
If one instantiates $(C_1,C_2)$ safely from the literature (Appendix~\ref{app:S2}) and takes a small $\eta$ (e.g., $\eta=10^{-9}$), then at $m=m_1$ and $\alpha=1$ the upper bound is $\ll 10^{-10}$ while the lower bound is $\approx 0.13$ (up to $O(10^{-8})$), leaving an overwhelming margin. These numerics are not used in the proof.
\end{remark}

% ======================================================================
% Part III — Structural Corollaries (post-Theorem; brief proofs)
% ======================================================================
\section*{Part III --- Structural Corollaries (after the main theorem)}
\addcontentsline{toc}{section}{Part III --- Structural Corollaries (after the main theorem)}

\paragraph{Standing assumption for this part.}
Assume the \emph{Main Theorem (Part~II)}: for every nontrivial height $m>0$, the per--height tilt satisfies $a(m)=0$.

\medskip

\begin{corollary}[Canonical columns]\label{cor:canonical-columns}
Define $\Podd(n)=(1-\cos\pi n)/2$ and $\Peven(n)=(1+\cos\pi n)/2$. Let $k:\Z\to\Z$ be the odd--indexer $k(2j-1)=j$, $k(2j)=j+1$ (e.g.\ $k(n)=\tfrac{n}{2}+\tfrac{1-\cos\pi n}{4}$). For any real $x\in(0,2)$ set
\[
\UR(x,n)=\Podd(n)\,\big(x+\ii\,m_{k(n)}\big)\;-\;4\big(n+1-k(n)\big)\,\Peven(n),\qquad
\UL(x,n)=\Podd(n)\,\big(2-x+\ii\,m_{k(n)}\big)\;-\;4\big(n+1-k(n)\big)\,\Peven(n).
\]
Under $a(m)=0$ at each nontrivial height, the canonical choice $x=1$ yields $\UR(1,n)=\UL(1,n)$ for all $n\in\Z$.
\end{corollary}

\begin{corollary}[Collapsed canonical stream: mod--4 face]\label{cor:collapsed-mod4}
Define the stream
\[
\Ucore(n):=\Podd(n)\,\big(1+\ii\,m_{k(n)}\big)\;-\;4\big(n+1-k(n)\big)\,\Peven(n).
\]
Then $\Ucore(2j-1)=1+\ii m_j$ and $\Ucore(2j)=-4(j+1)$ for all $j\in\Z$.
\end{corollary}

\begin{corollary}[Collapsed canonical stream: trigonometric face]\label{cor:collapsed-mod2}
Using $\sin^2(\pi n/2)=\Podd(n)$ and $\cos^2(\pi n/2)=\Peven(n)$,
\[
\Ucore(n)=\sin^2\!\Big(\frac{\pi n}{2}\Big)\,\big(1+\ii\,m_{k(n)}\big)\;-\;4\big(n+1-k(n)\big)\,\cos^2\!\Big(\frac{\pi n}{2}\Big).
\]
\end{corollary}

\begin{corollary}[Single--frequency collapse]\label{cor:single-frequency}
There exist functions $c(n),d(n)$ such that
\[
\Ucore(n)=(c+d)\;+\;(c-d)\,\cos(\pi n),\qquad
c=2\big(k(n)-n-1\big),\quad d=\frac{1+\ii\,m_{k(n)}}{2}.
\]
\end{corollary}

\begin{corollary}[Self--indexed recurrence]\label{cor:self-indexed}
With initial values $\Ucore(0)=-4$ and $\Ucore(1)=1+\ii m_1$, for all $n\ge2$,
\[
\Ucore(n)=\Podd(n)\,\Big(1+\ii\,m_{-\Ucore(n-1)/4}\Big)\;-\;\Peven(n)\,\Big(\Ucore(n-2)+4(n+1)\Big).
\]
\end{corollary}

\begin{corollary}[Seed $\to$ rectifier $\to$ physical streams]\label{cor:rectifier}
Let $\chi_4(n):=(-1)^{\lfloor n/2\rfloor}$ and define, for $f>0$ and gain $\lambda\in\R$,
\[
s_{f,k}(n)=f\lambda\Big[\sin\!\Big(\frac{\pi n}{2}\Big)\big(1+\ii\,m_k\big)-4n\,\cos\!\Big(\frac{\pi n}{2}\Big)\Big].
\]
Then $\chi_4(n)\,\sin(\pi n/2)=\Podd(n)$ and $\chi_4(n)\,\cos(\pi n/2)=\Peven(n)$, hence
\[
\chi_4(n)\,s_{f,k}(n)=f\lambda\Big[\Podd(n)\big(1+\ii\,m_k\big)-4n\,\Peven(n)\Big].
\]
Setting $\lambda=\tfrac12$ and replacing $k$ by $k(n)$ gives the physical stream $S_f(n):=\frac{f}{2}\,\Ucore(n)$.
\end{corollary}

\begin{corollary}[Curvature extractor \& $\zeta(2)$ disguise]\label{cor:curvature}
Let $F(n):=\Imag \Ucore(n)$. Then $F(2j-1)=m_j$, $F(2j)=0$, and
\[
m_j=\frac{2}{\pi^2}\,\Imag\big(\Ucore''(2j)\big)
=\frac{1}{3\,\zeta(2)}\,\Imag\big(\Ucore''(2j)\big)
=\frac{2}{3\,\zeta(2)}\sum_{\ell\in\Z}\frac{m_\ell}{\big(2(j-\ell)+1\big)^2}.
\]
For the discrete second difference $\Delta^2 U(n):=U(n+1)-2U(n)+U(n-1)$, one also has  
$\Imag\Delta^2 U(2j)=m_{j+1}+m_j$.
\end{corollary}

% ----------------------------------------------------------------------
% Part III (continued) — Prime-locked corollaries and generator
% ----------------------------------------------------------------------
\section*{Part III (continued) --- Prime--Locked Corollaries and Generator}
\addcontentsline{toc}{section}{Part III (continued) --- Prime--Locked Corollaries and Generator}

\paragraph{Standing hypotheses and notation.}
Assume the Main Theorem of Part~II. Let $t_j$ be the increasing ordinates of zeros on $\Real s=\tfrac12$ (counting multiplicity), and set $m_j:=2t_j$ (width--2 ordinates). Write $\theta(t)$ for the Riemann--Siegel theta function and
\[
S(t)=\frac{1}{\pi}\arg\zeta\!\big(\tfrac12+\mathrm{i}t\big),\qquad
\theta'(t)=\frac12\log\frac{t}{2\pi}+O(t^{-1}).
\]
We use the residual envelope (Lemma~\ref{lem:residual}) and the shape--only $L^2$ boundary control (Lemmas~\ref{lem:upper-disc}, \ref{lem:allocL2}, Corollary~\ref{cor:lower}).

Fix once and for all
\begin{equation}\label{eq:PW-choices}
\varepsilon:=\tfrac12,\qquad
X_j:=(\log t_j)^{\,2-\varepsilon}=(\log t_j)^{\,3/2},
\end{equation}
and a Paley--Wiener weight $W\in C_c^\infty([0,1])$ with $0\le W\le 1$ and $\int_0^1 W(y)\,dy=1$ (see Appendix~\ref{app:PW}).

Define for $\Delta t>0$ the prime integral
\[
\mathcal P_{X_j}(t_j,\Delta t)
:=
-\sum_{n\ge1}\frac{\Lambda(n)}{\sqrt n\,\log n}\,
W\!\Big(\frac{n}{X_j}\Big)
\Big[\sin\!\big((t_j+\Delta t)\log n\big)-\sin\!\big(t_j\log n\big)\Big].
\]

\begin{corollary}[C1: Two--tick prime--locked quantization]\label{cor:C1}
Let $\Delta t_j:=t_{j+1}-t_j$. Then
\begin{equation}\label{eq:C1}
\theta(t_{j+1})-\theta(t_j)\;+\;\mathcal P_{X_j}(t_j,\Delta t_j)\;=\;\pi\;+\;\mathcal E_j\,,
\end{equation}
with the explicit bound
\begin{equation}\label{eq:C1E}
|\mathcal E_j|\ \le\ \frac{A_\theta}{t_j}\ +\ \frac{A_W}{\sqrt{X_j}}\ +\ \frac{A_{\rm loc}}{(\log m_j)^{2}}\,.
\end{equation}
Here $A_\theta>0$ is absolute (from $\theta''(t)=O(1/t)$), $A_W>0$ depends only on $W$,
and the local term
\[
A_{\rm loc}=A_{\rm loc}\!\big(\eta;C_1,C_2,C_{\rm tr},C_{\mathrm H},C_h'',K^{\star}_{\rm alloc}\big)
\]
depends only on the Part~II constants.
\end{corollary}

\begin{corollary}[C2: Prime--modulated first--order gap]\label{cor:C2}
Let $t_\ast:=t_j+\tfrac12\Delta t_j$ and $m_\ast:=2t_\ast$. Then
\begin{equation}\label{eq:C2}
\Delta m_j\ =\ \frac{4\pi}{
\ \theta'(t_\ast)\ -\ \displaystyle\sum_{n\ge1}\frac{\Lambda(n)}{\sqrt n}\,
W\!\Big(\frac{n}{X_j}\Big)\cos\!\big(t_\ast\log n\big)}\ +\ R_j\,,
\end{equation}
with
\begin{equation}\label{eq:C2R}
|R_j|\ \le\ \frac{B_\theta}{t_j(\log m_j)^{2}}\ +\ \frac{B_W\,(\log X_j)^{2}}{(\log m_j)^{3}}\sqrt{X_j}\ +\ \frac{B_{\rm loc}}{(\log m_j)^{2}}\,.
\end{equation}
Here $B_\theta>0$ is absolute, $B_W>0$ depends only on $W$, and $B_{\rm loc}$ depends only on the Part~II constants.
\end{corollary}

\begin{corollary}[C3: Even--site curvature $\leftrightarrow$ prime update]\label{cor:C3}
Recall $\Imag\Delta^2U(2j)=m_{j+1}+m_j$ (Corollary~\ref{cor:curvature}). For any $J\ge1$,
\[
\frac{1}{J}\sum_{r=0}^{J-1}\Big(\Imag\Delta^2 U(2(j+r)) - 2m_{j+r}\Big)
=\ \frac{1}{J}\sum_{r=0}^{J-1}\big(m_{j+r+1}-m_{j+r}\big).
\]
Substituting $\Delta m_{k}$ from \eqref{eq:C2} yields a block--averaged, explicit prime
formula for the even--site curvature, with total error bounded by
$\tfrac{1}{J}\sum_{r=0}^{J-1}\big(|R_{j+r}|+|R_{j+r+1}|\big)$.
\end{corollary}

\begin{corollary}[C4: Newton contraction on a polylog window]\label{cor:C4}
Let $G_{X_j}(\Delta m):=\theta\big(\tfrac{m_j+\Delta m}{2}\big)-\theta\big(\tfrac{m_j}{2}\big)
-\mathcal P_{X_j}\!\big(\tfrac{m_j}{2},\tfrac{\Delta m}{2}\big)-\pi$.
With $X_j$ as in \eqref{eq:PW-choices} there exists $j_0$ such that for all $j\ge j_0$ and all $\Delta m$
in a neighborhood of the true gap,
\[
\Big|\partial_{\Delta m}G_{X_j}\Big|
\ \ge\ \tfrac18\log t_j,\qquad
\Big|\partial^2_{\Delta m}G_{X_j}\Big|
\ \ll\ \frac{(\log X_j)^2\sqrt{X_j}}{(\log t_j)^2}
= \frac{(\log\log t_j)^2}{(\log t_j)^{\,2-\varepsilon/2}}.
\]
Hence damped Newton with any fixed $\lambda\in(0,1]$ converges in $O(1)$ steps from any initial guess within $c/\log t_j$ of the root, with contraction factor $1-\kappa/\log t_j$ for some $\kappa>0$ independent of $j$.
\end{corollary}

\begin{corollary}[C5: Canonical Weil weight and prime powers]\label{cor:C5}
Let $\phi\in C_c^\infty(\R)$ be even, $\operatorname{supp}\phi\subset[-1,1]$, $\phi(0)=1$,
and put $W=\widehat\phi\!\restriction_{[0,1]}$. Replace $\Lambda(n)$ by $\Lambda(p^k)=\log p$
at $n=p^k$ and $0$ otherwise, i.e.
\[
\mathcal P^{\rm Weil}_{X_j}(t,\Delta t)
:=
-\sum_{p^k\ge1}\frac{\Lambda(p^k)}{p^{k/2}k\log p}\,
W\!\Big(\frac{p^k}{X_j}\Big)
\Big[\sin\!\big((t+\Delta t)\,k\log p\big)-\sin\!\big(t\,k\log p\big)\Big].
\]
Then Corollaries~\ref{cor:C1} and~\ref{cor:C2} hold with $\mathcal P_{X_j}$ replaced by $\mathcal P^{\rm Weil}_{X_j}$.
\end{corollary}

\begin{theorem}[Deterministic prime--locked generator of $\{m_j\}$]\label{thm:generator}
Fix $W$ and $X_j=(\log t_j)^{3/2}$ as in \eqref{eq:PW-choices}.
Given the seed $m_1$ (Appendix~\ref{app:firstheight-certified}) and the Main Theorem (Part~II),
define $m_{j+1}$ for $j\ge1$ as the unique solution of
\begin{equation}\label{eq:generator-eqn}
\theta\!\Big(\frac{m_{j+1}}{2}\Big)-\theta\!\Big(\frac{m_j}{2}\Big)
\;+\;
\mathcal P^{\rm Weil}_{X_j}\!\Big(\frac{m_j}{2},\frac{m_{j+1}-m_j}{2}\Big)
\;=\;\pi\,.
\end{equation}
For all $j\ge j_0$ (some explicit startup index depending only on $W$), \eqref{eq:generator-eqn}
has a unique solution obtained by damped Newton in $O(1)$ steps with contraction factor
$1-\kappa/\log t_j$. Moreover,
\[
m_{j+1}-m_j\ =\ \frac{4\pi}{
\ \theta'(t_\ast)\ -\ \displaystyle\sum_{n\ge1}\frac{\Lambda(n)}{\sqrt n}\,
W\!\Big(\frac{n}{X_j}\Big)\cos\!\big(t_\ast\log n\big)}\ +\ R_j
\]
with $t_\ast=\tfrac12(m_j+m_{j+1})$ and $R_j$ bounded as in \eqref{eq:C2R}.
The finitely many indices $1\le j<j_0$ can be handled by the finite verification band of Part~II.
\end{theorem}

%------------------------------------------------------------------------------------------
% Appendices
%------------------------------------------------------------------------------------------

\appendix

\section{Hinge proof (eight-line variant)}
For completeness, one may also verify the monotonicity of $\log|\chi_2|$ via $\partial_\sigma\log|\Gamma|=\Real\psi$ and $\psi(1-z)-\psi(z)=\pi\cot(\pi z)$ directly; the cosh-bound form appears in Theorem~\ref{thm:hinge}.

\section{Constants ledger (sources \& transport)}
\begin{itemize}
\item Digamma (DLMF §5.11): $\psi(z)=\log z+O(1)$ uniformly on vertical strips; transported to width-2 gives $\Real\psi((1+v)/4)=\log|m|+O(1)$ on $\partial B$.
\item $\zeta'/\zeta$ (Titchmarsh §14; Ivi\'c Ch.~9): for $1/2\le \sigma\le 1,\ t\ge 3$,
$\displaystyle \frac{\zeta'}{\zeta}(\sigma+\ii t)=\sum_{|\Imag\rho-t|\le 1}\frac{1}{\sigma+\ii t-\rho}+O(\log t)$.
Removing local poles via $\Zloc$ yields Lemma~\ref{lem:residual}.
\item Lipschitz Hilbert/Cauchy: bounded on $L^2(\Gamma)$ for Lipschitz curves; boundary traces between $\partial\D$ and $\Gamma$ are bounded with constants depending only on the Lipschitz character (Coifman--McIntosh--Meyer).
\end{itemize}

\section{Bridges (one-liners)}
\begin{itemize}
\item Bridge~1. If \eqref{eq:rouche-ratio} holds, then $E$ and $\Gout$ have the same zero count, $\Gout$ is zero-free, $|W|=1$ on $\partial B$. Hence $\log|W|\equiv 0$, and by the open mapping theorem $W\equiv e^{\ii\theta_B}$.
\item Bridge~2. If $W_1,W_2$ are unimodular constants on overlapping boxes, they agree on overlaps, hence globally.
\end{itemize}

\section{Conformal normalization}
Take $\varphi:\D\to B(\alpha,m,\delta)$ conformal with $\varphi(0)=\alpha+\ii m$ and $\varphi(\pm 1)$ the top corners. By symmetry, $\varphi((-1,1))$ is the horizontal centerline; thus there exists a unique $r_0\in(0,1)$ with $\varphi(\pm r_0)=\pm(a+\ii m)$.

\section{Outer/Rouch\'e certification protocol (rigorous outline)}\label{app:cert}
\begin{itemize}
\item Boundary intervals. Interval bounds for $|E|$, $\arg E$ on $\partial B$ at grid size $N_{\rm side}$.
\item Validated Poisson. Interval Dirichlet solver on $\D$ for $U=\log|\,\Gout|$, with conformal push-forward to $\partial B$.
\item Phase reconstruction. Interval Hilbert on $\partial\D$, conformal trace to $\partial B$.
\item Grid$\to$continuum. Lipschitz enclosure via $\sup_{\partial B}|E'/E|$ and explicit pair terms.
\item Certificate. Check $\sup_{\partial B}|E-\Gout|/|\,\Gout|<1$.
\end{itemize}
The grid$\to$continuum step uses a shape-only Lipschitz/trace bound on $\partial B$ to convert a mesh supremum into a boundary supremum, making the Rouch\'e ratio verifiable with controlled constants.

\section{Toolbox (structural; not used in proofs)}
Catalog of auxiliary identities/filters (modulated families, ray curvature extractor). Structural and not used in Section~\ref{sec:tail} proofs.

\section{Certified first nontrivial zero}\label{app:firstheight-certified}
We cite rigorously verified computations of Platt (and Platt--Trudgian):
\begin{theorem}[Platt 2017; Platt--Trudgian 2021]
There are no nontrivial zeros of $\zeta(s)$ with $0<\Imag s<t_1$, and the first nontrivial zero occurs at
$t_1=14.134725141734693790457251983562\ldots$ (with rigorous interval bounds).
\end{theorem}
References:
D.\,J.\,Platt, \emph{Isolating some nontrivial zeros of $\zeta(s)$}, Math. Comp. 86 (2017), 2449–2467;
D.\,J.\,Platt \& T.\,S.\,Trudgian, \emph{The Riemann hypothesis is true up to $3\cdot 10^{12}$}, Bull. Lond. Math. Soc. 53 (2021), 792–797.
Set $m_1:=2t_1$.

\section{Operator norms on Lipschitz boundaries (existence and shape-only dependence)}\label{app:S1}
On a Lipschitz Jordan curve $\Gamma$ (e.g., the rectangle boundary), the boundary Hilbert transform (conjugation) defines a bounded operator on $L^2(\Gamma)$ whose norm depends only on the Lipschitz character of $\Gamma$; the Cauchy transform is likewise bounded. Conformal boundary trace maps between $\partial\D$ and $\Gamma$ are bounded in $L^2$ with operator norms depending only on the chord-arc constants of $\Gamma$. (See Coifman--McIntosh--Meyer (1982); Duren, Ch.~II; Garnett, Ch.~II.)
Moreover, on chord–arc curves (which include rectangles) harmonic measure $\omega_z$ and arclength $ds$ are $A_\infty$-equivalent; the associated $L^2$-comparability constants depend only on the chord–arc data. We fold these shape-only constants into $C_{\rm tr}$ and into the boundary Hilbert norm $C_{\mathrm H}$ used in Lemma~\ref{lem:upper-disc}.
Since $B(\alpha,m,\delta)$ normalizes to the unit square via an affine map, all such operator norms are shape-only constants (independent of $m,\alpha,a$). We denote by $C_{\rm tr}$ a generic shape-only trace constant and by $C_{\mathrm H}$ the $L^2$ operator norm of boundary Hilbert/conjugation on $\partial B$.

\section{Instantiating $(C_1,C_2)$ from explicit literature bounds (optional)}\label{app:S2}
Let $F=E/Z_{\rm loc}$ with $Z_{\rm loc}$ removing local zeros with $|\Imag\rho-m|\le 1$. On $1/2\le\sigma\le 1$ and $t\ge 3$,
\[
\frac{\zeta'}{\zeta}(\sigma+\ii t)=\sum_{|\Imag\rho-t|\le 1}\frac{1}{\sigma+\ii t-\rho}+O(\log t)
\]
(Titchmarsh §14; Ivi\'c Ch.~9), and on vertical strips $\psi$ satisfies $\Real\psi(x+\ii y)=\log\sqrt{x^2+y^2}+O(1)$ (DLMF §5.11). Transporting to width~2 and dividing out $Z_{\rm loc}$ yields
\[
\sup_{\partial B}\Big|\frac{F'}{F}\Big|\ \le\ C_1\log m + C_2,
\]
with absolute constants $C_1,C_2>0$; any choices respecting the cited explicit estimates are legitimate. On $\partial B$ we have $\frac{E'}{E}=\frac{F'}{F}+\frac{(Z_{\rm loc})'}{Z_{\rm loc}}$ (Lemma~\ref{lem:bridge-logs}); the local sum is finite under the boundary‑contact convention, so $L=\sup_{\partial B}|E'/E|$ is controlled by the residual bound plus finitely many explicit local terms.
Given any such $(C_1,C_2)$ and a fixed $\eta\in(0,1)$, one may select $M_0(\eta)$ by enforcing the symbolic inequality \eqref{eq:M0-criterion}, which depends only on $(C_{\mathrm{up}},C_h'',K_{\rm alloc}^{\star},c_0)$ (shape-only) and $(C_1,C_2)$ (residual).

\section{A concrete Paley--Wiener weight and benign constants}\label{app:PW}
Let $\eta\in C^\infty(\R)$ be the standard bump
\[
\eta(y)=\begin{cases}\exp\!\big(-1/(y(1-y))\big),&y\in(0,1),\\[2pt]0,&\text{elsewhere,}\end{cases}
\]
and set $W(y):=c_W\,\eta(y)$ on $[0,1]$ with $c_W:=\big(\int_0^1 \eta\big)^{-1}$ so that $\int_0^1W=1$ and $0\le W\le c_W$.
Then $c_W<\infty$ is an absolute number (numerically $c_W\approx 1.28$). With this choice:
\begin{itemize}
\item (Chebyshev bound) For all $X\ge 16$,
\[
\sum_{n\le X}\frac{\Lambda(n)}{\sqrt n}\,W\!\Big(\frac{n}{X}\Big)
\ \le\ c_W\,\sum_{n\le X}\frac{\Lambda(n)}{\sqrt n}
\ \le\ 2c_W\,\sqrt X\ .
\]
Thus in Cor.~\ref{cor:C1} we may take $A_W:=2c_W$.
\item (Cubic sinusoid remainder) In Cor.~\ref{cor:C2}, since $\log n\le\log X$ and
$\sum_{n\le X}\Lambda(n)/\sqrt n\ll\sqrt X$, we may take
$B_W:=8c_W$ in \eqref{eq:C2R}:
\[
\frac{B_W\,(\log X)^2}{(\log m)^3}\sqrt X
\ \text{dominates }\ \sum_{n\le X}\frac{\Lambda(n)}{\sqrt n}\Big(\tfrac{\Delta t}{2}\log n\Big)^{\!3}.
\]
\item (Archimedean curvature) Using $\theta''(t)= \tfrac1{2t}+O(t^{-3})$,
we may set $A_\theta:=1$ and $B_\theta:=1$ for all $t\ge 14$.
\item (Local term) The constants $A_{\rm loc},B_{\rm loc}$ are explicit functions of the Part~II constants
$\eta; C_1,C_2,C_{\rm tr},C_{\mathrm H},C_h'',K^{\star}_{\rm alloc}$ via Lemmas~\ref{lem:residual}, \ref{lem:upper-disc}, \ref{lem:allocL2}, and Corollary~\ref{cor:lower}. They are independent of $j$.
\end{itemize}
With $\eta\in(0,1)$ and the fixed $W$ above, the generator (Theorem~\ref{thm:generator}) is fully specified without any free numerical tuning.

% -----------------------------------------------------------------------------------------
% Bibliography
% -----------------------------------------------------------------------------------------

\begin{thebibliography}{99}

\bibitem{Ahlfors}
L.~V.~Ahlfors, \emph{Complex Analysis}, 3rd ed., McGraw--Hill, 1979.

\bibitem{AxlerBourdonRamey}
S.~Axler, P.~Bourdon, and W.~Ramey, \emph{Harmonic Function Theory}, 2nd ed., Springer, 2001.

\bibitem{CoifmanMcIntoshMeyer}
R.~R.~Coifman, A.~McIntosh, and Y.~Meyer,
L’int\'egrale de Cauchy d\'efinit un op\'erateur born\'e sur $L^2$ pour les courbes lipschitziennes,
\emph{Ann. of Math.} \textbf{116} (1982), 361--387.

\bibitem{Conway}
J.~B.~Conway, \emph{Functions of One Complex Variable}, 2nd ed., Springer, 1978.

\bibitem{DLMF}
NIST Digital Library of Mathematical Functions, \S5.5 (Digamma reflection), \S5.11 (vertical--strip bounds).
\url{https://dlmf.nist.gov/}

\bibitem{DurenHp}
P.~L.~Duren, \emph{Theory of $H^p$ Spaces}, Academic Press, 1970.

\bibitem{GarnettBAF}
J.~B.~Garnett, \emph{Bounded Analytic Functions}, Springer, 2007.

\bibitem{GarnettMarshall}
J.~B.~Garnett and D.~E.~Marshall, \emph{Harmonic Measure}, Cambridge Univ. Press, 2005.

\bibitem{Ivic}
A.~Ivi\'c, \emph{The Riemann Zeta-Function}, John Wiley \& Sons, 1985.

\bibitem{Kellogg}
O.~D.~Kellogg, \emph{Foundations of Potential Theory}, Dover, 1953.

\bibitem{Platt2017}
D.~J.~Platt, Isolating some nontrivial zeros of $\zeta(s)$, \emph{Math. Comp.} \textbf{86} (2017), 2449–2467.

\bibitem{PlattTrudgian2021}
D.~J.\,Platt and T.\,S.~Trudgian, The Riemann hypothesis is true up to $3\cdot 10^{12}$,
\emph{Bull. Lond. Math. Soc.} \textbf{53} (2021), 792–797.

\bibitem{Pommerenke}
Ch.~Pommerenke, \emph{Boundary Behaviour of Conformal Maps}, Springer, 1992.

\bibitem{Ransford}
T.~Ransford, \emph{Potential Theory in the Complex Plane}, Cambridge Univ. Press, 1995.

\bibitem{Titchmarsh}
E.~C.~Titchmarsh (rev. D.~R.~Heath--Brown), \emph{The Theory of the Riemann Zeta-Function}, 2nd ed., Oxford, 1986.

\end{thebibliography}

\end{document}
